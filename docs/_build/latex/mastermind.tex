%% Generated by Sphinx.
\def\sphinxdocclass{report}
\documentclass[letterpaper,10pt,italian]{sphinxmanual}
\ifdefined\pdfpxdimen
   \let\sphinxpxdimen\pdfpxdimen\else\newdimen\sphinxpxdimen
\fi \sphinxpxdimen=.75bp\relax

\PassOptionsToPackage{warn}{textcomp}
\usepackage[utf8]{inputenc}
\ifdefined\DeclareUnicodeCharacter
% support both utf8 and utf8x syntaxes
  \ifdefined\DeclareUnicodeCharacterAsOptional
    \def\sphinxDUC#1{\DeclareUnicodeCharacter{"#1}}
  \else
    \let\sphinxDUC\DeclareUnicodeCharacter
  \fi
  \sphinxDUC{00A0}{\nobreakspace}
  \sphinxDUC{2500}{\sphinxunichar{2500}}
  \sphinxDUC{2502}{\sphinxunichar{2502}}
  \sphinxDUC{2514}{\sphinxunichar{2514}}
  \sphinxDUC{251C}{\sphinxunichar{251C}}
  \sphinxDUC{2572}{\textbackslash}
\fi
\usepackage{cmap}
\usepackage[T1]{fontenc}
\usepackage{amsmath,amssymb,amstext}
\usepackage{babel}



\usepackage{times}
\expandafter\ifx\csname T@LGR\endcsname\relax
\else
% LGR was declared as font encoding
  \substitutefont{LGR}{\rmdefault}{cmr}
  \substitutefont{LGR}{\sfdefault}{cmss}
  \substitutefont{LGR}{\ttdefault}{cmtt}
\fi
\expandafter\ifx\csname T@X2\endcsname\relax
  \expandafter\ifx\csname T@T2A\endcsname\relax
  \else
  % T2A was declared as font encoding
    \substitutefont{T2A}{\rmdefault}{cmr}
    \substitutefont{T2A}{\sfdefault}{cmss}
    \substitutefont{T2A}{\ttdefault}{cmtt}
  \fi
\else
% X2 was declared as font encoding
  \substitutefont{X2}{\rmdefault}{cmr}
  \substitutefont{X2}{\sfdefault}{cmss}
  \substitutefont{X2}{\ttdefault}{cmtt}
\fi


\usepackage[Sonny]{fncychap}
\ChNameVar{\Large\normalfont\sffamily}
\ChTitleVar{\Large\normalfont\sffamily}
\usepackage{sphinx}

\fvset{fontsize=\small}
\usepackage{geometry}

% Include hyperref last.
\usepackage{hyperref}
% Fix anchor placement for figures with captions.
\usepackage{hypcap}% it must be loaded after hyperref.
% Set up styles of URL: it should be placed after hyperref.
\urlstyle{same}

\usepackage{sphinxmessages}
\setcounter{tocdepth}{1}



\title{MasterMind}
\date{06 giu 2019}
\release{1.0.0}
\author{Francesco Pio Stelluti, Francesco Coppola}
\newcommand{\sphinxlogo}{\vbox{}}
\renewcommand{\releasename}{Release}
\makeindex
\begin{document}

\ifdefined\shorthandoff
  \ifnum\catcode`\=\string=\active\shorthandoff{=}\fi
  \ifnum\catcode`\"=\active\shorthandoff{"}\fi
\fi

\pagestyle{empty}
\sphinxmaketitle
\pagestyle{plain}
\sphinxtableofcontents
\pagestyle{normal}
\phantomsection\label{\detokenize{index::doc}}

\begin{quote}

\sphinxstyleemphasis{«Our education might stop, if we so choose. Our brains’ never does.
The brain will keep reacting to how we decide to use it. The difference
is not whether or not we learn, but what and how we learn.»}
\begin{quote}

\sphinxstyleemphasis{Maria Konnikova} %
\begin{footnote}[1]\sphinxAtStartFootnote
\sphinxhref{https://en.wikipedia.org/wiki/Mastermind:\_How\_to\_Think\_Like\_Sherlock\_Holmes}{Maria Konnikova, Mastermind: How to Think Like Sherlock Holmes}
%
\end{footnote}
\end{quote}
\end{quote}

All’interno delle seguenti pagine sarà possibile trovare la documentazione
generata per il progetto \sphinxstylestrong{MasterMind}, realizzato per il corso di \sphinxstyleemphasis{Programmazione Avanzata}
dell’anno 2018/2019.

Lo sviluppo di tale codice è da attribuire interamente agli studenti \sphinxstylestrong{Francesco Pio Stelluti} e \sphinxstylestrong{Francesco Coppola}.


\chapter{Introduzione}
\label{\detokenize{introduzione:introduzione}}\label{\detokenize{introduzione::doc}}
Il progetto è stato indirizzato ad all’implementazione tramite linguaggio \sphinxstylestrong{Java}
del gioco da tavolo \sphinxstylestrong{Mastermind} %
\begin{footnote}[1]\sphinxAtStartFootnote
\sphinxhref{https://it.wikipedia.org/wiki/Mastermind}{Mastermind}
%
\end{footnote}. Nell’ideare la struttura del progetto si è puntato
alla \sphinxstylestrong{massima modularità possibile}, per quanto non totale, ottenuta tramite l’applicazione
di determinati design pattern.


\section{Struttura fondamentale del progetto}
\label{\detokenize{introduzione:struttura-fondamentale-del-progetto}}
\noindent\sphinxincludegraphics{{MasterMind}.png}

L’idea alla base della struttura del gioco riguarda le \sphinxstyleemphasis{interazioni} tra l’utente umano ed un’istanza
di una classe che estende \sphinxcode{\sphinxupquote{StartView}}. Tramite questa interazione è possibile decidere quali impostazioni
e quali implementazioni dei giocatori, rispettivamente un \sphinxcode{\sphinxupquote{CodeBreaker}} ed un \sphinxcode{\sphinxupquote{CodeMaker}},
impiegare all’interno di singole partite. I giocatori potranno poi interagire all’interno della partita
comunicando con una istanza di una classe che estende \sphinxcode{\sphinxupquote{InteractionView}}, dalla quale ottengono informazioni
sulla partita in corso e grazie alla quale hanno una possibile interazione con l’utente umano, e con
una istanza di \sphinxcode{\sphinxupquote{BoardController}}, alla quale \sphinxstylestrong{comunicano} decisioni di gioco quali la sequenza da indovinare o le sequenze
valide come tentativo per poter indovinare tale sequenza.

Alla base della sequenza, a rappresentazione dei pioli impiegati nel gioco originale, sono presenti valori della
classe enum \sphinxcode{\sphinxupquote{ColorPegs}}, contenente otto colori.


\section{Estendibilità ed implementazioni fornite di default}
\label{\detokenize{introduzione:estendibilita-ed-implementazioni-fornite-di-default}}
Si è deciso di adottare una struttura molto rigida per quanto riguarda la rappresentazione dei pioli e della plancia
di gioco, non offrendo possibilità di aggiungere ulteriori implementazioni o diversificazioni di quelle che sono le classi
\sphinxstylestrong{ColorPegs}, \sphinxstylestrong{BoardModel} e \sphinxstylestrong{BoardCoordinator}.
Diverso il discorso sul piano delle implementazioni di giocatori o delle interfacce di comunicazione con l’utente umano.
È infatti possibile aggiungere classi che estendono \sphinxcode{\sphinxupquote{CodeMaker}} e \sphinxcode{\sphinxupquote{CodeBreaker}}, fornendo anche le relative classi \sphinxstyleemphasis{factory} che estendono
rispettivamente \sphinxcode{\sphinxupquote{MakerFactory}} e \sphinxcode{\sphinxupquote{BreakerFactory}}, senza che il codice venga ricompilato.
Per fare ciò si è deciso di implementare una classe astratta \sphinxcode{\sphinxupquote{PlayerFactoryRegistry}}, estesa nel progetto in questione
da \sphinxcode{\sphinxupquote{MakerFactoryRegistry}} e \sphinxcode{\sphinxupquote{BreakerFactoryRegistry}}, classi che permettono di \sphinxstylestrong{collezionare a runtime} informazioni
riguardo le factory puntate a generare istannze di classi estensione di \sphinxcode{\sphinxupquote{CodeMaker}} e \sphinxcode{\sphinxupquote{CodeBreaker}}.
Analogamente è possibile aggiungere classi estensione di \sphinxcode{\sphinxupquote{StartView}} per fornire particolari \sphinxstyleemphasis{viste} indirizzate \sphinxstylestrong{all’interazione}
con l’utente fisico durante l’impostazione e l’avvio di nuove partite. Ad ogni \sphinxcode{\sphinxupquote{StartView}} si richiede di associare anche una classe
che estenda \sphinxcode{\sphinxupquote{InteractionView}} che sia \sphinxstylestrong{coerente} con la particolare estensione di StartView trattata e di includere il metodo \sphinxstylestrong{main} per permettere
l’avvio effettivo del programma.

Di default sono fornite delle implementazioni di quelle che sono le classi rappresentanti i giocatori e l’interazione con l’utente umano:
\begin{itemize}
\item {} 
\sphinxstylestrong{ConsoleStartView}: estensione di \sphinxcode{\sphinxupquote{StartView}}, fornisce un’interazione con l’utente fisico per l’impostazione e l’avvio di nuove partite tramite console.

\item {} 
\sphinxstylestrong{ConsoleInteractionView}: estensione di \sphinxcode{\sphinxupquote{InteractionView}}, è strettamente associata con la classe \sphinxcode{\sphinxupquote{ConsoleStartView}} fornisce un’interazione con l’utente fisico in caso siano necessarie per \sphinxstyleemphasis{impartire} nuove decisioni durante lo svolgimento di una partita.

\item {} 
\sphinxstylestrong{InteractiveMaker}: estensione di \sphinxcode{\sphinxupquote{CodeMaker}}, fornisce l’implementazione di un giocatore comandato dall’utente umano attraverso l’interazione fornita da una classe estensione di \sphinxcode{\sphinxupquote{InteractionView}}. È possibile ottenere istanze di questa estensione tramite la classe \sphinxcode{\sphinxupquote{InteractiveMakerFactory}}.

\item {} 
\sphinxstylestrong{InteractiveBreaker}: estensione di \sphinxcode{\sphinxupquote{CodeBreaker}}, fornisce l’implementazione di un giocatore comandato dall’utente umano attraverso l’interazione fornita da una classe estensione di \sphinxcode{\sphinxupquote{InteractionView}}. È possibile ottenere istanze di questa estensione tramite la classe \sphinxcode{\sphinxupquote{InteractiveBreakerFactory}}.

\item {} 
\sphinxstylestrong{RandomBotMaker}: estensione di \sphinxcode{\sphinxupquote{CodeMaker}}, fornisce l’implementazione di un giocatore comandato da un \sphinxstylestrong{IA} che agisce fornendo sequenze randomiche. È possibile ottenere istanze di questa estensione tramite la classe \sphinxcode{\sphinxupquote{RandomBotMakerFactory}}.

\item {} 
\sphinxstylestrong{RandomBotBreaker}: estensione di \sphinxcode{\sphinxupquote{CodeBreaker}}, fornisce l’implementazione di un giocatore comandato da un \sphinxstylestrong{IA} che agisce fornendo sequenze randomiche. È possibile ottenere istanze di questa estensione tramite la classe \sphinxcode{\sphinxupquote{RandomBotBreakerFactory}}.

\end{itemize}

Per ulteriori informazioni circa le classi elencate si rimanda alle relative {\hyperref[\detokenize{source/packages::doc}]{\sphinxcrossref{\DUrole{doc}{sezioni}}}}.


\section{Informazioni fondamentali circa il primo avvio}
\label{\detokenize{introduzione:informazioni-fondamentali-circa-il-primo-avvio}}
Il caricamento a \sphinxstylestrong{runtime} delle informazioni relative alle classi factory grazie alle quali ottenere istanze di classi che estendono
\sphinxcode{\sphinxupquote{CodeBreaker}} e \sphinxcode{\sphinxupquote{CodeMaker}} è stato reso possibile grazie alla lettura di specifici file testuali. In loro assenza il software creerà
dei file standard, comunicando all’utente questa decisione, da modificare \sphinxstylestrong{obbligatoriamente} con le giuste informazioni per avere un
corretto avvio ed una corretta esecuzione del programma.


\section{Responsabilità delle classi}
\label{\detokenize{introduzione:responsabilita-delle-classi}}
Si rimanda alle {\hyperref[\detokenize{source/packages::doc}]{\sphinxcrossref{\DUrole{doc}{sezioni}}}} riguardanti le implementazioni delle singole classi per ulteriori informazioni.


\section{Design pattern impiegati}
\label{\detokenize{introduzione:design-pattern-impiegati}}
1. \sphinxstylestrong{Model View Controller} %
\begin{footnote}[2]\sphinxAtStartFootnote
\sphinxhref{https://it.wikipedia.org/wiki/Model-view-controller}{MVC}
%
\end{footnote}
Rappresenta la struttura alla base dell’intero gioco. È stata implementata tramite le classi \sphinxcode{\sphinxupquote{StartView}}, \sphinxcode{\sphinxupquote{InteractionView}}, \sphinxcode{\sphinxupquote{BoardModel}} e \sphinxcode{\sphinxupquote{BoardCoordinator}}.

2. \sphinxstylestrong{Observer} %
\begin{footnote}[3]\sphinxAtStartFootnote
\sphinxhref{https://italiancoders.it/observer-pattern/}{Observer}
%
\end{footnote}
Implementato fornendo come classe da osservare \sphinxcode{\sphinxupquote{BoardModel}} e come classi che osservano \sphinxcode{\sphinxupquote{InteractionView}} e \sphinxcode{\sphinxupquote{CurrentGameStats}}. Dalla versione 9 di Java l’interfaccia Observer, pensata nell’ottica di questo design pattern, risulta deprecata. La sua implementazione è quindi da vedere in un’ottica puramente accademica e finalizzata all’apprendimento del concetto alla base del pattern.

3. \sphinxstylestrong{Singleton} %
\begin{footnote}[4]\sphinxAtStartFootnote
\sphinxhref{https://it.wikipedia.org/wiki/Singleton}{Singleton}
%
\end{footnote}
Presente all’interno delle classi \sphinxcode{\sphinxupquote{ConsoleStartView}} e \sphinxcode{\sphinxupquote{ConsoleInteractionView}}, esso garantisce che siano presenti \sphinxstylestrong{singole} istanze di tali classi all’interno del progetto.

4. \sphinxstylestrong{Factory} %
\begin{footnote}[5]\sphinxAtStartFootnote
\sphinxhref{https://italiancoders.it/factory-method-design-pattern/}{Factory}
%
\end{footnote}
Implementato tramite l’interfaccia \sphinxcode{\sphinxupquote{PlayerFactory}}, implementata da \sphinxcode{\sphinxupquote{BreakerFactory}} e \sphinxcode{\sphinxupquote{MakerFactory}}, \sphinxstylestrong{classi astratte} da estendere tramite classi factory che forniscano istanze di classi estensione rispettivamente di \sphinxcode{\sphinxupquote{CodeBreaker}} e \sphinxcode{\sphinxupquote{CodeMaker}}.


\section{Testing}
\label{\detokenize{introduzione:testing}}
Sono stati ideati dei test, scritti sotto ambiente \sphinxstylestrong{JUnit 5} %
\begin{footnote}[6]\sphinxAtStartFootnote
\sphinxhref{https://junit.org/junit5}{JUnit}
%
\end{footnote}, per poter testare in modo mirato le singole \sphinxstyleemphasis{funzionalità} del progetto.
Per ulteriori informazioni si rimanda alle {\hyperref[\detokenize{test/packages::doc}]{\sphinxcrossref{\DUrole{doc}{sezioni}}}}  riguardanti le implementazioni di tali test.


\section{Gradle}
\label{\detokenize{introduzione:gradle}}
Nell’ottica di garantire continuità al progetto si è deciso anche di implementare il tool di building \sphinxstylestrong{Gradle} %
\begin{footnote}[7]\sphinxAtStartFootnote
\sphinxhref{https://gradle.org/}{Gradle}
%
\end{footnote}, in versione 5.1.1,
per facilitare il deploy e la distribuzione di tale software all’interno di altri sistemi.


\chapter{Sphinx e le sue potenzialità}
\label{\detokenize{sphinx:sphinx-e-le-sue-potenzialita}}\label{\detokenize{sphinx::doc}}
L’intera documentazione generata della quale si sta usufruendo è frutto dell’unione tra Sphinx e JavaDoc, due strumenti
dedicati alla generazione di testi a partire da del mero e puro codice.


\section{Strumenti con cui è stata realizzata}
\label{\detokenize{sphinx:strumenti-con-cui-e-stata-realizzata}}
Solitamente per documentare in maniera \sphinxstyleemphasis{raffinata} un progetto \sphinxstylestrong{Java} viene utilizzato
lo strumento fornito dall’IDE di sviluppo stesso \sphinxstylestrong{JavaDoc} %
\begin{footnote}[1]\sphinxAtStartFootnote
\sphinxstyleemphasis{Javadoc è un applicativo incluso nel Java Development Kit della Sun Microsystems, utilizzato per la generazione automatica della documentazione del codice sorgente scritto in linguaggio Java}
%
\end{footnote}.

Esso offre degli incredibili vantaggi, come la facilità d’utilizzo e soprattutto un
layout ben noto all’interno della community dei developers Java che permette
di trovare informazioni in maniera decisamente veloce.

La pecca più grande di tale strumento però resta la datazione dei vari stili che compongono
i file CSS e l’assenza di un’eleganza generale complessiva.

Per risolvere tale mancanza quindi si è pensato di ricorrere a \sphinxstylestrong{Sphinx} %
\begin{footnote}[2]\sphinxAtStartFootnote
\sphinxstyleemphasis{Software Open Source per l’autogenerazione di documentazioni a partire da un codice sorgente generico}
%
\end{footnote}.

Poi mediante l’utilizzo di un’estensione nominata \sphinxcode{\sphinxupquote{javasphinx}} %
\begin{footnote}[3]\sphinxAtStartFootnote
\sphinxhref{https://bronto.github.io/javasphinx/}{Javasphinx}
%
\end{footnote} è stato possibile
convertire i vari commenti \sphinxstylestrong{JavaDoc} secondo lo standard perseguito da Sphinx stesso, e così
facendo abbiamo ottenuto sia una documentazione piacevole per la vista che
facile ed intutiva da poter seguire.


\section{Autogenerazione della sintassi convertita da JavaDoc a Sphinx}
\label{\detokenize{sphinx:autogenerazione-della-sintassi-convertita-da-javadoc-a-sphinx}}
Per fare questa operazione è necessario innanzitutto installare \sphinxcode{\sphinxupquote{javasphinx}}
sulla propria macchina, attraverso il seguente comando:

\begin{sphinxVerbatim}[commandchars=\\\{\}]
\PYGZdl{} pip install javasphinx
\end{sphinxVerbatim}

Una volta effettuato ciò sarà necessario inserire l’estensione \sphinxcode{\sphinxupquote{javasphinx}} appena installata
nel file \sphinxcode{\sphinxupquote{conf.py}} generato da Sphinx.

A questo bisognerà definire lo standard Java da seguire, all’interno del file
\sphinxcode{\sphinxupquote{conf.py}}, nel seguente modo:

\sphinxcode{\sphinxupquote{javadoc\_url\_map = \{ '\textless{}namespace\_here\textgreater{}' : ('\textless{}base\_url\_here\textgreater{}', 'javadoc') \}}}

Arrivati a questo punto basterà lanciare il comando:

\begin{sphinxVerbatim}[commandchars=\\\{\}]
\PYGZdl{} javasphinx\PYGZhy{}apidoc \PYGZhy{}o docs/source/ \PYGZhy{}\PYGZhy{}title=\PYGZsq{}\PYGZlt{}name\PYGZus{}here\PYGZgt{}\PYGZsq{} ../path/to/java\PYGZus{}dirtoscan
\end{sphinxVerbatim}

La documentazione quindi sarà pronta per essere usata nei vari file con estensione \sphinxcode{\sphinxupquote{.rst}} che, attraverso il comando \sphinxcode{\sphinxupquote{make}}, diventaranno file \sphinxcode{\sphinxupquote{.html}}.


\chapter{Documentazione del codice}
\label{\detokenize{source/packages:documentazione-del-codice}}\label{\detokenize{source/packages::doc}}
Nella seguente pagina sarà possibile accedere alle informazioni che descrivono in maniera \sphinxstyleemphasis{dettagliata} ogni
\sphinxstylestrong{classe} ed ogni \sphinxstylestrong{package} appartenente al parco software prodotto.

Per rendere più chiara la composizione della struttura del progetto, e quindi comprendere più dettagliatamente quello
che è stato realizzato, abbiamo reso disponibile un \sphinxstylestrong{diagramma UML} il quale è possibile visualizzare qui di seguito.

\noindent\sphinxincludegraphics{{diagram}.png}


\section{it.unicam.cs.pa.mastermind.factories}
\label{\detokenize{source/it/unicam/cs/pa/mastermind/factories/package-index:it-unicam-cs-pa-mastermind-factories}}\label{\detokenize{source/it/unicam/cs/pa/mastermind/factories/package-index::doc}}
Il package contiene le varie factory che hanno il compito di generare nuovi player durante il processo di esecuzione in maniera dinamica ed efficiente.

\phantomsection\label{\detokenize{source/it/unicam/cs/pa/mastermind/factories/package-index:package-it.unicam.cs.pa.mastermind.factories}}\index{it.unicam.cs.pa.mastermind.factories (package)@\spxentry{it.unicam.cs.pa.mastermind.factories}\spxextra{package}}

\subsection{BreakerFactory}
\label{\detokenize{source/it/unicam/cs/pa/mastermind/factories/BreakerFactory:breakerfactory}}\label{\detokenize{source/it/unicam/cs/pa/mastermind/factories/BreakerFactory::doc}}\index{BreakerFactory (Java class)@\spxentry{BreakerFactory}\spxextra{Java class}}

\begin{fulllineitems}
\phantomsection\label{\detokenize{source/it/unicam/cs/pa/mastermind/factories/BreakerFactory:it.unicam.cs.pa.mastermind.factories.BreakerFactory}}\pysigline{public abstract class \sphinxbfcode{\sphinxupquote{BreakerFactory}} implements {\hyperref[\detokenize{source/it/unicam/cs/pa/mastermind/factories/PlayerFactory:it.unicam.cs.pa.mastermind.factories.PlayerFactory}]{\sphinxcrossref{PlayerFactory}}}}
Classe factory astratta estensione di \sphinxcode{\sphinxupquote{PlayerFactory}} da estendere con classi factory concrete finalizzate all’ottenimento di istanze di \sphinxcode{\sphinxupquote{CodeBreaker}}.
\begin{quote}\begin{description}
\item[{Author}] \leavevmode
Francesco Pio Stelluti, Francesco Coppola

\end{description}\end{quote}

\end{fulllineitems}



\subsubsection{Methods}
\label{\detokenize{source/it/unicam/cs/pa/mastermind/factories/BreakerFactory:methods}}

\paragraph{getBreaker}
\label{\detokenize{source/it/unicam/cs/pa/mastermind/factories/BreakerFactory:getbreaker}}\index{getBreaker() (Java method)@\spxentry{getBreaker()}\spxextra{Java method}}

\begin{fulllineitems}
\phantomsection\label{\detokenize{source/it/unicam/cs/pa/mastermind/factories/BreakerFactory:it.unicam.cs.pa.mastermind.factories.BreakerFactory.getBreaker()}}\pysiglinewithargsret{public abstract {\hyperref[\detokenize{source/it/unicam/cs/pa/mastermind/players/CodeBreaker:it.unicam.cs.pa.mastermind.players.CodeBreaker}]{\sphinxcrossref{CodeBreaker}}} \sphinxbfcode{\sphinxupquote{getBreaker}}}{}{}
\end{fulllineitems}



\subsection{InteractiveBreakerFactory}
\label{\detokenize{source/it/unicam/cs/pa/mastermind/factories/InteractiveBreakerFactory:interactivebreakerfactory}}\label{\detokenize{source/it/unicam/cs/pa/mastermind/factories/InteractiveBreakerFactory::doc}}\index{InteractiveBreakerFactory (Java class)@\spxentry{InteractiveBreakerFactory}\spxextra{Java class}}

\begin{fulllineitems}
\phantomsection\label{\detokenize{source/it/unicam/cs/pa/mastermind/factories/InteractiveBreakerFactory:it.unicam.cs.pa.mastermind.factories.InteractiveBreakerFactory}}\pysigline{public class \sphinxbfcode{\sphinxupquote{InteractiveBreakerFactory}} extends {\hyperref[\detokenize{source/it/unicam/cs/pa/mastermind/factories/BreakerFactory:it.unicam.cs.pa.mastermind.factories.BreakerFactory}]{\sphinxcrossref{BreakerFactory}}}}
Classe factory estensione di \sphinxcode{\sphinxupquote{BreakerFactory}} impiegata per ottenere istanze di \sphinxcode{\sphinxupquote{InteractiveBreaker}}.
\begin{quote}\begin{description}
\item[{Author}] \leavevmode
Francesco Pio Stelluti, Francesco Coppola

\end{description}\end{quote}

\end{fulllineitems}



\subsubsection{Methods}
\label{\detokenize{source/it/unicam/cs/pa/mastermind/factories/InteractiveBreakerFactory:methods}}

\paragraph{getBreaker}
\label{\detokenize{source/it/unicam/cs/pa/mastermind/factories/InteractiveBreakerFactory:getbreaker}}\index{getBreaker() (Java method)@\spxentry{getBreaker()}\spxextra{Java method}}

\begin{fulllineitems}
\phantomsection\label{\detokenize{source/it/unicam/cs/pa/mastermind/factories/InteractiveBreakerFactory:it.unicam.cs.pa.mastermind.factories.InteractiveBreakerFactory.getBreaker()}}\pysiglinewithargsret{public {\hyperref[\detokenize{source/it/unicam/cs/pa/mastermind/players/CodeBreaker:it.unicam.cs.pa.mastermind.players.CodeBreaker}]{\sphinxcrossref{CodeBreaker}}} \sphinxbfcode{\sphinxupquote{getBreaker}}}{}{}
\end{fulllineitems}



\subsection{InteractiveMakerFactory}
\label{\detokenize{source/it/unicam/cs/pa/mastermind/factories/InteractiveMakerFactory:interactivemakerfactory}}\label{\detokenize{source/it/unicam/cs/pa/mastermind/factories/InteractiveMakerFactory::doc}}\index{InteractiveMakerFactory (Java class)@\spxentry{InteractiveMakerFactory}\spxextra{Java class}}

\begin{fulllineitems}
\phantomsection\label{\detokenize{source/it/unicam/cs/pa/mastermind/factories/InteractiveMakerFactory:it.unicam.cs.pa.mastermind.factories.InteractiveMakerFactory}}\pysigline{public class \sphinxbfcode{\sphinxupquote{InteractiveMakerFactory}} extends {\hyperref[\detokenize{source/it/unicam/cs/pa/mastermind/factories/MakerFactory:it.unicam.cs.pa.mastermind.factories.MakerFactory}]{\sphinxcrossref{MakerFactory}}}}
Classe factory estensione di \sphinxcode{\sphinxupquote{MakerFactory}} impiegata per ottenere istanze di \sphinxcode{\sphinxupquote{InteractiveMaker}}.
\begin{quote}\begin{description}
\item[{Author}] \leavevmode
Francesco Pio Stelluti, Francesco Coppola

\end{description}\end{quote}

\end{fulllineitems}



\subsubsection{Methods}
\label{\detokenize{source/it/unicam/cs/pa/mastermind/factories/InteractiveMakerFactory:methods}}

\paragraph{getMaker}
\label{\detokenize{source/it/unicam/cs/pa/mastermind/factories/InteractiveMakerFactory:getmaker}}\index{getMaker() (Java method)@\spxentry{getMaker()}\spxextra{Java method}}

\begin{fulllineitems}
\phantomsection\label{\detokenize{source/it/unicam/cs/pa/mastermind/factories/InteractiveMakerFactory:it.unicam.cs.pa.mastermind.factories.InteractiveMakerFactory.getMaker()}}\pysiglinewithargsret{public {\hyperref[\detokenize{source/it/unicam/cs/pa/mastermind/players/CodeMaker:it.unicam.cs.pa.mastermind.players.CodeMaker}]{\sphinxcrossref{CodeMaker}}} \sphinxbfcode{\sphinxupquote{getMaker}}}{}{}
\end{fulllineitems}



\subsection{MakerFactory}
\label{\detokenize{source/it/unicam/cs/pa/mastermind/factories/MakerFactory:makerfactory}}\label{\detokenize{source/it/unicam/cs/pa/mastermind/factories/MakerFactory::doc}}\index{MakerFactory (Java class)@\spxentry{MakerFactory}\spxextra{Java class}}

\begin{fulllineitems}
\phantomsection\label{\detokenize{source/it/unicam/cs/pa/mastermind/factories/MakerFactory:it.unicam.cs.pa.mastermind.factories.MakerFactory}}\pysigline{public abstract class \sphinxbfcode{\sphinxupquote{MakerFactory}} implements {\hyperref[\detokenize{source/it/unicam/cs/pa/mastermind/factories/PlayerFactory:it.unicam.cs.pa.mastermind.factories.PlayerFactory}]{\sphinxcrossref{PlayerFactory}}}}
Classe factory astratta estensione di \sphinxcode{\sphinxupquote{PlayerFactory}} da estendere con classi factory concrete finalizzate all’ottenimento di istanze di \sphinxcode{\sphinxupquote{CodeMaker}}.
\begin{quote}\begin{description}
\item[{Author}] \leavevmode
Francesco Pio Stelluti, Francesco Coppola

\end{description}\end{quote}

\end{fulllineitems}



\subsubsection{Methods}
\label{\detokenize{source/it/unicam/cs/pa/mastermind/factories/MakerFactory:methods}}

\paragraph{getMaker}
\label{\detokenize{source/it/unicam/cs/pa/mastermind/factories/MakerFactory:getmaker}}\index{getMaker() (Java method)@\spxentry{getMaker()}\spxextra{Java method}}

\begin{fulllineitems}
\phantomsection\label{\detokenize{source/it/unicam/cs/pa/mastermind/factories/MakerFactory:it.unicam.cs.pa.mastermind.factories.MakerFactory.getMaker()}}\pysiglinewithargsret{public abstract {\hyperref[\detokenize{source/it/unicam/cs/pa/mastermind/players/CodeMaker:it.unicam.cs.pa.mastermind.players.CodeMaker}]{\sphinxcrossref{CodeMaker}}} \sphinxbfcode{\sphinxupquote{getMaker}}}{}{}
\end{fulllineitems}



\subsection{PlayerFactory}
\label{\detokenize{source/it/unicam/cs/pa/mastermind/factories/PlayerFactory:playerfactory}}\label{\detokenize{source/it/unicam/cs/pa/mastermind/factories/PlayerFactory::doc}}\index{PlayerFactory (Java interface)@\spxentry{PlayerFactory}\spxextra{Java interface}}

\begin{fulllineitems}
\phantomsection\label{\detokenize{source/it/unicam/cs/pa/mastermind/factories/PlayerFactory:it.unicam.cs.pa.mastermind.factories.PlayerFactory}}\pysigline{public interface \sphinxbfcode{\sphinxupquote{PlayerFactory}}}
Interfaccia \sphinxcode{\sphinxupquote{PlayerFactory}} che consente l’implementazione da parte delle altri classi ed usata quale label da applicare alle classi che la implementano.
\begin{quote}\begin{description}
\item[{Author}] \leavevmode
Francesco Pio Stelluti, Francesco Coppola

\end{description}\end{quote}

\end{fulllineitems}



\subsection{RandomBotBreakerFactory}
\label{\detokenize{source/it/unicam/cs/pa/mastermind/factories/RandomBotBreakerFactory:randombotbreakerfactory}}\label{\detokenize{source/it/unicam/cs/pa/mastermind/factories/RandomBotBreakerFactory::doc}}\index{RandomBotBreakerFactory (Java class)@\spxentry{RandomBotBreakerFactory}\spxextra{Java class}}

\begin{fulllineitems}
\phantomsection\label{\detokenize{source/it/unicam/cs/pa/mastermind/factories/RandomBotBreakerFactory:it.unicam.cs.pa.mastermind.factories.RandomBotBreakerFactory}}\pysigline{public class \sphinxbfcode{\sphinxupquote{RandomBotBreakerFactory}} extends {\hyperref[\detokenize{source/it/unicam/cs/pa/mastermind/factories/BreakerFactory:it.unicam.cs.pa.mastermind.factories.BreakerFactory}]{\sphinxcrossref{BreakerFactory}}}}
Classe factory estensione di \sphinxcode{\sphinxupquote{BreakerFactory}} impiegata per ottenere istanze di \sphinxcode{\sphinxupquote{RandomBotBreaker}}.
\begin{quote}\begin{description}
\item[{Author}] \leavevmode
Francesco Pio Stelluti, Francesco Coppola

\end{description}\end{quote}

\end{fulllineitems}



\subsubsection{Methods}
\label{\detokenize{source/it/unicam/cs/pa/mastermind/factories/RandomBotBreakerFactory:methods}}

\paragraph{getBreaker}
\label{\detokenize{source/it/unicam/cs/pa/mastermind/factories/RandomBotBreakerFactory:getbreaker}}\index{getBreaker() (Java method)@\spxentry{getBreaker()}\spxextra{Java method}}

\begin{fulllineitems}
\phantomsection\label{\detokenize{source/it/unicam/cs/pa/mastermind/factories/RandomBotBreakerFactory:it.unicam.cs.pa.mastermind.factories.RandomBotBreakerFactory.getBreaker()}}\pysiglinewithargsret{public {\hyperref[\detokenize{source/it/unicam/cs/pa/mastermind/players/CodeBreaker:it.unicam.cs.pa.mastermind.players.CodeBreaker}]{\sphinxcrossref{CodeBreaker}}} \sphinxbfcode{\sphinxupquote{getBreaker}}}{}{}
\end{fulllineitems}



\subsection{RandomBotMakerFactory}
\label{\detokenize{source/it/unicam/cs/pa/mastermind/factories/RandomBotMakerFactory:randombotmakerfactory}}\label{\detokenize{source/it/unicam/cs/pa/mastermind/factories/RandomBotMakerFactory::doc}}\index{RandomBotMakerFactory (Java class)@\spxentry{RandomBotMakerFactory}\spxextra{Java class}}

\begin{fulllineitems}
\phantomsection\label{\detokenize{source/it/unicam/cs/pa/mastermind/factories/RandomBotMakerFactory:it.unicam.cs.pa.mastermind.factories.RandomBotMakerFactory}}\pysigline{public class \sphinxbfcode{\sphinxupquote{RandomBotMakerFactory}} extends {\hyperref[\detokenize{source/it/unicam/cs/pa/mastermind/factories/MakerFactory:it.unicam.cs.pa.mastermind.factories.MakerFactory}]{\sphinxcrossref{MakerFactory}}}}
Classe factory estensione di \sphinxcode{\sphinxupquote{MakerFactory}} impiegata per ottenere istanze di \sphinxcode{\sphinxupquote{RandomBotMaker}}.
\begin{quote}\begin{description}
\item[{Author}] \leavevmode
Francesco Pio Stelluti, Francesco Coppola

\end{description}\end{quote}

\end{fulllineitems}



\subsubsection{Methods}
\label{\detokenize{source/it/unicam/cs/pa/mastermind/factories/RandomBotMakerFactory:methods}}

\paragraph{getMaker}
\label{\detokenize{source/it/unicam/cs/pa/mastermind/factories/RandomBotMakerFactory:getmaker}}\index{getMaker() (Java method)@\spxentry{getMaker()}\spxextra{Java method}}

\begin{fulllineitems}
\phantomsection\label{\detokenize{source/it/unicam/cs/pa/mastermind/factories/RandomBotMakerFactory:it.unicam.cs.pa.mastermind.factories.RandomBotMakerFactory.getMaker()}}\pysiglinewithargsret{public {\hyperref[\detokenize{source/it/unicam/cs/pa/mastermind/players/CodeMaker:it.unicam.cs.pa.mastermind.players.CodeMaker}]{\sphinxcrossref{CodeMaker}}} \sphinxbfcode{\sphinxupquote{getMaker}}}{}{}
\end{fulllineitems}



\section{it.unicam.cs.pa.mastermind.gamecore}
\label{\detokenize{source/it/unicam/cs/pa/mastermind/gamecore/package-index:it-unicam-cs-pa-mastermind-gamecore}}\label{\detokenize{source/it/unicam/cs/pa/mastermind/gamecore/package-index::doc}}
Il package contiene le componenti chiave relative ad una singola partita, quali la plancia di gioco, il controllore di tale plancia e l’istanza della partita correntemente attiva.

\phantomsection\label{\detokenize{source/it/unicam/cs/pa/mastermind/gamecore/package-index:package-it.unicam.cs.pa.mastermind.gamecore}}\index{it.unicam.cs.pa.mastermind.gamecore (package)@\spxentry{it.unicam.cs.pa.mastermind.gamecore}\spxextra{package}}

\subsection{BoardController}
\label{\detokenize{source/it/unicam/cs/pa/mastermind/gamecore/BoardController:boardcontroller}}\label{\detokenize{source/it/unicam/cs/pa/mastermind/gamecore/BoardController::doc}}\index{BoardController (Java class)@\spxentry{BoardController}\spxextra{Java class}}

\begin{fulllineitems}
\phantomsection\label{\detokenize{source/it/unicam/cs/pa/mastermind/gamecore/BoardController:it.unicam.cs.pa.mastermind.gamecore.BoardController}}\pysigline{public class \sphinxbfcode{\sphinxupquote{BoardController}}}
\sphinxstylestrong{Responsabilità}: gestire le interazioni dall’esterno e dirette alla modifica di un’istanza \sphinxcode{\sphinxupquote{BoardModel}}.
\begin{quote}\begin{description}
\item[{Author}] \leavevmode
Francesco Pio Stelluti, Francesco Coppola

\end{description}\end{quote}

\end{fulllineitems}



\subsubsection{Constructors}
\label{\detokenize{source/it/unicam/cs/pa/mastermind/gamecore/BoardController:constructors}}

\paragraph{BoardController}
\label{\detokenize{source/it/unicam/cs/pa/mastermind/gamecore/BoardController:id1}}\index{BoardController(BoardModel) (Java constructor)@\spxentry{BoardController(BoardModel)}\spxextra{Java constructor}}

\begin{fulllineitems}
\phantomsection\label{\detokenize{source/it/unicam/cs/pa/mastermind/gamecore/BoardController:it.unicam.cs.pa.mastermind.gamecore.BoardController.BoardController(BoardModel)}}\pysiglinewithargsret{public \sphinxbfcode{\sphinxupquote{BoardController}}}{{\hyperref[\detokenize{source/it/unicam/cs/pa/mastermind/gamecore/BoardModel:it.unicam.cs.pa.mastermind.gamecore.BoardModel}]{\sphinxcrossref{BoardModel}}}\sphinxstyleemphasis{ newBoard}}{}
Costruttore
\begin{quote}\begin{description}
\item[{Parametri}] \leavevmode\begin{itemize}
\item {} 
\sphinxstyleliteralstrong{\sphinxupquote{newBoard}} \textendash{} la \sphinxcode{\sphinxupquote{BoardModel}} che si desidera gestire

\end{itemize}

\end{description}\end{quote}

\end{fulllineitems}



\subsubsection{Methods}
\label{\detokenize{source/it/unicam/cs/pa/mastermind/gamecore/BoardController:methods}}

\paragraph{getBoardReference}
\label{\detokenize{source/it/unicam/cs/pa/mastermind/gamecore/BoardController:getboardreference}}\index{getBoardReference() (Java method)@\spxentry{getBoardReference()}\spxextra{Java method}}

\begin{fulllineitems}
\phantomsection\label{\detokenize{source/it/unicam/cs/pa/mastermind/gamecore/BoardController:it.unicam.cs.pa.mastermind.gamecore.BoardController.getBoardReference()}}\pysiglinewithargsret{public {\hyperref[\detokenize{source/it/unicam/cs/pa/mastermind/gamecore/BoardModel:it.unicam.cs.pa.mastermind.gamecore.BoardModel}]{\sphinxcrossref{BoardModel}}} \sphinxbfcode{\sphinxupquote{getBoardReference}}}{}{}~\begin{quote}\begin{description}
\item[{Ritorna}] \leavevmode
BoardModel su cui agisce l’istanza di \sphinxcode{\sphinxupquote{BoardCoordinator}} corrente.

\end{description}\end{quote}

\end{fulllineitems}



\paragraph{insertCodeToGuess}
\label{\detokenize{source/it/unicam/cs/pa/mastermind/gamecore/BoardController:insertcodetoguess}}\index{insertCodeToGuess(List) (Java method)@\spxentry{insertCodeToGuess(List)}\spxextra{Java method}}

\begin{fulllineitems}
\phantomsection\label{\detokenize{source/it/unicam/cs/pa/mastermind/gamecore/BoardController:it.unicam.cs.pa.mastermind.gamecore.BoardController.insertCodeToGuess(List)}}\pysiglinewithargsret{public boolean \sphinxbfcode{\sphinxupquote{insertCodeToGuess}}}{\sphinxhref{http://docs.oracle.com/javase/8/docs/api/java/util/List.html}{List}\textless{}{\hyperref[\detokenize{source/it/unicam/cs/pa/mastermind/gamecore/ColorPegs:it.unicam.cs.pa.mastermind.gamecore.ColorPegs}]{\sphinxcrossref{ColorPegs}}}\textgreater{}\sphinxstyleemphasis{ toGuess}}{}
Metodo che consente l’inserimento di una sequenza da indovinare all’interno della \sphinxcode{\sphinxupquote{BoardModel}}.
\begin{quote}\begin{description}
\item[{Parametri}] \leavevmode\begin{itemize}
\item {} 
\sphinxstyleliteralstrong{\sphinxupquote{toGuess}} \textendash{} la \sphinxcode{\sphinxupquote{List}} di \sphinxcode{\sphinxupquote{ColorPegs}} contenente i valori che si vogliono inserire come sequenza da indovinare.

\end{itemize}

\item[{Ritorna}] \leavevmode
boolean a rappresentazione dell’esito dell’inserimento

\end{description}\end{quote}

\end{fulllineitems}



\paragraph{insertNewAttempt}
\label{\detokenize{source/it/unicam/cs/pa/mastermind/gamecore/BoardController:insertnewattempt}}\index{insertNewAttempt(List) (Java method)@\spxentry{insertNewAttempt(List)}\spxextra{Java method}}

\begin{fulllineitems}
\phantomsection\label{\detokenize{source/it/unicam/cs/pa/mastermind/gamecore/BoardController:it.unicam.cs.pa.mastermind.gamecore.BoardController.insertNewAttempt(List)}}\pysiglinewithargsret{public boolean \sphinxbfcode{\sphinxupquote{insertNewAttempt}}}{\sphinxhref{http://docs.oracle.com/javase/8/docs/api/java/util/List.html}{List}\textless{}{\hyperref[\detokenize{source/it/unicam/cs/pa/mastermind/gamecore/ColorPegs:it.unicam.cs.pa.mastermind.gamecore.ColorPegs}]{\sphinxcrossref{ColorPegs}}}\textgreater{}\sphinxstyleemphasis{ attempt}}{}
Metodo che consente l’inserimento di un nuovo tentativo all’interno della \sphinxcode{\sphinxupquote{BoardModel}}.
\begin{quote}\begin{description}
\item[{Parametri}] \leavevmode\begin{itemize}
\item {} 
\sphinxstyleliteralstrong{\sphinxupquote{attempt}} \textendash{} la \sphinxcode{\sphinxupquote{List}} di \sphinxcode{\sphinxupquote{ColorPegs}} contenente i valori che si vogliono inserire all’interno della \sphinxcode{\sphinxupquote{BoardModel}}

\end{itemize}

\item[{Ritorna}] \leavevmode
boolean a rappresentazione dell’esito dell’inserimento

\end{description}\end{quote}

\end{fulllineitems}



\subsection{BoardModel}
\label{\detokenize{source/it/unicam/cs/pa/mastermind/gamecore/BoardModel:boardmodel}}\label{\detokenize{source/it/unicam/cs/pa/mastermind/gamecore/BoardModel::doc}}\index{BoardModel (Java class)@\spxentry{BoardModel}\spxextra{Java class}}

\begin{fulllineitems}
\phantomsection\label{\detokenize{source/it/unicam/cs/pa/mastermind/gamecore/BoardModel:it.unicam.cs.pa.mastermind.gamecore.BoardModel}}\pysigline{public class \sphinxbfcode{\sphinxupquote{BoardModel}}}
\sphinxstylestrong{Responsabilità}: gestire le informazioni relative ad una plancia di gioco.
\begin{quote}\begin{description}
\item[{Author}] \leavevmode
Francesco Pio Stelluti, Francesco Coppola

\end{description}\end{quote}

\end{fulllineitems}



\subsubsection{Constructors}
\label{\detokenize{source/it/unicam/cs/pa/mastermind/gamecore/BoardModel:constructors}}

\paragraph{BoardModel}
\label{\detokenize{source/it/unicam/cs/pa/mastermind/gamecore/BoardModel:id1}}\index{BoardModel(int, int) (Java constructor)@\spxentry{BoardModel(int, int)}\spxextra{Java constructor}}

\begin{fulllineitems}
\phantomsection\label{\detokenize{source/it/unicam/cs/pa/mastermind/gamecore/BoardModel:it.unicam.cs.pa.mastermind.gamecore.BoardModel.BoardModel(int, int)}}\pysiglinewithargsret{public \sphinxbfcode{\sphinxupquote{BoardModel}}}{int\sphinxstyleemphasis{ sequenceLength}, int\sphinxstyleemphasis{ maxAttempts}}{}
Costruttore di una plancia
\begin{quote}\begin{description}
\item[{Parametri}] \leavevmode\begin{itemize}
\item {} 
\sphinxstyleliteralstrong{\sphinxupquote{sequenceLength}} \textendash{} massima delle sequenze presenti in questa plancia

\item {} 
\sphinxstyleliteralstrong{\sphinxupquote{maxAttempts}} \textendash{} numero massimo di tentativi possibili per indovinare la \sphinxcode{\sphinxupquote{sequenceToGuess}}

\end{itemize}

\end{description}\end{quote}

\end{fulllineitems}



\subsubsection{Methods}
\label{\detokenize{source/it/unicam/cs/pa/mastermind/gamecore/BoardModel:methods}}

\paragraph{addAttempt}
\label{\detokenize{source/it/unicam/cs/pa/mastermind/gamecore/BoardModel:addattempt}}\index{addAttempt(List) (Java method)@\spxentry{addAttempt(List)}\spxextra{Java method}}

\begin{fulllineitems}
\phantomsection\label{\detokenize{source/it/unicam/cs/pa/mastermind/gamecore/BoardModel:it.unicam.cs.pa.mastermind.gamecore.BoardModel.addAttempt(List)}}\pysiglinewithargsret{public boolean \sphinxbfcode{\sphinxupquote{addAttempt}}}{\sphinxhref{http://docs.oracle.com/javase/8/docs/api/java/util/List.html}{List}\textless{}{\hyperref[\detokenize{source/it/unicam/cs/pa/mastermind/gamecore/ColorPegs:it.unicam.cs.pa.mastermind.gamecore.ColorPegs}]{\sphinxcrossref{ColorPegs}}}\textgreater{}\sphinxstyleemphasis{ attempt}}{}
Aggiunge alla plancia una nuova sequenza di pioli tentativo e la relativa sequenza di pioli indizio, calcolata all’interno del metodo
\begin{quote}\begin{description}
\item[{Parametri}] \leavevmode\begin{itemize}
\item {} 
\sphinxstyleliteralstrong{\sphinxupquote{attempt}} \textendash{} la sequenza da inserire

\end{itemize}

\item[{Solleva}] \leavevmode\begin{itemize}
\item {} 
\sphinxhref{http://docs.oracle.com/javase/8/docs/api/java/lang/IllegalArgumentException.html}{\sphinxstyleliteralstrong{\sphinxupquote{IllegalArgumentException}}} \textendash{} in caso di inserimento illegale

\end{itemize}

\item[{Ritorna}] \leavevmode
boolean relativo alla riuscita dell’inserimento

\end{description}\end{quote}

\end{fulllineitems}



\paragraph{addObserver}
\label{\detokenize{source/it/unicam/cs/pa/mastermind/gamecore/BoardModel:addobserver}}\index{addObserver(BoardObserver) (Java method)@\spxentry{addObserver(BoardObserver)}\spxextra{Java method}}

\begin{fulllineitems}
\phantomsection\label{\detokenize{source/it/unicam/cs/pa/mastermind/gamecore/BoardModel:it.unicam.cs.pa.mastermind.gamecore.BoardModel.addObserver(BoardObserver)}}\pysiglinewithargsret{public void \sphinxbfcode{\sphinxupquote{addObserver}}}{{\hyperref[\detokenize{source/it/unicam/cs/pa/mastermind/ui/BoardObserver:it.unicam.cs.pa.mastermind.ui.BoardObserver}]{\sphinxcrossref{BoardObserver}}}\sphinxstyleemphasis{ observer}}{}
Metodo il quale registra un nuovo \sphinxcode{\sphinxupquote{BoardObserver}} e notifica tutti i \sphinxcode{\sphinxupquote{BoardObserver}} attualmente associati all’istanza di \sphinxcode{\sphinxupquote{BoardModel}}.
\begin{quote}\begin{description}
\item[{Parametri}] \leavevmode\begin{itemize}
\item {} 
\sphinxstyleliteralstrong{\sphinxupquote{observer}} \textendash{} nuova istanza di \sphinxcode{\sphinxupquote{BoardObserver}} da aggiungere

\end{itemize}

\end{description}\end{quote}

\end{fulllineitems}



\paragraph{attemptsInserted}
\label{\detokenize{source/it/unicam/cs/pa/mastermind/gamecore/BoardModel:attemptsinserted}}\index{attemptsInserted() (Java method)@\spxentry{attemptsInserted()}\spxextra{Java method}}

\begin{fulllineitems}
\phantomsection\label{\detokenize{source/it/unicam/cs/pa/mastermind/gamecore/BoardModel:it.unicam.cs.pa.mastermind.gamecore.BoardModel.attemptsInserted()}}\pysiglinewithargsret{public int \sphinxbfcode{\sphinxupquote{attemptsInserted}}}{}{}~\begin{quote}\begin{description}
\item[{Ritorna}] \leavevmode
int numero di tentativi inseriti fino ad ora

\end{description}\end{quote}

\end{fulllineitems}



\paragraph{getAttemptAndClueList}
\label{\detokenize{source/it/unicam/cs/pa/mastermind/gamecore/BoardModel:getattemptandcluelist}}\index{getAttemptAndClueList() (Java method)@\spxentry{getAttemptAndClueList()}\spxextra{Java method}}

\begin{fulllineitems}
\phantomsection\label{\detokenize{source/it/unicam/cs/pa/mastermind/gamecore/BoardModel:it.unicam.cs.pa.mastermind.gamecore.BoardModel.getAttemptAndClueList()}}\pysiglinewithargsret{public \sphinxhref{http://docs.oracle.com/javase/8/docs/api/java/util/List.html}{List}\textless{}\sphinxhref{http://docs.oracle.com/javase/8/docs/api/java/util/Map.html}{Map}.Entry\textless{}\sphinxhref{http://docs.oracle.com/javase/8/docs/api/java/util/List.html}{List}\textless{}{\hyperref[\detokenize{source/it/unicam/cs/pa/mastermind/gamecore/ColorPegs:it.unicam.cs.pa.mastermind.gamecore.ColorPegs}]{\sphinxcrossref{ColorPegs}}}\textgreater{}, \sphinxhref{http://docs.oracle.com/javase/8/docs/api/java/util/List.html}{List}\textless{}{\hyperref[\detokenize{source/it/unicam/cs/pa/mastermind/gamecore/ColorPegs:it.unicam.cs.pa.mastermind.gamecore.ColorPegs}]{\sphinxcrossref{ColorPegs}}}\textgreater{}\textgreater{}\textgreater{} \sphinxbfcode{\sphinxupquote{getAttemptAndClueList}}}{}{}~\begin{quote}\begin{description}
\item[{Ritorna}] \leavevmode
List contenenti Map.Entry con le sequenze di \sphinxcode{\sphinxupquote{ColorPegs}} inserite come tentativo e le relative sequenze indizio

\end{description}\end{quote}

\end{fulllineitems}



\paragraph{getClueFromAttempt}
\label{\detokenize{source/it/unicam/cs/pa/mastermind/gamecore/BoardModel:getcluefromattempt}}\index{getClueFromAttempt(List) (Java method)@\spxentry{getClueFromAttempt(List)}\spxextra{Java method}}

\begin{fulllineitems}
\phantomsection\label{\detokenize{source/it/unicam/cs/pa/mastermind/gamecore/BoardModel:it.unicam.cs.pa.mastermind.gamecore.BoardModel.getClueFromAttempt(List)}}\pysiglinewithargsret{public \sphinxhref{http://docs.oracle.com/javase/8/docs/api/java/util/List.html}{List}\textless{}{\hyperref[\detokenize{source/it/unicam/cs/pa/mastermind/gamecore/ColorPegs:it.unicam.cs.pa.mastermind.gamecore.ColorPegs}]{\sphinxcrossref{ColorPegs}}}\textgreater{} \sphinxbfcode{\sphinxupquote{getClueFromAttempt}}}{\sphinxhref{http://docs.oracle.com/javase/8/docs/api/java/util/List.html}{List}\textless{}{\hyperref[\detokenize{source/it/unicam/cs/pa/mastermind/gamecore/ColorPegs:it.unicam.cs.pa.mastermind.gamecore.ColorPegs}]{\sphinxcrossref{ColorPegs}}}\textgreater{}\sphinxstyleemphasis{ attempt}}{}
Calcolo della sequenza di \sphinxcode{\sphinxupquote{ColorPegs}} indizio a fronte di una sequenza di \sphinxcode{\sphinxupquote{ColorPegs}} assicurata valida come tentativo.
\begin{quote}\begin{description}
\item[{Parametri}] \leavevmode\begin{itemize}
\item {} 
\sphinxstyleliteralstrong{\sphinxupquote{attempt}} \textendash{} la lista che si inserisce come tentativo di risoluzione.

\item {} 
\sphinxstyleliteralstrong{\sphinxupquote{toGuess}} \textendash{} la lista che contiene la sequenza da indovinare.

\end{itemize}

\item[{Ritorna}] \leavevmode
List di indizi generata a partire dalla lista di tentativi.

\end{description}\end{quote}

\end{fulllineitems}



\paragraph{getSequenceLength}
\label{\detokenize{source/it/unicam/cs/pa/mastermind/gamecore/BoardModel:getsequencelength}}\index{getSequenceLength() (Java method)@\spxentry{getSequenceLength()}\spxextra{Java method}}

\begin{fulllineitems}
\phantomsection\label{\detokenize{source/it/unicam/cs/pa/mastermind/gamecore/BoardModel:it.unicam.cs.pa.mastermind.gamecore.BoardModel.getSequenceLength()}}\pysiglinewithargsret{public int \sphinxbfcode{\sphinxupquote{getSequenceLength}}}{}{}~\begin{quote}\begin{description}
\item[{Ritorna}] \leavevmode
int lunghezza massima delle sequenze presenti in questa plancia

\end{description}\end{quote}

\end{fulllineitems}



\paragraph{getSequenceToGuess}
\label{\detokenize{source/it/unicam/cs/pa/mastermind/gamecore/BoardModel:getsequencetoguess}}\index{getSequenceToGuess() (Java method)@\spxentry{getSequenceToGuess()}\spxextra{Java method}}

\begin{fulllineitems}
\phantomsection\label{\detokenize{source/it/unicam/cs/pa/mastermind/gamecore/BoardModel:it.unicam.cs.pa.mastermind.gamecore.BoardModel.getSequenceToGuess()}}\pysiglinewithargsret{public \sphinxhref{http://docs.oracle.com/javase/8/docs/api/java/util/List.html}{List}\textless{}{\hyperref[\detokenize{source/it/unicam/cs/pa/mastermind/gamecore/ColorPegs:it.unicam.cs.pa.mastermind.gamecore.ColorPegs}]{\sphinxcrossref{ColorPegs}}}\textgreater{} \sphinxbfcode{\sphinxupquote{getSequenceToGuess}}}{}{}~\begin{quote}\begin{description}
\item[{Ritorna}] \leavevmode
List di \sphinxcode{\sphinxupquote{ColorPegs}} da indovinare.

\end{description}\end{quote}

\end{fulllineitems}



\paragraph{hasBreakerGuessed}
\label{\detokenize{source/it/unicam/cs/pa/mastermind/gamecore/BoardModel:hasbreakerguessed}}\index{hasBreakerGuessed() (Java method)@\spxentry{hasBreakerGuessed()}\spxextra{Java method}}

\begin{fulllineitems}
\phantomsection\label{\detokenize{source/it/unicam/cs/pa/mastermind/gamecore/BoardModel:it.unicam.cs.pa.mastermind.gamecore.BoardModel.hasBreakerGuessed()}}\pysiglinewithargsret{public boolean \sphinxbfcode{\sphinxupquote{hasBreakerGuessed}}}{}{}~\begin{quote}\begin{description}
\item[{Ritorna}] \leavevmode
boolean che indica se il giocatore Breaker ha indovinato o meno la sequenza del Maker in base alle informazioni contenute nella plancia

\end{description}\end{quote}

\end{fulllineitems}



\paragraph{isBoardEmpty}
\label{\detokenize{source/it/unicam/cs/pa/mastermind/gamecore/BoardModel:isboardempty}}\index{isBoardEmpty() (Java method)@\spxentry{isBoardEmpty()}\spxextra{Java method}}

\begin{fulllineitems}
\phantomsection\label{\detokenize{source/it/unicam/cs/pa/mastermind/gamecore/BoardModel:it.unicam.cs.pa.mastermind.gamecore.BoardModel.isBoardEmpty()}}\pysiglinewithargsret{public boolean \sphinxbfcode{\sphinxupquote{isBoardEmpty}}}{}{}~\begin{quote}\begin{description}
\item[{Ritorna}] \leavevmode
boolean che indica se sono stati inseriti o meno tentativi nella plancia

\end{description}\end{quote}

\end{fulllineitems}



\paragraph{lastAttemptAndClue}
\label{\detokenize{source/it/unicam/cs/pa/mastermind/gamecore/BoardModel:lastattemptandclue}}\index{lastAttemptAndClue() (Java method)@\spxentry{lastAttemptAndClue()}\spxextra{Java method}}

\begin{fulllineitems}
\phantomsection\label{\detokenize{source/it/unicam/cs/pa/mastermind/gamecore/BoardModel:it.unicam.cs.pa.mastermind.gamecore.BoardModel.lastAttemptAndClue()}}\pysiglinewithargsret{public \sphinxhref{http://docs.oracle.com/javase/8/docs/api/java/util/Map.html}{Map}.Entry\textless{}\sphinxhref{http://docs.oracle.com/javase/8/docs/api/java/util/List.html}{List}\textless{}{\hyperref[\detokenize{source/it/unicam/cs/pa/mastermind/gamecore/ColorPegs:it.unicam.cs.pa.mastermind.gamecore.ColorPegs}]{\sphinxcrossref{ColorPegs}}}\textgreater{}, \sphinxhref{http://docs.oracle.com/javase/8/docs/api/java/util/List.html}{List}\textless{}{\hyperref[\detokenize{source/it/unicam/cs/pa/mastermind/gamecore/ColorPegs:it.unicam.cs.pa.mastermind.gamecore.ColorPegs}]{\sphinxcrossref{ColorPegs}}}\textgreater{}\textgreater{} \sphinxbfcode{\sphinxupquote{lastAttemptAndClue}}}{}{}~\begin{quote}\begin{description}
\item[{Ritorna}] \leavevmode
Map.Entry contenente l’ultima sequenza di \sphinxcode{\sphinxupquote{ColorPegs}} inserita come tentativo e la relativa sequenza indizio.

\end{description}\end{quote}

\end{fulllineitems}



\paragraph{leftAttempts}
\label{\detokenize{source/it/unicam/cs/pa/mastermind/gamecore/BoardModel:leftattempts}}\index{leftAttempts() (Java method)@\spxentry{leftAttempts()}\spxextra{Java method}}

\begin{fulllineitems}
\phantomsection\label{\detokenize{source/it/unicam/cs/pa/mastermind/gamecore/BoardModel:it.unicam.cs.pa.mastermind.gamecore.BoardModel.leftAttempts()}}\pysiglinewithargsret{public int \sphinxbfcode{\sphinxupquote{leftAttempts}}}{}{}~\begin{quote}\begin{description}
\item[{Ritorna}] \leavevmode
int numero di tentativi rimasti

\end{description}\end{quote}

\end{fulllineitems}



\paragraph{setSequenceToGuess}
\label{\detokenize{source/it/unicam/cs/pa/mastermind/gamecore/BoardModel:setsequencetoguess}}\index{setSequenceToGuess(List) (Java method)@\spxentry{setSequenceToGuess(List)}\spxextra{Java method}}

\begin{fulllineitems}
\phantomsection\label{\detokenize{source/it/unicam/cs/pa/mastermind/gamecore/BoardModel:it.unicam.cs.pa.mastermind.gamecore.BoardModel.setSequenceToGuess(List)}}\pysiglinewithargsret{public boolean \sphinxbfcode{\sphinxupquote{setSequenceToGuess}}}{\sphinxhref{http://docs.oracle.com/javase/8/docs/api/java/util/List.html}{List}\textless{}{\hyperref[\detokenize{source/it/unicam/cs/pa/mastermind/gamecore/ColorPegs:it.unicam.cs.pa.mastermind.gamecore.ColorPegs}]{\sphinxcrossref{ColorPegs}}}\textgreater{}\sphinxstyleemphasis{ toGuess}}{}
Imposta la sequenza di pioli da indovinare.
\begin{quote}\begin{description}
\item[{Parametri}] \leavevmode\begin{itemize}
\item {} 
\sphinxstyleliteralstrong{\sphinxupquote{toGuess}} \textendash{} lista di \sphinxcode{\sphinxupquote{ColorPegs}} della sequenza da indovinare

\end{itemize}

\item[{Solleva}] \leavevmode\begin{itemize}
\item {} 
\sphinxhref{http://docs.oracle.com/javase/8/docs/api/java/lang/IllegalArgumentException.html}{\sphinxstyleliteralstrong{\sphinxupquote{IllegalArgumentException}}} \textendash{} se la lunghezza della sequenza inserita non è valida

\end{itemize}

\item[{Ritorna}] \leavevmode
un booleano a seconda della riuscita o meno dell’inserimento nella plancia di gioco

\end{description}\end{quote}

\end{fulllineitems}



\subsection{ColorPegs}
\label{\detokenize{source/it/unicam/cs/pa/mastermind/gamecore/ColorPegs:colorpegs}}\label{\detokenize{source/it/unicam/cs/pa/mastermind/gamecore/ColorPegs::doc}}\index{ColorPegs (Java enum)@\spxentry{ColorPegs}\spxextra{Java enum}}

\begin{fulllineitems}
\phantomsection\label{\detokenize{source/it/unicam/cs/pa/mastermind/gamecore/ColorPegs:it.unicam.cs.pa.mastermind.gamecore.ColorPegs}}\pysigline{public enum \sphinxbfcode{\sphinxupquote{ColorPegs}}}
\sphinxstylestrong{Responsabilità}: rappresentare gli elementi alla base delle sequenze trattate durante le partite di gioco.
\begin{quote}\begin{description}
\item[{Author}] \leavevmode
Francesco Pio Stelluti, Francesco Coppola

\end{description}\end{quote}

\end{fulllineitems}



\subsubsection{Enum Constants}
\label{\detokenize{source/it/unicam/cs/pa/mastermind/gamecore/ColorPegs:enum-constants}}

\paragraph{BLACK}
\label{\detokenize{source/it/unicam/cs/pa/mastermind/gamecore/ColorPegs:black}}\index{BLACK (Java field)@\spxentry{BLACK}\spxextra{Java field}}

\begin{fulllineitems}
\phantomsection\label{\detokenize{source/it/unicam/cs/pa/mastermind/gamecore/ColorPegs:it.unicam.cs.pa.mastermind.gamecore.ColorPegs.BLACK}}\pysigline{public static final {\hyperref[\detokenize{source/it/unicam/cs/pa/mastermind/gamecore/ColorPegs:it.unicam.cs.pa.mastermind.gamecore.ColorPegs}]{\sphinxcrossref{ColorPegs}}} \sphinxbfcode{\sphinxupquote{BLACK}}}
\end{fulllineitems}



\paragraph{BLUE}
\label{\detokenize{source/it/unicam/cs/pa/mastermind/gamecore/ColorPegs:blue}}\index{BLUE (Java field)@\spxentry{BLUE}\spxextra{Java field}}

\begin{fulllineitems}
\phantomsection\label{\detokenize{source/it/unicam/cs/pa/mastermind/gamecore/ColorPegs:it.unicam.cs.pa.mastermind.gamecore.ColorPegs.BLUE}}\pysigline{public static final {\hyperref[\detokenize{source/it/unicam/cs/pa/mastermind/gamecore/ColorPegs:it.unicam.cs.pa.mastermind.gamecore.ColorPegs}]{\sphinxcrossref{ColorPegs}}} \sphinxbfcode{\sphinxupquote{BLUE}}}
\end{fulllineitems}



\paragraph{CYAN}
\label{\detokenize{source/it/unicam/cs/pa/mastermind/gamecore/ColorPegs:cyan}}\index{CYAN (Java field)@\spxentry{CYAN}\spxextra{Java field}}

\begin{fulllineitems}
\phantomsection\label{\detokenize{source/it/unicam/cs/pa/mastermind/gamecore/ColorPegs:it.unicam.cs.pa.mastermind.gamecore.ColorPegs.CYAN}}\pysigline{public static final {\hyperref[\detokenize{source/it/unicam/cs/pa/mastermind/gamecore/ColorPegs:it.unicam.cs.pa.mastermind.gamecore.ColorPegs}]{\sphinxcrossref{ColorPegs}}} \sphinxbfcode{\sphinxupquote{CYAN}}}
\end{fulllineitems}



\paragraph{GREEN}
\label{\detokenize{source/it/unicam/cs/pa/mastermind/gamecore/ColorPegs:green}}\index{GREEN (Java field)@\spxentry{GREEN}\spxextra{Java field}}

\begin{fulllineitems}
\phantomsection\label{\detokenize{source/it/unicam/cs/pa/mastermind/gamecore/ColorPegs:it.unicam.cs.pa.mastermind.gamecore.ColorPegs.GREEN}}\pysigline{public static final {\hyperref[\detokenize{source/it/unicam/cs/pa/mastermind/gamecore/ColorPegs:it.unicam.cs.pa.mastermind.gamecore.ColorPegs}]{\sphinxcrossref{ColorPegs}}} \sphinxbfcode{\sphinxupquote{GREEN}}}
\end{fulllineitems}



\paragraph{PURPLE}
\label{\detokenize{source/it/unicam/cs/pa/mastermind/gamecore/ColorPegs:purple}}\index{PURPLE (Java field)@\spxentry{PURPLE}\spxextra{Java field}}

\begin{fulllineitems}
\phantomsection\label{\detokenize{source/it/unicam/cs/pa/mastermind/gamecore/ColorPegs:it.unicam.cs.pa.mastermind.gamecore.ColorPegs.PURPLE}}\pysigline{public static final {\hyperref[\detokenize{source/it/unicam/cs/pa/mastermind/gamecore/ColorPegs:it.unicam.cs.pa.mastermind.gamecore.ColorPegs}]{\sphinxcrossref{ColorPegs}}} \sphinxbfcode{\sphinxupquote{PURPLE}}}
\end{fulllineitems}



\paragraph{RED}
\label{\detokenize{source/it/unicam/cs/pa/mastermind/gamecore/ColorPegs:red}}\index{RED (Java field)@\spxentry{RED}\spxextra{Java field}}

\begin{fulllineitems}
\phantomsection\label{\detokenize{source/it/unicam/cs/pa/mastermind/gamecore/ColorPegs:it.unicam.cs.pa.mastermind.gamecore.ColorPegs.RED}}\pysigline{public static final {\hyperref[\detokenize{source/it/unicam/cs/pa/mastermind/gamecore/ColorPegs:it.unicam.cs.pa.mastermind.gamecore.ColorPegs}]{\sphinxcrossref{ColorPegs}}} \sphinxbfcode{\sphinxupquote{RED}}}
\end{fulllineitems}



\paragraph{WHITE}
\label{\detokenize{source/it/unicam/cs/pa/mastermind/gamecore/ColorPegs:white}}\index{WHITE (Java field)@\spxentry{WHITE}\spxextra{Java field}}

\begin{fulllineitems}
\phantomsection\label{\detokenize{source/it/unicam/cs/pa/mastermind/gamecore/ColorPegs:it.unicam.cs.pa.mastermind.gamecore.ColorPegs.WHITE}}\pysigline{public static final {\hyperref[\detokenize{source/it/unicam/cs/pa/mastermind/gamecore/ColorPegs:it.unicam.cs.pa.mastermind.gamecore.ColorPegs}]{\sphinxcrossref{ColorPegs}}} \sphinxbfcode{\sphinxupquote{WHITE}}}
\end{fulllineitems}



\paragraph{YELLOW}
\label{\detokenize{source/it/unicam/cs/pa/mastermind/gamecore/ColorPegs:yellow}}\index{YELLOW (Java field)@\spxentry{YELLOW}\spxextra{Java field}}

\begin{fulllineitems}
\phantomsection\label{\detokenize{source/it/unicam/cs/pa/mastermind/gamecore/ColorPegs:it.unicam.cs.pa.mastermind.gamecore.ColorPegs.YELLOW}}\pysigline{public static final {\hyperref[\detokenize{source/it/unicam/cs/pa/mastermind/gamecore/ColorPegs:it.unicam.cs.pa.mastermind.gamecore.ColorPegs}]{\sphinxcrossref{ColorPegs}}} \sphinxbfcode{\sphinxupquote{YELLOW}}}
\end{fulllineitems}



\subsection{CurrentGameStats}
\label{\detokenize{source/it/unicam/cs/pa/mastermind/gamecore/CurrentGameStats:currentgamestats}}\label{\detokenize{source/it/unicam/cs/pa/mastermind/gamecore/CurrentGameStats::doc}}\index{CurrentGameStats (Java class)@\spxentry{CurrentGameStats}\spxextra{Java class}}

\begin{fulllineitems}
\phantomsection\label{\detokenize{source/it/unicam/cs/pa/mastermind/gamecore/CurrentGameStats:it.unicam.cs.pa.mastermind.gamecore.CurrentGameStats}}\pysigline{public class \sphinxbfcode{\sphinxupquote{CurrentGameStats}} extends {\hyperref[\detokenize{source/it/unicam/cs/pa/mastermind/ui/BoardObserver:it.unicam.cs.pa.mastermind.ui.BoardObserver}]{\sphinxcrossref{BoardObserver}}}}
\sphinxstylestrong{Responsabilità}: tenere traccia delle informazioni necessarie per poter decretare se una partita è terminata o meno.
\begin{quote}\begin{description}
\item[{Author}] \leavevmode
Francesco Pio Stelluti, Francesco Coppola

\end{description}\end{quote}

\end{fulllineitems}



\subsubsection{Constructors}
\label{\detokenize{source/it/unicam/cs/pa/mastermind/gamecore/CurrentGameStats:constructors}}

\paragraph{CurrentGameStats}
\label{\detokenize{source/it/unicam/cs/pa/mastermind/gamecore/CurrentGameStats:id1}}\index{CurrentGameStats(BoardModel) (Java constructor)@\spxentry{CurrentGameStats(BoardModel)}\spxextra{Java constructor}}

\begin{fulllineitems}
\phantomsection\label{\detokenize{source/it/unicam/cs/pa/mastermind/gamecore/CurrentGameStats:it.unicam.cs.pa.mastermind.gamecore.CurrentGameStats.CurrentGameStats(BoardModel)}}\pysiglinewithargsret{public \sphinxbfcode{\sphinxupquote{CurrentGameStats}}}{{\hyperref[\detokenize{source/it/unicam/cs/pa/mastermind/gamecore/BoardModel:it.unicam.cs.pa.mastermind.gamecore.BoardModel}]{\sphinxcrossref{BoardModel}}}\sphinxstyleemphasis{ board}}{}
Costruttore.

\end{fulllineitems}



\subsubsection{Methods}
\label{\detokenize{source/it/unicam/cs/pa/mastermind/gamecore/CurrentGameStats:methods}}

\paragraph{getAttempts}
\label{\detokenize{source/it/unicam/cs/pa/mastermind/gamecore/CurrentGameStats:getattempts}}\index{getAttempts() (Java method)@\spxentry{getAttempts()}\spxextra{Java method}}

\begin{fulllineitems}
\phantomsection\label{\detokenize{source/it/unicam/cs/pa/mastermind/gamecore/CurrentGameStats:it.unicam.cs.pa.mastermind.gamecore.CurrentGameStats.getAttempts()}}\pysiglinewithargsret{public int \sphinxbfcode{\sphinxupquote{getAttempts}}}{}{}
Metodo attraverso il quale vengono restituiti i tentativi rimanenti al player per vincere il game corrente.
\begin{quote}\begin{description}
\item[{Ritorna}] \leavevmode
int numero di tentativi che sono stati necessari al Breaker per vincere.

\end{description}\end{quote}

\end{fulllineitems}



\paragraph{getHasBreakerWon}
\label{\detokenize{source/it/unicam/cs/pa/mastermind/gamecore/CurrentGameStats:gethasbreakerwon}}\index{getHasBreakerWon() (Java method)@\spxentry{getHasBreakerWon()}\spxextra{Java method}}

\begin{fulllineitems}
\phantomsection\label{\detokenize{source/it/unicam/cs/pa/mastermind/gamecore/CurrentGameStats:it.unicam.cs.pa.mastermind.gamecore.CurrentGameStats.getHasBreakerWon()}}\pysiglinewithargsret{public boolean \sphinxbfcode{\sphinxupquote{getHasBreakerWon}}}{}{}
Metodo che stabilisce la vittoria del giocatore Breaker o meno.
\begin{quote}\begin{description}
\item[{Ritorna}] \leavevmode
boolean che indica se il Breaker ha vinto o meno.

\end{description}\end{quote}

\end{fulllineitems}



\paragraph{getHasMakerWon}
\label{\detokenize{source/it/unicam/cs/pa/mastermind/gamecore/CurrentGameStats:gethasmakerwon}}\index{getHasMakerWon() (Java method)@\spxentry{getHasMakerWon()}\spxextra{Java method}}

\begin{fulllineitems}
\phantomsection\label{\detokenize{source/it/unicam/cs/pa/mastermind/gamecore/CurrentGameStats:it.unicam.cs.pa.mastermind.gamecore.CurrentGameStats.getHasMakerWon()}}\pysiglinewithargsret{public boolean \sphinxbfcode{\sphinxupquote{getHasMakerWon}}}{}{}
Metodo che stabilisce la vittoria del giocatore Maker o meno.
\begin{quote}\begin{description}
\item[{Ritorna}] \leavevmode
boolean che indica se il Maker ha vinto o meno.

\end{description}\end{quote}

\end{fulllineitems}



\paragraph{getMessage}
\label{\detokenize{source/it/unicam/cs/pa/mastermind/gamecore/CurrentGameStats:getmessage}}\index{getMessage() (Java method)@\spxentry{getMessage()}\spxextra{Java method}}

\begin{fulllineitems}
\phantomsection\label{\detokenize{source/it/unicam/cs/pa/mastermind/gamecore/CurrentGameStats:it.unicam.cs.pa.mastermind.gamecore.CurrentGameStats.getMessage()}}\pysiglinewithargsret{public \sphinxhref{http://docs.oracle.com/javase/8/docs/api/java/lang/String.html}{String} \sphinxbfcode{\sphinxupquote{getMessage}}}{}{}
Metodo che comunica l’esito finale della partita corrente.
\begin{quote}\begin{description}
\item[{Ritorna}] \leavevmode
String che comunica il vincitore attuale della partita

\end{description}\end{quote}

\end{fulllineitems}



\paragraph{toggleBreakerWin}
\label{\detokenize{source/it/unicam/cs/pa/mastermind/gamecore/CurrentGameStats:togglebreakerwin}}\index{toggleBreakerWin(int) (Java method)@\spxentry{toggleBreakerWin(int)}\spxextra{Java method}}

\begin{fulllineitems}
\phantomsection\label{\detokenize{source/it/unicam/cs/pa/mastermind/gamecore/CurrentGameStats:it.unicam.cs.pa.mastermind.gamecore.CurrentGameStats.toggleBreakerWin(int)}}\pysiglinewithargsret{public void \sphinxbfcode{\sphinxupquote{toggleBreakerWin}}}{int\sphinxstyleemphasis{ attempts}}{}
Toggle sulle variabili private per indicare la vittoria del Breaker.
\begin{quote}\begin{description}
\item[{Parametri}] \leavevmode\begin{itemize}
\item {} 
\sphinxstyleliteralstrong{\sphinxupquote{attempts}} \textendash{} il numero di tentativi impiegati dal Breaker per vincere

\end{itemize}

\end{description}\end{quote}

\end{fulllineitems}



\paragraph{toggleMakerWin}
\label{\detokenize{source/it/unicam/cs/pa/mastermind/gamecore/CurrentGameStats:togglemakerwin}}\index{toggleMakerWin() (Java method)@\spxentry{toggleMakerWin()}\spxextra{Java method}}

\begin{fulllineitems}
\phantomsection\label{\detokenize{source/it/unicam/cs/pa/mastermind/gamecore/CurrentGameStats:it.unicam.cs.pa.mastermind.gamecore.CurrentGameStats.toggleMakerWin()}}\pysiglinewithargsret{public void \sphinxbfcode{\sphinxupquote{toggleMakerWin}}}{}{}
Toggle sulle variabili private per indicare la vittoria del Maker.

\end{fulllineitems}



\paragraph{update}
\label{\detokenize{source/it/unicam/cs/pa/mastermind/gamecore/CurrentGameStats:update}}\index{update() (Java method)@\spxentry{update()}\spxextra{Java method}}

\begin{fulllineitems}
\phantomsection\label{\detokenize{source/it/unicam/cs/pa/mastermind/gamecore/CurrentGameStats:it.unicam.cs.pa.mastermind.gamecore.CurrentGameStats.update()}}\pysiglinewithargsret{public void \sphinxbfcode{\sphinxupquote{update}}}{}{}
\end{fulllineitems}



\subsection{NewGameStats}
\label{\detokenize{source/it/unicam/cs/pa/mastermind/gamecore/NewGameStats:newgamestats}}\label{\detokenize{source/it/unicam/cs/pa/mastermind/gamecore/NewGameStats::doc}}\index{NewGameStats (Java class)@\spxentry{NewGameStats}\spxextra{Java class}}

\begin{fulllineitems}
\phantomsection\label{\detokenize{source/it/unicam/cs/pa/mastermind/gamecore/NewGameStats:it.unicam.cs.pa.mastermind.gamecore.NewGameStats}}\pysigline{public class \sphinxbfcode{\sphinxupquote{NewGameStats}}}
\sphinxstylestrong{Responsabilità}: tenere traccia delle informazioni necessarie per poter iniziare una nuova partita dopo che ne è stata conclusa una.
\begin{quote}\begin{description}
\item[{Author}] \leavevmode
Francesco Pio Stelluti, Francesco Coppola

\end{description}\end{quote}

\end{fulllineitems}



\subsubsection{Constructors}
\label{\detokenize{source/it/unicam/cs/pa/mastermind/gamecore/NewGameStats:constructors}}

\paragraph{NewGameStats}
\label{\detokenize{source/it/unicam/cs/pa/mastermind/gamecore/NewGameStats:id1}}\index{NewGameStats(boolean, boolean) (Java constructor)@\spxentry{NewGameStats(boolean, boolean)}\spxextra{Java constructor}}

\begin{fulllineitems}
\phantomsection\label{\detokenize{source/it/unicam/cs/pa/mastermind/gamecore/NewGameStats:it.unicam.cs.pa.mastermind.gamecore.NewGameStats.NewGameStats(boolean, boolean)}}\pysiglinewithargsret{public \sphinxbfcode{\sphinxupquote{NewGameStats}}}{boolean\sphinxstyleemphasis{ toContinue}, boolean\sphinxstyleemphasis{ keepSettings}}{}
Costruttore
\begin{quote}\begin{description}
\item[{Parametri}] \leavevmode\begin{itemize}
\item {} 
\sphinxstyleliteralstrong{\sphinxupquote{toContinue}} \textendash{} volontà dell’utente umano di continuare a giocare o meno.

\item {} 
\sphinxstyleliteralstrong{\sphinxupquote{keepSettings}} \textendash{} volontà dell’utente umano di continuare a giocare con le medesime impostazioni o meno.

\end{itemize}

\end{description}\end{quote}

\end{fulllineitems}



\subsubsection{Methods}
\label{\detokenize{source/it/unicam/cs/pa/mastermind/gamecore/NewGameStats:methods}}

\paragraph{getContinue}
\label{\detokenize{source/it/unicam/cs/pa/mastermind/gamecore/NewGameStats:getcontinue}}\index{getContinue() (Java method)@\spxentry{getContinue()}\spxextra{Java method}}

\begin{fulllineitems}
\phantomsection\label{\detokenize{source/it/unicam/cs/pa/mastermind/gamecore/NewGameStats:it.unicam.cs.pa.mastermind.gamecore.NewGameStats.getContinue()}}\pysiglinewithargsret{public boolean \sphinxbfcode{\sphinxupquote{getContinue}}}{}{}~\begin{quote}\begin{description}
\item[{Ritorna}] \leavevmode
boolean volontà dell’utente umano di continuare a giocare o meno.

\end{description}\end{quote}

\end{fulllineitems}



\paragraph{getKeepSettings}
\label{\detokenize{source/it/unicam/cs/pa/mastermind/gamecore/NewGameStats:getkeepsettings}}\index{getKeepSettings() (Java method)@\spxentry{getKeepSettings()}\spxextra{Java method}}

\begin{fulllineitems}
\phantomsection\label{\detokenize{source/it/unicam/cs/pa/mastermind/gamecore/NewGameStats:it.unicam.cs.pa.mastermind.gamecore.NewGameStats.getKeepSettings()}}\pysiglinewithargsret{public boolean \sphinxbfcode{\sphinxupquote{getKeepSettings}}}{}{}~\begin{quote}\begin{description}
\item[{Ritorna}] \leavevmode
boolean volontà dell’utente umano di continuare a giocare con le medesime impostazioni o meno.

\end{description}\end{quote}

\end{fulllineitems}



\subsection{SingleMatch}
\label{\detokenize{source/it/unicam/cs/pa/mastermind/gamecore/SingleMatch:singlematch}}\label{\detokenize{source/it/unicam/cs/pa/mastermind/gamecore/SingleMatch::doc}}\index{SingleMatch (Java class)@\spxentry{SingleMatch}\spxextra{Java class}}

\begin{fulllineitems}
\phantomsection\label{\detokenize{source/it/unicam/cs/pa/mastermind/gamecore/SingleMatch:it.unicam.cs.pa.mastermind.gamecore.SingleMatch}}\pysigline{public class \sphinxbfcode{\sphinxupquote{SingleMatch}}}
\sphinxstylestrong{Responsabilità}: gestione dello svolgimento di una singola partita di gioco.
\begin{quote}\begin{description}
\item[{Author}] \leavevmode
Francesco Pio Stelluti, Francesco Coppola

\end{description}\end{quote}

\end{fulllineitems}



\subsubsection{Fields}
\label{\detokenize{source/it/unicam/cs/pa/mastermind/gamecore/SingleMatch:fields}}

\paragraph{gameStats}
\label{\detokenize{source/it/unicam/cs/pa/mastermind/gamecore/SingleMatch:gamestats}}\index{gameStats (Java field)@\spxentry{gameStats}\spxextra{Java field}}

\begin{fulllineitems}
\phantomsection\label{\detokenize{source/it/unicam/cs/pa/mastermind/gamecore/SingleMatch:it.unicam.cs.pa.mastermind.gamecore.SingleMatch.gameStats}}\pysigline{ {\hyperref[\detokenize{source/it/unicam/cs/pa/mastermind/gamecore/CurrentGameStats:it.unicam.cs.pa.mastermind.gamecore.CurrentGameStats}]{\sphinxcrossref{CurrentGameStats}}} \sphinxbfcode{\sphinxupquote{gameStats}}}
Oggetto contenente informazioni relative al vincitore della partita in corso.

\end{fulllineitems}



\subsubsection{Constructors}
\label{\detokenize{source/it/unicam/cs/pa/mastermind/gamecore/SingleMatch:constructors}}

\paragraph{SingleMatch}
\label{\detokenize{source/it/unicam/cs/pa/mastermind/gamecore/SingleMatch:id1}}\index{SingleMatch(int, int, InteractionView, CodeBreaker, CodeMaker) (Java constructor)@\spxentry{SingleMatch(int, int, InteractionView, CodeBreaker, CodeMaker)}\spxextra{Java constructor}}

\begin{fulllineitems}
\phantomsection\label{\detokenize{source/it/unicam/cs/pa/mastermind/gamecore/SingleMatch:it.unicam.cs.pa.mastermind.gamecore.SingleMatch.SingleMatch(int, int, InteractionView, CodeBreaker, CodeMaker)}}\pysiglinewithargsret{public \sphinxbfcode{\sphinxupquote{SingleMatch}}}{int\sphinxstyleemphasis{ sequenceLength}, int\sphinxstyleemphasis{ attempts}, {\hyperref[\detokenize{source/it/unicam/cs/pa/mastermind/ui/InteractionView:it.unicam.cs.pa.mastermind.ui.InteractionView}]{\sphinxcrossref{InteractionView}}}\sphinxstyleemphasis{ view}, {\hyperref[\detokenize{source/it/unicam/cs/pa/mastermind/players/CodeBreaker:it.unicam.cs.pa.mastermind.players.CodeBreaker}]{\sphinxcrossref{CodeBreaker}}}\sphinxstyleemphasis{ currentBreaker}, {\hyperref[\detokenize{source/it/unicam/cs/pa/mastermind/players/CodeMaker:it.unicam.cs.pa.mastermind.players.CodeMaker}]{\sphinxcrossref{CodeMaker}}}\sphinxstyleemphasis{ currentMaker}}{}
Costruttore di una singola partita
\begin{quote}\begin{description}
\item[{Parametri}] \leavevmode\begin{itemize}
\item {} 
\sphinxstyleliteralstrong{\sphinxupquote{sequenceLength}} \textendash{} relativa alle sequenze di \sphinxcode{\sphinxupquote{CodePegs}} impiegate nella partita.

\item {} 
\sphinxstyleliteralstrong{\sphinxupquote{attempts}} \textendash{} massimi per il giocatore Breaker per indovinare.

\item {} 
\sphinxstyleliteralstrong{\sphinxupquote{view}} \textendash{} Istanza della particolare implementazione di \sphinxcode{\sphinxupquote{InteractionView}} scelta per l’istanza di partita in corso.

\item {} 
\sphinxstyleliteralstrong{\sphinxupquote{currentBreaker}} \textendash{} istanza del giocatore che decodifica.

\item {} 
\sphinxstyleliteralstrong{\sphinxupquote{currentMaker}} \textendash{} istanza del giocatore che codifica.

\end{itemize}

\end{description}\end{quote}

\end{fulllineitems}



\subsubsection{Methods}
\label{\detokenize{source/it/unicam/cs/pa/mastermind/gamecore/SingleMatch:methods}}

\paragraph{start}
\label{\detokenize{source/it/unicam/cs/pa/mastermind/gamecore/SingleMatch:start}}\index{start() (Java method)@\spxentry{start()}\spxextra{Java method}}

\begin{fulllineitems}
\phantomsection\label{\detokenize{source/it/unicam/cs/pa/mastermind/gamecore/SingleMatch:it.unicam.cs.pa.mastermind.gamecore.SingleMatch.start()}}\pysiglinewithargsret{public void \sphinxbfcode{\sphinxupquote{start}}}{}{}
Avvio e gestione completa di una singola partita di gioco.

\end{fulllineitems}



\section{it.unicam.cs.pa.mastermind.players}
\label{\detokenize{source/it/unicam/cs/pa/mastermind/players/package-index:it-unicam-cs-pa-mastermind-players}}\label{\detokenize{source/it/unicam/cs/pa/mastermind/players/package-index::doc}}
Nel seguente package sono definiti i due principali attori del gioco, il Maker, colui che decide la sequenza da indovinare, e il Breaker, colui che deve cercare di indovinare la sequenza decisa dal Maker. All’interno del medesimo package è possibile trovare le implementazioni per queste due responsabilità.

\phantomsection\label{\detokenize{source/it/unicam/cs/pa/mastermind/players/package-index:package-it.unicam.cs.pa.mastermind.players}}\index{it.unicam.cs.pa.mastermind.players (package)@\spxentry{it.unicam.cs.pa.mastermind.players}\spxextra{package}}

\subsection{BadRegistryException}
\label{\detokenize{source/it/unicam/cs/pa/mastermind/players/BadRegistryException:badregistryexception}}\label{\detokenize{source/it/unicam/cs/pa/mastermind/players/BadRegistryException::doc}}\index{BadRegistryException (Java class)@\spxentry{BadRegistryException}\spxextra{Java class}}

\begin{fulllineitems}
\phantomsection\label{\detokenize{source/it/unicam/cs/pa/mastermind/players/BadRegistryException:it.unicam.cs.pa.mastermind.players.BadRegistryException}}\pysigline{public class \sphinxbfcode{\sphinxupquote{BadRegistryException}} extends \sphinxhref{http://docs.oracle.com/javase/8/docs/api/java/lang/Exception.html}{Exception}}
Eccezione personalizzata impiegata in tutti quei casi in cui ci sia stato un problema nell’inizializzazione di istanze di \sphinxcode{\sphinxupquote{PlayerFactoryRegistry}}
\begin{quote}\begin{description}
\item[{Author}] \leavevmode
Francesco

\end{description}\end{quote}

\end{fulllineitems}



\subsubsection{Constructors}
\label{\detokenize{source/it/unicam/cs/pa/mastermind/players/BadRegistryException:constructors}}

\paragraph{BadRegistryException}
\label{\detokenize{source/it/unicam/cs/pa/mastermind/players/BadRegistryException:id1}}\index{BadRegistryException(String) (Java constructor)@\spxentry{BadRegistryException(String)}\spxextra{Java constructor}}

\begin{fulllineitems}
\phantomsection\label{\detokenize{source/it/unicam/cs/pa/mastermind/players/BadRegistryException:it.unicam.cs.pa.mastermind.players.BadRegistryException.BadRegistryException(String)}}\pysiglinewithargsret{public \sphinxbfcode{\sphinxupquote{BadRegistryException}}}{\sphinxhref{http://docs.oracle.com/javase/8/docs/api/java/lang/String.html}{String}\sphinxstyleemphasis{ message}}{}
\end{fulllineitems}



\subsection{BreakerFactoryRegistry}
\label{\detokenize{source/it/unicam/cs/pa/mastermind/players/BreakerFactoryRegistry:breakerfactoryregistry}}\label{\detokenize{source/it/unicam/cs/pa/mastermind/players/BreakerFactoryRegistry::doc}}\index{BreakerFactoryRegistry (Java class)@\spxentry{BreakerFactoryRegistry}\spxextra{Java class}}

\begin{fulllineitems}
\phantomsection\label{\detokenize{source/it/unicam/cs/pa/mastermind/players/BreakerFactoryRegistry:it.unicam.cs.pa.mastermind.players.BreakerFactoryRegistry}}\pysigline{public class \sphinxbfcode{\sphinxupquote{BreakerFactoryRegistry}} extends {\hyperref[\detokenize{source/it/unicam/cs/pa/mastermind/players/PlayerFactoryRegistry:it.unicam.cs.pa.mastermind.players.PlayerFactoryRegistry}]{\sphinxcrossref{PlayerFactoryRegistry}}}}
Estensione di \sphinxcode{\sphinxupquote{PlayerFactoryRegistry}} per poter contenere informazioni circa le implementazioni di \sphinxcode{\sphinxupquote{BreakerFactory}}.
\begin{quote}\begin{description}
\item[{Author}] \leavevmode
Francesco Pio Stelluti, Francesco Coppola

\end{description}\end{quote}

\end{fulllineitems}



\subsubsection{Constructors}
\label{\detokenize{source/it/unicam/cs/pa/mastermind/players/BreakerFactoryRegistry:constructors}}

\paragraph{BreakerFactoryRegistry}
\label{\detokenize{source/it/unicam/cs/pa/mastermind/players/BreakerFactoryRegistry:id1}}\index{BreakerFactoryRegistry(String) (Java constructor)@\spxentry{BreakerFactoryRegistry(String)}\spxextra{Java constructor}}

\begin{fulllineitems}
\phantomsection\label{\detokenize{source/it/unicam/cs/pa/mastermind/players/BreakerFactoryRegistry:it.unicam.cs.pa.mastermind.players.BreakerFactoryRegistry.BreakerFactoryRegistry(String)}}\pysiglinewithargsret{public \sphinxbfcode{\sphinxupquote{BreakerFactoryRegistry}}}{\sphinxhref{http://docs.oracle.com/javase/8/docs/api/java/lang/String.html}{String}\sphinxstyleemphasis{ path}}{}
\end{fulllineitems}



\subsection{CodeBreaker}
\label{\detokenize{source/it/unicam/cs/pa/mastermind/players/CodeBreaker:codebreaker}}\label{\detokenize{source/it/unicam/cs/pa/mastermind/players/CodeBreaker::doc}}\index{CodeBreaker (Java class)@\spxentry{CodeBreaker}\spxextra{Java class}}

\begin{fulllineitems}
\phantomsection\label{\detokenize{source/it/unicam/cs/pa/mastermind/players/CodeBreaker:it.unicam.cs.pa.mastermind.players.CodeBreaker}}\pysigline{public abstract class \sphinxbfcode{\sphinxupquote{CodeBreaker}}}
\sphinxstylestrong{Responsabilità}: gestire le interazioni del giocatore Breaker
\begin{quote}\begin{description}
\item[{Author}] \leavevmode
Francesco Pio Stelluti, Francesco Coppola

\end{description}\end{quote}

\end{fulllineitems}



\subsubsection{Methods}
\label{\detokenize{source/it/unicam/cs/pa/mastermind/players/CodeBreaker:methods}}

\paragraph{getAttempt}
\label{\detokenize{source/it/unicam/cs/pa/mastermind/players/CodeBreaker:getattempt}}\index{getAttempt(InteractionView) (Java method)@\spxentry{getAttempt(InteractionView)}\spxextra{Java method}}

\begin{fulllineitems}
\phantomsection\label{\detokenize{source/it/unicam/cs/pa/mastermind/players/CodeBreaker:it.unicam.cs.pa.mastermind.players.CodeBreaker.getAttempt(InteractionView)}}\pysiglinewithargsret{public abstract \sphinxhref{http://docs.oracle.com/javase/8/docs/api/java/util/List.html}{List}\textless{}{\hyperref[\detokenize{source/it/unicam/cs/pa/mastermind/gamecore/ColorPegs:it.unicam.cs.pa.mastermind.gamecore.ColorPegs}]{\sphinxcrossref{ColorPegs}}}\textgreater{} \sphinxbfcode{\sphinxupquote{getAttempt}}}{{\hyperref[\detokenize{source/it/unicam/cs/pa/mastermind/ui/InteractionView:it.unicam.cs.pa.mastermind.ui.InteractionView}]{\sphinxcrossref{InteractionView}}}\sphinxstyleemphasis{ intView}}{}
Restituisce la sequenza di \sphinxcode{\sphinxupquote{ColorPegs}} valida come singolo tentativo.
\begin{quote}\begin{description}
\item[{Parametri}] \leavevmode\begin{itemize}
\item {} 
\sphinxstyleliteralstrong{\sphinxupquote{intView}} \textendash{} necessario per ottenere informazioni riguardo il gioco

\end{itemize}

\item[{Ritorna}] \leavevmode
List di \sphinxcode{\sphinxupquote{ColorPegs}} valida come singolo tentativo

\end{description}\end{quote}

\end{fulllineitems}



\paragraph{hasGivenUp}
\label{\detokenize{source/it/unicam/cs/pa/mastermind/players/CodeBreaker:hasgivenup}}\index{hasGivenUp() (Java method)@\spxentry{hasGivenUp()}\spxextra{Java method}}

\begin{fulllineitems}
\phantomsection\label{\detokenize{source/it/unicam/cs/pa/mastermind/players/CodeBreaker:it.unicam.cs.pa.mastermind.players.CodeBreaker.hasGivenUp()}}\pysiglinewithargsret{public boolean \sphinxbfcode{\sphinxupquote{hasGivenUp}}}{}{}~\begin{quote}\begin{description}
\item[{Ritorna}] \leavevmode
la volontà del giocatore \sphinxcode{\sphinxupquote{CodeBreaker}} di arrendersi

\end{description}\end{quote}

\end{fulllineitems}



\paragraph{toggleGiveUp}
\label{\detokenize{source/it/unicam/cs/pa/mastermind/players/CodeBreaker:togglegiveup}}\index{toggleGiveUp() (Java method)@\spxentry{toggleGiveUp()}\spxextra{Java method}}

\begin{fulllineitems}
\phantomsection\label{\detokenize{source/it/unicam/cs/pa/mastermind/players/CodeBreaker:it.unicam.cs.pa.mastermind.players.CodeBreaker.toggleGiveUp()}}\pysiglinewithargsret{public void \sphinxbfcode{\sphinxupquote{toggleGiveUp}}}{}{}
Imposta la volontà del giocatore \sphinxcode{\sphinxupquote{CodeBreaker}} di arrendersi.

\end{fulllineitems}



\subsection{CodeMaker}
\label{\detokenize{source/it/unicam/cs/pa/mastermind/players/CodeMaker:codemaker}}\label{\detokenize{source/it/unicam/cs/pa/mastermind/players/CodeMaker::doc}}\index{CodeMaker (Java class)@\spxentry{CodeMaker}\spxextra{Java class}}

\begin{fulllineitems}
\phantomsection\label{\detokenize{source/it/unicam/cs/pa/mastermind/players/CodeMaker:it.unicam.cs.pa.mastermind.players.CodeMaker}}\pysigline{public abstract class \sphinxbfcode{\sphinxupquote{CodeMaker}}}
\sphinxstylestrong{Responsabilità}: gestire le interazioni del giocatore Maker
\begin{quote}\begin{description}
\item[{Author}] \leavevmode
Francesco Pio Stelluti, Francesco Coppola

\end{description}\end{quote}

\end{fulllineitems}



\subsubsection{Methods}
\label{\detokenize{source/it/unicam/cs/pa/mastermind/players/CodeMaker:methods}}

\paragraph{getCodeToGuess}
\label{\detokenize{source/it/unicam/cs/pa/mastermind/players/CodeMaker:getcodetoguess}}\index{getCodeToGuess(InteractionView) (Java method)@\spxentry{getCodeToGuess(InteractionView)}\spxextra{Java method}}

\begin{fulllineitems}
\phantomsection\label{\detokenize{source/it/unicam/cs/pa/mastermind/players/CodeMaker:it.unicam.cs.pa.mastermind.players.CodeMaker.getCodeToGuess(InteractionView)}}\pysiglinewithargsret{public abstract \sphinxhref{http://docs.oracle.com/javase/8/docs/api/java/util/List.html}{List}\textless{}{\hyperref[\detokenize{source/it/unicam/cs/pa/mastermind/gamecore/ColorPegs:it.unicam.cs.pa.mastermind.gamecore.ColorPegs}]{\sphinxcrossref{ColorPegs}}}\textgreater{} \sphinxbfcode{\sphinxupquote{getCodeToGuess}}}{{\hyperref[\detokenize{source/it/unicam/cs/pa/mastermind/ui/InteractionView:it.unicam.cs.pa.mastermind.ui.InteractionView}]{\sphinxcrossref{InteractionView}}}\sphinxstyleemphasis{ intView}}{}
Restituisce la sequenza di \sphinxcode{\sphinxupquote{ColorPegs}} valida come sequenza da indovinare.
\begin{quote}\begin{description}
\item[{Parametri}] \leavevmode\begin{itemize}
\item {} 
\sphinxstyleliteralstrong{\sphinxupquote{intView}} \textendash{} necessario per ottenere informazioni riguardo il gioco

\end{itemize}

\item[{Ritorna}] \leavevmode
List di \sphinxcode{\sphinxupquote{ColorPegs}} valida come sequenza da indovinare

\end{description}\end{quote}

\end{fulllineitems}



\subsection{InteractiveBreaker}
\label{\detokenize{source/it/unicam/cs/pa/mastermind/players/InteractiveBreaker:interactivebreaker}}\label{\detokenize{source/it/unicam/cs/pa/mastermind/players/InteractiveBreaker::doc}}\index{InteractiveBreaker (Java class)@\spxentry{InteractiveBreaker}\spxextra{Java class}}

\begin{fulllineitems}
\phantomsection\label{\detokenize{source/it/unicam/cs/pa/mastermind/players/InteractiveBreaker:it.unicam.cs.pa.mastermind.players.InteractiveBreaker}}\pysigline{public class \sphinxbfcode{\sphinxupquote{InteractiveBreaker}} extends {\hyperref[\detokenize{source/it/unicam/cs/pa/mastermind/players/CodeBreaker:it.unicam.cs.pa.mastermind.players.CodeBreaker}]{\sphinxcrossref{CodeBreaker}}}}
Estensione di \sphinxcode{\sphinxupquote{CodeBreaker}} mirata ad una gestione del comportamento del giocatore tramite interazioni con l’utente umano.
\begin{quote}\begin{description}
\item[{Author}] \leavevmode
Francesco Pio Stelluti, Francesco Coppola

\end{description}\end{quote}

\end{fulllineitems}



\subsubsection{Constructors}
\label{\detokenize{source/it/unicam/cs/pa/mastermind/players/InteractiveBreaker:constructors}}

\paragraph{InteractiveBreaker}
\label{\detokenize{source/it/unicam/cs/pa/mastermind/players/InteractiveBreaker:id1}}\index{InteractiveBreaker() (Java constructor)@\spxentry{InteractiveBreaker()}\spxextra{Java constructor}}

\begin{fulllineitems}
\phantomsection\label{\detokenize{source/it/unicam/cs/pa/mastermind/players/InteractiveBreaker:it.unicam.cs.pa.mastermind.players.InteractiveBreaker.InteractiveBreaker()}}\pysiglinewithargsret{public \sphinxbfcode{\sphinxupquote{InteractiveBreaker}}}{}{}
\end{fulllineitems}



\subsubsection{Methods}
\label{\detokenize{source/it/unicam/cs/pa/mastermind/players/InteractiveBreaker:methods}}

\paragraph{getAttempt}
\label{\detokenize{source/it/unicam/cs/pa/mastermind/players/InteractiveBreaker:getattempt}}\index{getAttempt(InteractionView) (Java method)@\spxentry{getAttempt(InteractionView)}\spxextra{Java method}}

\begin{fulllineitems}
\phantomsection\label{\detokenize{source/it/unicam/cs/pa/mastermind/players/InteractiveBreaker:it.unicam.cs.pa.mastermind.players.InteractiveBreaker.getAttempt(InteractionView)}}\pysiglinewithargsret{public \sphinxhref{http://docs.oracle.com/javase/8/docs/api/java/util/List.html}{List}\textless{}{\hyperref[\detokenize{source/it/unicam/cs/pa/mastermind/gamecore/ColorPegs:it.unicam.cs.pa.mastermind.gamecore.ColorPegs}]{\sphinxcrossref{ColorPegs}}}\textgreater{} \sphinxbfcode{\sphinxupquote{getAttempt}}}{{\hyperref[\detokenize{source/it/unicam/cs/pa/mastermind/ui/InteractionView:it.unicam.cs.pa.mastermind.ui.InteractionView}]{\sphinxcrossref{InteractionView}}}\sphinxstyleemphasis{ intView}}{}
\end{fulllineitems}



\subsection{InteractiveMaker}
\label{\detokenize{source/it/unicam/cs/pa/mastermind/players/InteractiveMaker:interactivemaker}}\label{\detokenize{source/it/unicam/cs/pa/mastermind/players/InteractiveMaker::doc}}\index{InteractiveMaker (Java class)@\spxentry{InteractiveMaker}\spxextra{Java class}}

\begin{fulllineitems}
\phantomsection\label{\detokenize{source/it/unicam/cs/pa/mastermind/players/InteractiveMaker:it.unicam.cs.pa.mastermind.players.InteractiveMaker}}\pysigline{public class \sphinxbfcode{\sphinxupquote{InteractiveMaker}} extends {\hyperref[\detokenize{source/it/unicam/cs/pa/mastermind/players/CodeMaker:it.unicam.cs.pa.mastermind.players.CodeMaker}]{\sphinxcrossref{CodeMaker}}}}
Estensione di \sphinxcode{\sphinxupquote{CodeMaker}} mirata ad una gestione del comportamento del giocatore tramite interazioni con l’utente umano.
\begin{quote}\begin{description}
\item[{Author}] \leavevmode
Francesco Pio Stelluti, Francesco Coppola

\end{description}\end{quote}

\end{fulllineitems}



\subsubsection{Methods}
\label{\detokenize{source/it/unicam/cs/pa/mastermind/players/InteractiveMaker:methods}}

\paragraph{getCodeToGuess}
\label{\detokenize{source/it/unicam/cs/pa/mastermind/players/InteractiveMaker:getcodetoguess}}\index{getCodeToGuess(InteractionView) (Java method)@\spxentry{getCodeToGuess(InteractionView)}\spxextra{Java method}}

\begin{fulllineitems}
\phantomsection\label{\detokenize{source/it/unicam/cs/pa/mastermind/players/InteractiveMaker:it.unicam.cs.pa.mastermind.players.InteractiveMaker.getCodeToGuess(InteractionView)}}\pysiglinewithargsret{public \sphinxhref{http://docs.oracle.com/javase/8/docs/api/java/util/List.html}{List}\textless{}{\hyperref[\detokenize{source/it/unicam/cs/pa/mastermind/gamecore/ColorPegs:it.unicam.cs.pa.mastermind.gamecore.ColorPegs}]{\sphinxcrossref{ColorPegs}}}\textgreater{} \sphinxbfcode{\sphinxupquote{getCodeToGuess}}}{{\hyperref[\detokenize{source/it/unicam/cs/pa/mastermind/ui/InteractionView:it.unicam.cs.pa.mastermind.ui.InteractionView}]{\sphinxcrossref{InteractionView}}}\sphinxstyleemphasis{ intView}}{}
\end{fulllineitems}



\subsection{MakerFactoryRegistry}
\label{\detokenize{source/it/unicam/cs/pa/mastermind/players/MakerFactoryRegistry:makerfactoryregistry}}\label{\detokenize{source/it/unicam/cs/pa/mastermind/players/MakerFactoryRegistry::doc}}\index{MakerFactoryRegistry (Java class)@\spxentry{MakerFactoryRegistry}\spxextra{Java class}}

\begin{fulllineitems}
\phantomsection\label{\detokenize{source/it/unicam/cs/pa/mastermind/players/MakerFactoryRegistry:it.unicam.cs.pa.mastermind.players.MakerFactoryRegistry}}\pysigline{public class \sphinxbfcode{\sphinxupquote{MakerFactoryRegistry}} extends {\hyperref[\detokenize{source/it/unicam/cs/pa/mastermind/players/PlayerFactoryRegistry:it.unicam.cs.pa.mastermind.players.PlayerFactoryRegistry}]{\sphinxcrossref{PlayerFactoryRegistry}}}}
Estensione di \sphinxcode{\sphinxupquote{PlayerFactoryRegistry}} per poter contenere informazioni circa le implementazioni di \sphinxcode{\sphinxupquote{MakerFactory}}.
\begin{quote}\begin{description}
\item[{Author}] \leavevmode
Francesco Pio Stelluti, Francesco Coppola

\end{description}\end{quote}

\end{fulllineitems}



\subsubsection{Constructors}
\label{\detokenize{source/it/unicam/cs/pa/mastermind/players/MakerFactoryRegistry:constructors}}

\paragraph{MakerFactoryRegistry}
\label{\detokenize{source/it/unicam/cs/pa/mastermind/players/MakerFactoryRegistry:id1}}\index{MakerFactoryRegistry(String) (Java constructor)@\spxentry{MakerFactoryRegistry(String)}\spxextra{Java constructor}}

\begin{fulllineitems}
\phantomsection\label{\detokenize{source/it/unicam/cs/pa/mastermind/players/MakerFactoryRegistry:it.unicam.cs.pa.mastermind.players.MakerFactoryRegistry.MakerFactoryRegistry(String)}}\pysiglinewithargsret{public \sphinxbfcode{\sphinxupquote{MakerFactoryRegistry}}}{\sphinxhref{http://docs.oracle.com/javase/8/docs/api/java/lang/String.html}{String}\sphinxstyleemphasis{ path}}{}
\end{fulllineitems}



\subsection{PlayerFactoryRegistry}
\label{\detokenize{source/it/unicam/cs/pa/mastermind/players/PlayerFactoryRegistry:playerfactoryregistry}}\label{\detokenize{source/it/unicam/cs/pa/mastermind/players/PlayerFactoryRegistry::doc}}\index{PlayerFactoryRegistry (Java class)@\spxentry{PlayerFactoryRegistry}\spxextra{Java class}}

\begin{fulllineitems}
\phantomsection\label{\detokenize{source/it/unicam/cs/pa/mastermind/players/PlayerFactoryRegistry:it.unicam.cs.pa.mastermind.players.PlayerFactoryRegistry}}\pysigline{public abstract class \sphinxbfcode{\sphinxupquote{PlayerFactoryRegistry}}}
\sphinxstylestrong{Responsabilità}: gestione dinamica delle implementazioni delle classi factory di \sphinxcode{\sphinxupquote{CodeMaker}} e \sphinxcode{\sphinxupquote{CodeBreaker}}. Classe astratta estendibile da classi rappresentanti registri contenenti informazioni sulle classi factory impiegate per istanziare le implementazioni dei giocatori.
\begin{quote}\begin{description}
\item[{Author}] \leavevmode
Francesco Pio Stelluti, Francesco Coppola

\end{description}\end{quote}

\end{fulllineitems}



\subsubsection{Constructors}
\label{\detokenize{source/it/unicam/cs/pa/mastermind/players/PlayerFactoryRegistry:constructors}}

\paragraph{PlayerFactoryRegistry}
\label{\detokenize{source/it/unicam/cs/pa/mastermind/players/PlayerFactoryRegistry:id1}}\index{PlayerFactoryRegistry(String) (Java constructor)@\spxentry{PlayerFactoryRegistry(String)}\spxextra{Java constructor}}

\begin{fulllineitems}
\phantomsection\label{\detokenize{source/it/unicam/cs/pa/mastermind/players/PlayerFactoryRegistry:it.unicam.cs.pa.mastermind.players.PlayerFactoryRegistry.PlayerFactoryRegistry(String)}}\pysiglinewithargsret{public \sphinxbfcode{\sphinxupquote{PlayerFactoryRegistry}}}{\sphinxhref{http://docs.oracle.com/javase/8/docs/api/java/lang/String.html}{String}\sphinxstyleemphasis{ pathLettura}}{}
Costruttore di \sphinxcode{\sphinxupquote{PlayerFactoryRegistry}}.
\begin{quote}\begin{description}
\item[{Parametri}] \leavevmode\begin{itemize}
\item {} 
\sphinxstyleliteralstrong{\sphinxupquote{pathLettura}} \textendash{} associato al file da cui leggere informazioni da inserire all’interno di \sphinxcode{\sphinxupquote{registryFactoryPlayers}}.

\end{itemize}

\item[{Solleva}] \leavevmode\begin{itemize}
\item {} 
\sphinxstyleliteralstrong{\sphinxupquote{BadRegistryException}} \textendash{} in caso ci siano stati errori nell’inizializzazione del registro

\end{itemize}

\end{description}\end{quote}

\end{fulllineitems}



\subsubsection{Methods}
\label{\detokenize{source/it/unicam/cs/pa/mastermind/players/PlayerFactoryRegistry:methods}}

\paragraph{getFactoryByName}
\label{\detokenize{source/it/unicam/cs/pa/mastermind/players/PlayerFactoryRegistry:getfactorybyname}}\index{getFactoryByName(String) (Java method)@\spxentry{getFactoryByName(String)}\spxextra{Java method}}

\begin{fulllineitems}
\phantomsection\label{\detokenize{source/it/unicam/cs/pa/mastermind/players/PlayerFactoryRegistry:it.unicam.cs.pa.mastermind.players.PlayerFactoryRegistry.getFactoryByName(String)}}\pysiglinewithargsret{public {\hyperref[\detokenize{source/it/unicam/cs/pa/mastermind/factories/PlayerFactory:it.unicam.cs.pa.mastermind.factories.PlayerFactory}]{\sphinxcrossref{PlayerFactory}}} \sphinxbfcode{\sphinxupquote{getFactoryByName}}}{\sphinxhref{http://docs.oracle.com/javase/8/docs/api/java/lang/String.html}{String}\sphinxstyleemphasis{ name}}{}
Ottenimento di istanze di \sphinxcode{\sphinxupquote{PlayerFactory}}.
\begin{quote}\begin{description}
\item[{Parametri}] \leavevmode\begin{itemize}
\item {} 
\sphinxstyleliteralstrong{\sphinxupquote{name}} \textendash{} nome associato all’istanza di \sphinxcode{\sphinxupquote{PlayerFactory}}

\end{itemize}

\item[{Ritorna}] \leavevmode
PlayerFactory associato al nome

\end{description}\end{quote}

\end{fulllineitems}



\paragraph{getPlayerFactoriesInstances}
\label{\detokenize{source/it/unicam/cs/pa/mastermind/players/PlayerFactoryRegistry:getplayerfactoriesinstances}}\index{getPlayerFactoriesInstances() (Java method)@\spxentry{getPlayerFactoriesInstances()}\spxextra{Java method}}

\begin{fulllineitems}
\phantomsection\label{\detokenize{source/it/unicam/cs/pa/mastermind/players/PlayerFactoryRegistry:it.unicam.cs.pa.mastermind.players.PlayerFactoryRegistry.getPlayerFactoriesInstances()}}\pysiglinewithargsret{public \sphinxhref{http://docs.oracle.com/javase/8/docs/api/java/util/List.html}{List}\textless{}{\hyperref[\detokenize{source/it/unicam/cs/pa/mastermind/factories/PlayerFactory:it.unicam.cs.pa.mastermind.factories.PlayerFactory}]{\sphinxcrossref{PlayerFactory}}}\textgreater{} \sphinxbfcode{\sphinxupquote{getPlayerFactoriesInstances}}}{}{}~\begin{quote}\begin{description}
\item[{Ritorna}] \leavevmode
List contenente le istanze di \sphinxcode{\sphinxupquote{PlayerFactory}} presenti in \sphinxcode{\sphinxupquote{registryFactoryPlayers}}

\end{description}\end{quote}

\end{fulllineitems}



\paragraph{getPlayersNames}
\label{\detokenize{source/it/unicam/cs/pa/mastermind/players/PlayerFactoryRegistry:getplayersnames}}\index{getPlayersNames() (Java method)@\spxentry{getPlayersNames()}\spxextra{Java method}}

\begin{fulllineitems}
\phantomsection\label{\detokenize{source/it/unicam/cs/pa/mastermind/players/PlayerFactoryRegistry:it.unicam.cs.pa.mastermind.players.PlayerFactoryRegistry.getPlayersNames()}}\pysiglinewithargsret{public \sphinxhref{http://docs.oracle.com/javase/8/docs/api/java/util/List.html}{List}\textless{}\sphinxhref{http://docs.oracle.com/javase/8/docs/api/java/lang/String.html}{String}\textgreater{} \sphinxbfcode{\sphinxupquote{getPlayersNames}}}{}{}~\begin{quote}\begin{description}
\item[{Ritorna}] \leavevmode
List contenente i nomi associati alle istanze di \sphinxcode{\sphinxupquote{PlayerFactory}} presenti in \sphinxcode{\sphinxupquote{registryFactoryPlayers}}

\end{description}\end{quote}

\end{fulllineitems}



\subsection{RandomBotBreaker}
\label{\detokenize{source/it/unicam/cs/pa/mastermind/players/RandomBotBreaker:randombotbreaker}}\label{\detokenize{source/it/unicam/cs/pa/mastermind/players/RandomBotBreaker::doc}}\index{RandomBotBreaker (Java class)@\spxentry{RandomBotBreaker}\spxextra{Java class}}

\begin{fulllineitems}
\phantomsection\label{\detokenize{source/it/unicam/cs/pa/mastermind/players/RandomBotBreaker:it.unicam.cs.pa.mastermind.players.RandomBotBreaker}}\pysigline{public class \sphinxbfcode{\sphinxupquote{RandomBotBreaker}} extends {\hyperref[\detokenize{source/it/unicam/cs/pa/mastermind/players/CodeBreaker:it.unicam.cs.pa.mastermind.players.CodeBreaker}]{\sphinxcrossref{CodeBreaker}}}}
Estensione di \sphinxcode{\sphinxupquote{CodeBreaker}} mirata ad una gestione del comportamento del giocatore parzialmente random.
\begin{quote}\begin{description}
\item[{Author}] \leavevmode
Francesco Pio Stelluti, Francesco Coppola

\end{description}\end{quote}

\end{fulllineitems}



\subsubsection{Constructors}
\label{\detokenize{source/it/unicam/cs/pa/mastermind/players/RandomBotBreaker:constructors}}

\paragraph{RandomBotBreaker}
\label{\detokenize{source/it/unicam/cs/pa/mastermind/players/RandomBotBreaker:id1}}\index{RandomBotBreaker() (Java constructor)@\spxentry{RandomBotBreaker()}\spxextra{Java constructor}}

\begin{fulllineitems}
\phantomsection\label{\detokenize{source/it/unicam/cs/pa/mastermind/players/RandomBotBreaker:it.unicam.cs.pa.mastermind.players.RandomBotBreaker.RandomBotBreaker()}}\pysiglinewithargsret{public \sphinxbfcode{\sphinxupquote{RandomBotBreaker}}}{}{}
\end{fulllineitems}



\subsubsection{Methods}
\label{\detokenize{source/it/unicam/cs/pa/mastermind/players/RandomBotBreaker:methods}}

\paragraph{getAttempt}
\label{\detokenize{source/it/unicam/cs/pa/mastermind/players/RandomBotBreaker:getattempt}}\index{getAttempt(InteractionView) (Java method)@\spxentry{getAttempt(InteractionView)}\spxextra{Java method}}

\begin{fulllineitems}
\phantomsection\label{\detokenize{source/it/unicam/cs/pa/mastermind/players/RandomBotBreaker:it.unicam.cs.pa.mastermind.players.RandomBotBreaker.getAttempt(InteractionView)}}\pysiglinewithargsret{public \sphinxhref{http://docs.oracle.com/javase/8/docs/api/java/util/List.html}{List}\textless{}{\hyperref[\detokenize{source/it/unicam/cs/pa/mastermind/gamecore/ColorPegs:it.unicam.cs.pa.mastermind.gamecore.ColorPegs}]{\sphinxcrossref{ColorPegs}}}\textgreater{} \sphinxbfcode{\sphinxupquote{getAttempt}}}{{\hyperref[\detokenize{source/it/unicam/cs/pa/mastermind/ui/InteractionView:it.unicam.cs.pa.mastermind.ui.InteractionView}]{\sphinxcrossref{InteractionView}}}\sphinxstyleemphasis{ intView}}{}
\end{fulllineitems}



\subsection{RandomBotMaker}
\label{\detokenize{source/it/unicam/cs/pa/mastermind/players/RandomBotMaker:randombotmaker}}\label{\detokenize{source/it/unicam/cs/pa/mastermind/players/RandomBotMaker::doc}}\index{RandomBotMaker (Java class)@\spxentry{RandomBotMaker}\spxextra{Java class}}

\begin{fulllineitems}
\phantomsection\label{\detokenize{source/it/unicam/cs/pa/mastermind/players/RandomBotMaker:it.unicam.cs.pa.mastermind.players.RandomBotMaker}}\pysigline{public class \sphinxbfcode{\sphinxupquote{RandomBotMaker}} extends {\hyperref[\detokenize{source/it/unicam/cs/pa/mastermind/players/CodeMaker:it.unicam.cs.pa.mastermind.players.CodeMaker}]{\sphinxcrossref{CodeMaker}}}}
Estensione di \sphinxcode{\sphinxupquote{CodeMaker}} mirata ad una gestione del comportamento del giocatore totalmente random.
\begin{quote}\begin{description}
\item[{Author}] \leavevmode
Francesco Pio Stelluti, Francesco Coppola

\end{description}\end{quote}

\end{fulllineitems}



\subsubsection{Methods}
\label{\detokenize{source/it/unicam/cs/pa/mastermind/players/RandomBotMaker:methods}}

\paragraph{getCodeToGuess}
\label{\detokenize{source/it/unicam/cs/pa/mastermind/players/RandomBotMaker:getcodetoguess}}\index{getCodeToGuess(InteractionView) (Java method)@\spxentry{getCodeToGuess(InteractionView)}\spxextra{Java method}}

\begin{fulllineitems}
\phantomsection\label{\detokenize{source/it/unicam/cs/pa/mastermind/players/RandomBotMaker:it.unicam.cs.pa.mastermind.players.RandomBotMaker.getCodeToGuess(InteractionView)}}\pysiglinewithargsret{public \sphinxhref{http://docs.oracle.com/javase/8/docs/api/java/util/List.html}{List}\textless{}{\hyperref[\detokenize{source/it/unicam/cs/pa/mastermind/gamecore/ColorPegs:it.unicam.cs.pa.mastermind.gamecore.ColorPegs}]{\sphinxcrossref{ColorPegs}}}\textgreater{} \sphinxbfcode{\sphinxupquote{getCodeToGuess}}}{{\hyperref[\detokenize{source/it/unicam/cs/pa/mastermind/ui/InteractionView:it.unicam.cs.pa.mastermind.ui.InteractionView}]{\sphinxcrossref{InteractionView}}}\sphinxstyleemphasis{ intView}}{}
\end{fulllineitems}



\section{it.unicam.cs.pa.mastermind.ui}
\label{\detokenize{source/it/unicam/cs/pa/mastermind/ui/package-index:it-unicam-cs-pa-mastermind-ui}}\label{\detokenize{source/it/unicam/cs/pa/mastermind/ui/package-index::doc}}
Il seguente package contiene le classi relative a tutto ciò che concerne l’interfaccia di gioco con la quale comunicherà l’utente, sia esso o meno un giocatore attivo nel gioco. Attraverso le classi di questo package è possibile avere le interazioni iniziali con il programma e le interazioni durante lo svolgimento delle partite.

\phantomsection\label{\detokenize{source/it/unicam/cs/pa/mastermind/ui/package-index:package-it.unicam.cs.pa.mastermind.ui}}\index{it.unicam.cs.pa.mastermind.ui (package)@\spxentry{it.unicam.cs.pa.mastermind.ui}\spxextra{package}}

\subsection{AnsiUtility}
\label{\detokenize{source/it/unicam/cs/pa/mastermind/ui/AnsiUtility:ansiutility}}\label{\detokenize{source/it/unicam/cs/pa/mastermind/ui/AnsiUtility::doc}}\index{AnsiUtility (Java class)@\spxentry{AnsiUtility}\spxextra{Java class}}

\begin{fulllineitems}
\phantomsection\label{\detokenize{source/it/unicam/cs/pa/mastermind/ui/AnsiUtility:it.unicam.cs.pa.mastermind.ui.AnsiUtility}}\pysigline{public class \sphinxbfcode{\sphinxupquote{AnsiUtility}}}
La seguente classe ha il solo scopo di rendere la console di gioco più accattivante e user-friendly andando ad aggiungere una nota di colore ai vari ColorPegs che verranno inseriti.
\begin{quote}\begin{description}
\item[{Author}] \leavevmode
Francesco Pio Stelluti, Francesco Coppola

\end{description}\end{quote}

\end{fulllineitems}



\subsubsection{Fields}
\label{\detokenize{source/it/unicam/cs/pa/mastermind/ui/AnsiUtility:fields}}

\paragraph{ANSI\_BLACK\_BACKGROUND}
\label{\detokenize{source/it/unicam/cs/pa/mastermind/ui/AnsiUtility:ansi-black-background}}\index{ANSI\_BLACK\_BACKGROUND (Java field)@\spxentry{ANSI\_BLACK\_BACKGROUND}\spxextra{Java field}}

\begin{fulllineitems}
\phantomsection\label{\detokenize{source/it/unicam/cs/pa/mastermind/ui/AnsiUtility:it.unicam.cs.pa.mastermind.ui.AnsiUtility.ANSI_BLACK_BACKGROUND}}\pysigline{public static final \sphinxhref{http://docs.oracle.com/javase/8/docs/api/java/lang/String.html}{String} \sphinxbfcode{\sphinxupquote{ANSI\_BLACK\_BACKGROUND}}}
\end{fulllineitems}



\paragraph{ANSI\_BLUE\_BACKGROUND}
\label{\detokenize{source/it/unicam/cs/pa/mastermind/ui/AnsiUtility:ansi-blue-background}}\index{ANSI\_BLUE\_BACKGROUND (Java field)@\spxentry{ANSI\_BLUE\_BACKGROUND}\spxextra{Java field}}

\begin{fulllineitems}
\phantomsection\label{\detokenize{source/it/unicam/cs/pa/mastermind/ui/AnsiUtility:it.unicam.cs.pa.mastermind.ui.AnsiUtility.ANSI_BLUE_BACKGROUND}}\pysigline{public static final \sphinxhref{http://docs.oracle.com/javase/8/docs/api/java/lang/String.html}{String} \sphinxbfcode{\sphinxupquote{ANSI\_BLUE\_BACKGROUND}}}
\end{fulllineitems}



\paragraph{ANSI\_CYAN\_BACKGROUND}
\label{\detokenize{source/it/unicam/cs/pa/mastermind/ui/AnsiUtility:ansi-cyan-background}}\index{ANSI\_CYAN\_BACKGROUND (Java field)@\spxentry{ANSI\_CYAN\_BACKGROUND}\spxextra{Java field}}

\begin{fulllineitems}
\phantomsection\label{\detokenize{source/it/unicam/cs/pa/mastermind/ui/AnsiUtility:it.unicam.cs.pa.mastermind.ui.AnsiUtility.ANSI_CYAN_BACKGROUND}}\pysigline{public static final \sphinxhref{http://docs.oracle.com/javase/8/docs/api/java/lang/String.html}{String} \sphinxbfcode{\sphinxupquote{ANSI\_CYAN\_BACKGROUND}}}
\end{fulllineitems}



\paragraph{ANSI\_CYAN\_BOLD}
\label{\detokenize{source/it/unicam/cs/pa/mastermind/ui/AnsiUtility:ansi-cyan-bold}}\index{ANSI\_CYAN\_BOLD (Java field)@\spxentry{ANSI\_CYAN\_BOLD}\spxextra{Java field}}

\begin{fulllineitems}
\phantomsection\label{\detokenize{source/it/unicam/cs/pa/mastermind/ui/AnsiUtility:it.unicam.cs.pa.mastermind.ui.AnsiUtility.ANSI_CYAN_BOLD}}\pysigline{public static final \sphinxhref{http://docs.oracle.com/javase/8/docs/api/java/lang/String.html}{String} \sphinxbfcode{\sphinxupquote{ANSI\_CYAN\_BOLD}}}
\end{fulllineitems}



\paragraph{ANSI\_GREEN\_BACKGROUND}
\label{\detokenize{source/it/unicam/cs/pa/mastermind/ui/AnsiUtility:ansi-green-background}}\index{ANSI\_GREEN\_BACKGROUND (Java field)@\spxentry{ANSI\_GREEN\_BACKGROUND}\spxextra{Java field}}

\begin{fulllineitems}
\phantomsection\label{\detokenize{source/it/unicam/cs/pa/mastermind/ui/AnsiUtility:it.unicam.cs.pa.mastermind.ui.AnsiUtility.ANSI_GREEN_BACKGROUND}}\pysigline{public static final \sphinxhref{http://docs.oracle.com/javase/8/docs/api/java/lang/String.html}{String} \sphinxbfcode{\sphinxupquote{ANSI\_GREEN\_BACKGROUND}}}
\end{fulllineitems}



\paragraph{ANSI\_PURPLE\_BACKGROUND}
\label{\detokenize{source/it/unicam/cs/pa/mastermind/ui/AnsiUtility:ansi-purple-background}}\index{ANSI\_PURPLE\_BACKGROUND (Java field)@\spxentry{ANSI\_PURPLE\_BACKGROUND}\spxextra{Java field}}

\begin{fulllineitems}
\phantomsection\label{\detokenize{source/it/unicam/cs/pa/mastermind/ui/AnsiUtility:it.unicam.cs.pa.mastermind.ui.AnsiUtility.ANSI_PURPLE_BACKGROUND}}\pysigline{public static final \sphinxhref{http://docs.oracle.com/javase/8/docs/api/java/lang/String.html}{String} \sphinxbfcode{\sphinxupquote{ANSI\_PURPLE\_BACKGROUND}}}
\end{fulllineitems}



\paragraph{ANSI\_RED\_BACKGROUND}
\label{\detokenize{source/it/unicam/cs/pa/mastermind/ui/AnsiUtility:ansi-red-background}}\index{ANSI\_RED\_BACKGROUND (Java field)@\spxentry{ANSI\_RED\_BACKGROUND}\spxextra{Java field}}

\begin{fulllineitems}
\phantomsection\label{\detokenize{source/it/unicam/cs/pa/mastermind/ui/AnsiUtility:it.unicam.cs.pa.mastermind.ui.AnsiUtility.ANSI_RED_BACKGROUND}}\pysigline{public static final \sphinxhref{http://docs.oracle.com/javase/8/docs/api/java/lang/String.html}{String} \sphinxbfcode{\sphinxupquote{ANSI\_RED\_BACKGROUND}}}
\end{fulllineitems}



\paragraph{ANSI\_RED\_BOLD}
\label{\detokenize{source/it/unicam/cs/pa/mastermind/ui/AnsiUtility:ansi-red-bold}}\index{ANSI\_RED\_BOLD (Java field)@\spxentry{ANSI\_RED\_BOLD}\spxextra{Java field}}

\begin{fulllineitems}
\phantomsection\label{\detokenize{source/it/unicam/cs/pa/mastermind/ui/AnsiUtility:it.unicam.cs.pa.mastermind.ui.AnsiUtility.ANSI_RED_BOLD}}\pysigline{public static final \sphinxhref{http://docs.oracle.com/javase/8/docs/api/java/lang/String.html}{String} \sphinxbfcode{\sphinxupquote{ANSI\_RED\_BOLD}}}
\end{fulllineitems}



\paragraph{ANSI\_RESET}
\label{\detokenize{source/it/unicam/cs/pa/mastermind/ui/AnsiUtility:ansi-reset}}\index{ANSI\_RESET (Java field)@\spxentry{ANSI\_RESET}\spxextra{Java field}}

\begin{fulllineitems}
\phantomsection\label{\detokenize{source/it/unicam/cs/pa/mastermind/ui/AnsiUtility:it.unicam.cs.pa.mastermind.ui.AnsiUtility.ANSI_RESET}}\pysigline{public static final \sphinxhref{http://docs.oracle.com/javase/8/docs/api/java/lang/String.html}{String} \sphinxbfcode{\sphinxupquote{ANSI\_RESET}}}
\end{fulllineitems}



\paragraph{ANSI\_WHITE\_BACKGROUND}
\label{\detokenize{source/it/unicam/cs/pa/mastermind/ui/AnsiUtility:ansi-white-background}}\index{ANSI\_WHITE\_BACKGROUND (Java field)@\spxentry{ANSI\_WHITE\_BACKGROUND}\spxextra{Java field}}

\begin{fulllineitems}
\phantomsection\label{\detokenize{source/it/unicam/cs/pa/mastermind/ui/AnsiUtility:it.unicam.cs.pa.mastermind.ui.AnsiUtility.ANSI_WHITE_BACKGROUND}}\pysigline{public static final \sphinxhref{http://docs.oracle.com/javase/8/docs/api/java/lang/String.html}{String} \sphinxbfcode{\sphinxupquote{ANSI\_WHITE\_BACKGROUND}}}
\end{fulllineitems}



\paragraph{ANSI\_WHITE\_BOLD}
\label{\detokenize{source/it/unicam/cs/pa/mastermind/ui/AnsiUtility:ansi-white-bold}}\index{ANSI\_WHITE\_BOLD (Java field)@\spxentry{ANSI\_WHITE\_BOLD}\spxextra{Java field}}

\begin{fulllineitems}
\phantomsection\label{\detokenize{source/it/unicam/cs/pa/mastermind/ui/AnsiUtility:it.unicam.cs.pa.mastermind.ui.AnsiUtility.ANSI_WHITE_BOLD}}\pysigline{public static final \sphinxhref{http://docs.oracle.com/javase/8/docs/api/java/lang/String.html}{String} \sphinxbfcode{\sphinxupquote{ANSI\_WHITE\_BOLD}}}
\end{fulllineitems}



\paragraph{ANSI\_YELLOW}
\label{\detokenize{source/it/unicam/cs/pa/mastermind/ui/AnsiUtility:ansi-yellow}}\index{ANSI\_YELLOW (Java field)@\spxentry{ANSI\_YELLOW}\spxextra{Java field}}

\begin{fulllineitems}
\phantomsection\label{\detokenize{source/it/unicam/cs/pa/mastermind/ui/AnsiUtility:it.unicam.cs.pa.mastermind.ui.AnsiUtility.ANSI_YELLOW}}\pysigline{public static final \sphinxhref{http://docs.oracle.com/javase/8/docs/api/java/lang/String.html}{String} \sphinxbfcode{\sphinxupquote{ANSI\_YELLOW}}}
\end{fulllineitems}



\paragraph{ANSI\_YELLOW\_BACKGROUND}
\label{\detokenize{source/it/unicam/cs/pa/mastermind/ui/AnsiUtility:ansi-yellow-background}}\index{ANSI\_YELLOW\_BACKGROUND (Java field)@\spxentry{ANSI\_YELLOW\_BACKGROUND}\spxextra{Java field}}

\begin{fulllineitems}
\phantomsection\label{\detokenize{source/it/unicam/cs/pa/mastermind/ui/AnsiUtility:it.unicam.cs.pa.mastermind.ui.AnsiUtility.ANSI_YELLOW_BACKGROUND}}\pysigline{public static final \sphinxhref{http://docs.oracle.com/javase/8/docs/api/java/lang/String.html}{String} \sphinxbfcode{\sphinxupquote{ANSI\_YELLOW\_BACKGROUND}}}
\end{fulllineitems}



\subsection{BoardObserver}
\label{\detokenize{source/it/unicam/cs/pa/mastermind/ui/BoardObserver:boardobserver}}\label{\detokenize{source/it/unicam/cs/pa/mastermind/ui/BoardObserver::doc}}\index{BoardObserver (Java class)@\spxentry{BoardObserver}\spxextra{Java class}}

\begin{fulllineitems}
\phantomsection\label{\detokenize{source/it/unicam/cs/pa/mastermind/ui/BoardObserver:it.unicam.cs.pa.mastermind.ui.BoardObserver}}\pysigline{public abstract class \sphinxbfcode{\sphinxupquote{BoardObserver}}}
Classe astratta estendibile da tutte quelle classi coinvolte nel design pattern \sphinxstylestrong{Observer}, aventi quindi necessità di osservare e adattarsi in tempo reale ai cambiamenti di stato di oggetti di tipo BoardModel.
\begin{quote}\begin{description}
\item[{Author}] \leavevmode
Francesco Pio Stelluti, Francesco Coppola

\end{description}\end{quote}

\end{fulllineitems}



\subsubsection{Fields}
\label{\detokenize{source/it/unicam/cs/pa/mastermind/ui/BoardObserver:fields}}

\paragraph{subject}
\label{\detokenize{source/it/unicam/cs/pa/mastermind/ui/BoardObserver:subject}}\index{subject (Java field)@\spxentry{subject}\spxextra{Java field}}

\begin{fulllineitems}
\phantomsection\label{\detokenize{source/it/unicam/cs/pa/mastermind/ui/BoardObserver:it.unicam.cs.pa.mastermind.ui.BoardObserver.subject}}\pysigline{protected {\hyperref[\detokenize{source/it/unicam/cs/pa/mastermind/gamecore/BoardModel:it.unicam.cs.pa.mastermind.gamecore.BoardModel}]{\sphinxcrossref{BoardModel}}} \sphinxbfcode{\sphinxupquote{subject}}}
L’oggetto che viene osservato.

\end{fulllineitems}



\subsubsection{Methods}
\label{\detokenize{source/it/unicam/cs/pa/mastermind/ui/BoardObserver:methods}}

\paragraph{addSubject}
\label{\detokenize{source/it/unicam/cs/pa/mastermind/ui/BoardObserver:addsubject}}\index{addSubject(BoardModel) (Java method)@\spxentry{addSubject(BoardModel)}\spxextra{Java method}}

\begin{fulllineitems}
\phantomsection\label{\detokenize{source/it/unicam/cs/pa/mastermind/ui/BoardObserver:it.unicam.cs.pa.mastermind.ui.BoardObserver.addSubject(BoardModel)}}\pysiglinewithargsret{public void \sphinxbfcode{\sphinxupquote{addSubject}}}{{\hyperref[\detokenize{source/it/unicam/cs/pa/mastermind/gamecore/BoardModel:it.unicam.cs.pa.mastermind.gamecore.BoardModel}]{\sphinxcrossref{BoardModel}}}\sphinxstyleemphasis{ subject}}{}
Metodo per il quale viene aggiunto un altro elemento da osservare alla lista interna.
\begin{quote}\begin{description}
\item[{Parametri}] \leavevmode\begin{itemize}
\item {} 
\sphinxstyleliteralstrong{\sphinxupquote{subject}} \textendash{} il soggetto che si vuole osservare

\end{itemize}

\end{description}\end{quote}

\end{fulllineitems}



\paragraph{update}
\label{\detokenize{source/it/unicam/cs/pa/mastermind/ui/BoardObserver:update}}\index{update() (Java method)@\spxentry{update()}\spxextra{Java method}}

\begin{fulllineitems}
\phantomsection\label{\detokenize{source/it/unicam/cs/pa/mastermind/ui/BoardObserver:it.unicam.cs.pa.mastermind.ui.BoardObserver.update()}}\pysiglinewithargsret{public abstract void \sphinxbfcode{\sphinxupquote{update}}}{}{}
Aggiornamento dello stato interno dell’oggetto.

\end{fulllineitems}



\subsection{ConsoleInteractionView}
\label{\detokenize{source/it/unicam/cs/pa/mastermind/ui/ConsoleInteractionView:consoleinteractionview}}\label{\detokenize{source/it/unicam/cs/pa/mastermind/ui/ConsoleInteractionView::doc}}\index{ConsoleInteractionView (Java class)@\spxentry{ConsoleInteractionView}\spxextra{Java class}}

\begin{fulllineitems}
\phantomsection\label{\detokenize{source/it/unicam/cs/pa/mastermind/ui/ConsoleInteractionView:it.unicam.cs.pa.mastermind.ui.ConsoleInteractionView}}\pysigline{public class \sphinxbfcode{\sphinxupquote{ConsoleInteractionView}} extends {\hyperref[\detokenize{source/it/unicam/cs/pa/mastermind/ui/InteractionView:it.unicam.cs.pa.mastermind.ui.InteractionView}]{\sphinxcrossref{InteractionView}}}}
Implementazione con interazione via console della classe \sphinxcode{\sphinxupquote{InteractionView}}.
\begin{quote}\begin{description}
\item[{Author}] \leavevmode
Francesco Pio Stelluti, Francesco Coppola

\end{description}\end{quote}

\end{fulllineitems}



\subsubsection{Methods}
\label{\detokenize{source/it/unicam/cs/pa/mastermind/ui/ConsoleInteractionView:methods}}

\paragraph{endingScreen}
\label{\detokenize{source/it/unicam/cs/pa/mastermind/ui/ConsoleInteractionView:endingscreen}}\index{endingScreen(String) (Java method)@\spxentry{endingScreen(String)}\spxextra{Java method}}

\begin{fulllineitems}
\phantomsection\label{\detokenize{source/it/unicam/cs/pa/mastermind/ui/ConsoleInteractionView:it.unicam.cs.pa.mastermind.ui.ConsoleInteractionView.endingScreen(String)}}\pysiglinewithargsret{public void \sphinxbfcode{\sphinxupquote{endingScreen}}}{\sphinxhref{http://docs.oracle.com/javase/8/docs/api/java/lang/String.html}{String}\sphinxstyleemphasis{ gameEndingMessage}}{}
\end{fulllineitems}



\paragraph{getIndexSequence}
\label{\detokenize{source/it/unicam/cs/pa/mastermind/ui/ConsoleInteractionView:getindexsequence}}\index{getIndexSequence(boolean) (Java method)@\spxentry{getIndexSequence(boolean)}\spxextra{Java method}}

\begin{fulllineitems}
\phantomsection\label{\detokenize{source/it/unicam/cs/pa/mastermind/ui/ConsoleInteractionView:it.unicam.cs.pa.mastermind.ui.ConsoleInteractionView.getIndexSequence(boolean)}}\pysiglinewithargsret{public \sphinxhref{http://docs.oracle.com/javase/8/docs/api/java/util/List.html}{List}\textless{}\sphinxhref{http://docs.oracle.com/javase/8/docs/api/java/lang/Integer.html}{Integer}\textgreater{} \sphinxbfcode{\sphinxupquote{getIndexSequence}}}{boolean\sphinxstyleemphasis{ isBreaker}}{}
\end{fulllineitems}



\paragraph{getInstance}
\label{\detokenize{source/it/unicam/cs/pa/mastermind/ui/ConsoleInteractionView:getinstance}}\index{getInstance() (Java method)@\spxentry{getInstance()}\spxextra{Java method}}

\begin{fulllineitems}
\phantomsection\label{\detokenize{source/it/unicam/cs/pa/mastermind/ui/ConsoleInteractionView:it.unicam.cs.pa.mastermind.ui.ConsoleInteractionView.getInstance()}}\pysiglinewithargsret{public static {\hyperref[\detokenize{source/it/unicam/cs/pa/mastermind/ui/ConsoleInteractionView:it.unicam.cs.pa.mastermind.ui.ConsoleInteractionView}]{\sphinxcrossref{ConsoleInteractionView}}} \sphinxbfcode{\sphinxupquote{getInstance}}}{}{}~\begin{quote}\begin{description}
\item[{Ritorna}] \leavevmode
ConsoleStartView istanza singleton di \sphinxcode{\sphinxupquote{ConsoleInteractionView}}.

\end{description}\end{quote}

\end{fulllineitems}



\paragraph{init}
\label{\detokenize{source/it/unicam/cs/pa/mastermind/ui/ConsoleInteractionView:init}}\index{init(BufferedReader) (Java method)@\spxentry{init(BufferedReader)}\spxextra{Java method}}

\begin{fulllineitems}
\phantomsection\label{\detokenize{source/it/unicam/cs/pa/mastermind/ui/ConsoleInteractionView:it.unicam.cs.pa.mastermind.ui.ConsoleInteractionView.init(BufferedReader)}}\pysiglinewithargsret{public void \sphinxbfcode{\sphinxupquote{init}}}{\sphinxhref{http://docs.oracle.com/javase/8/docs/api/java/io/BufferedReader.html}{BufferedReader}\sphinxstyleemphasis{ newReader}}{}
Inizializzazione del reader associato all’istanza di \sphinxcode{\sphinxupquote{ConsoleInteractionView}}.
\begin{quote}\begin{description}
\item[{Parametri}] \leavevmode\begin{itemize}
\item {} 
\sphinxstyleliteralstrong{\sphinxupquote{newReader}} \textendash{} reader da associare all’istanza.

\end{itemize}

\end{description}\end{quote}

\end{fulllineitems}



\paragraph{update}
\label{\detokenize{source/it/unicam/cs/pa/mastermind/ui/ConsoleInteractionView:update}}\index{update() (Java method)@\spxentry{update()}\spxextra{Java method}}

\begin{fulllineitems}
\phantomsection\label{\detokenize{source/it/unicam/cs/pa/mastermind/ui/ConsoleInteractionView:it.unicam.cs.pa.mastermind.ui.ConsoleInteractionView.update()}}\pysiglinewithargsret{public void \sphinxbfcode{\sphinxupquote{update}}}{}{}
\end{fulllineitems}



\subsection{ConsoleStartView}
\label{\detokenize{source/it/unicam/cs/pa/mastermind/ui/ConsoleStartView:consolestartview}}\label{\detokenize{source/it/unicam/cs/pa/mastermind/ui/ConsoleStartView::doc}}\index{ConsoleStartView (Java class)@\spxentry{ConsoleStartView}\spxextra{Java class}}

\begin{fulllineitems}
\phantomsection\label{\detokenize{source/it/unicam/cs/pa/mastermind/ui/ConsoleStartView:it.unicam.cs.pa.mastermind.ui.ConsoleStartView}}\pysigline{public class \sphinxbfcode{\sphinxupquote{ConsoleStartView}} extends {\hyperref[\detokenize{source/it/unicam/cs/pa/mastermind/ui/StartView:it.unicam.cs.pa.mastermind.ui.StartView}]{\sphinxcrossref{StartView}}}}
Implementazione con interazione via console della classe \sphinxcode{\sphinxupquote{StartView}}.
\begin{quote}\begin{description}
\item[{Author}] \leavevmode
Francesco Pio Stelluti, Francesco Coppola

\end{description}\end{quote}

\end{fulllineitems}



\subsubsection{Methods}
\label{\detokenize{source/it/unicam/cs/pa/mastermind/ui/ConsoleStartView:methods}}

\paragraph{askNewAttempts}
\label{\detokenize{source/it/unicam/cs/pa/mastermind/ui/ConsoleStartView:asknewattempts}}\index{askNewAttempts() (Java method)@\spxentry{askNewAttempts()}\spxextra{Java method}}

\begin{fulllineitems}
\phantomsection\label{\detokenize{source/it/unicam/cs/pa/mastermind/ui/ConsoleStartView:it.unicam.cs.pa.mastermind.ui.ConsoleStartView.askNewAttempts()}}\pysiglinewithargsret{protected int \sphinxbfcode{\sphinxupquote{askNewAttempts}}}{}{}
\end{fulllineitems}



\paragraph{askNewGameSettings}
\label{\detokenize{source/it/unicam/cs/pa/mastermind/ui/ConsoleStartView:asknewgamesettings}}\index{askNewGameSettings() (Java method)@\spxentry{askNewGameSettings()}\spxextra{Java method}}

\begin{fulllineitems}
\phantomsection\label{\detokenize{source/it/unicam/cs/pa/mastermind/ui/ConsoleStartView:it.unicam.cs.pa.mastermind.ui.ConsoleStartView.askNewGameSettings()}}\pysiglinewithargsret{protected {\hyperref[\detokenize{source/it/unicam/cs/pa/mastermind/gamecore/NewGameStats:it.unicam.cs.pa.mastermind.gamecore.NewGameStats}]{\sphinxcrossref{NewGameStats}}} \sphinxbfcode{\sphinxupquote{askNewGameSettings}}}{}{}
\end{fulllineitems}



\paragraph{askNewLength}
\label{\detokenize{source/it/unicam/cs/pa/mastermind/ui/ConsoleStartView:asknewlength}}\index{askNewLength() (Java method)@\spxentry{askNewLength()}\spxextra{Java method}}

\begin{fulllineitems}
\phantomsection\label{\detokenize{source/it/unicam/cs/pa/mastermind/ui/ConsoleStartView:it.unicam.cs.pa.mastermind.ui.ConsoleStartView.askNewLength()}}\pysiglinewithargsret{protected int \sphinxbfcode{\sphinxupquote{askNewLength}}}{}{}
\end{fulllineitems}



\paragraph{askNewSettings}
\label{\detokenize{source/it/unicam/cs/pa/mastermind/ui/ConsoleStartView:asknewsettings}}\index{askNewSettings() (Java method)@\spxentry{askNewSettings()}\spxextra{Java method}}

\begin{fulllineitems}
\phantomsection\label{\detokenize{source/it/unicam/cs/pa/mastermind/ui/ConsoleStartView:it.unicam.cs.pa.mastermind.ui.ConsoleStartView.askNewSettings()}}\pysiglinewithargsret{protected boolean \sphinxbfcode{\sphinxupquote{askNewSettings}}}{}{}
\end{fulllineitems}



\paragraph{badEnding}
\label{\detokenize{source/it/unicam/cs/pa/mastermind/ui/ConsoleStartView:badending}}\index{badEnding(String) (Java method)@\spxentry{badEnding(String)}\spxextra{Java method}}

\begin{fulllineitems}
\phantomsection\label{\detokenize{source/it/unicam/cs/pa/mastermind/ui/ConsoleStartView:it.unicam.cs.pa.mastermind.ui.ConsoleStartView.badEnding(String)}}\pysiglinewithargsret{protected void \sphinxbfcode{\sphinxupquote{badEnding}}}{\sphinxhref{http://docs.oracle.com/javase/8/docs/api/java/lang/String.html}{String}\sphinxstyleemphasis{ reason}}{}
\end{fulllineitems}



\paragraph{ending}
\label{\detokenize{source/it/unicam/cs/pa/mastermind/ui/ConsoleStartView:ending}}\index{ending() (Java method)@\spxentry{ending()}\spxextra{Java method}}

\begin{fulllineitems}
\phantomsection\label{\detokenize{source/it/unicam/cs/pa/mastermind/ui/ConsoleStartView:it.unicam.cs.pa.mastermind.ui.ConsoleStartView.ending()}}\pysiglinewithargsret{protected void \sphinxbfcode{\sphinxupquote{ending}}}{}{}
\end{fulllineitems}



\paragraph{getInstance}
\label{\detokenize{source/it/unicam/cs/pa/mastermind/ui/ConsoleStartView:getinstance}}\index{getInstance() (Java method)@\spxentry{getInstance()}\spxextra{Java method}}

\begin{fulllineitems}
\phantomsection\label{\detokenize{source/it/unicam/cs/pa/mastermind/ui/ConsoleStartView:it.unicam.cs.pa.mastermind.ui.ConsoleStartView.getInstance()}}\pysiglinewithargsret{public static {\hyperref[\detokenize{source/it/unicam/cs/pa/mastermind/ui/ConsoleStartView:it.unicam.cs.pa.mastermind.ui.ConsoleStartView}]{\sphinxcrossref{ConsoleStartView}}} \sphinxbfcode{\sphinxupquote{getInstance}}}{}{}~\begin{quote}\begin{description}
\item[{Ritorna}] \leavevmode
ConsoleStartView istanza singleton di \sphinxcode{\sphinxupquote{ConsoleStartView}}.

\end{description}\end{quote}

\end{fulllineitems}



\paragraph{getInteractionView}
\label{\detokenize{source/it/unicam/cs/pa/mastermind/ui/ConsoleStartView:getinteractionview}}\index{getInteractionView() (Java method)@\spxentry{getInteractionView()}\spxextra{Java method}}

\begin{fulllineitems}
\phantomsection\label{\detokenize{source/it/unicam/cs/pa/mastermind/ui/ConsoleStartView:it.unicam.cs.pa.mastermind.ui.ConsoleStartView.getInteractionView()}}\pysiglinewithargsret{protected {\hyperref[\detokenize{source/it/unicam/cs/pa/mastermind/ui/InteractionView:it.unicam.cs.pa.mastermind.ui.InteractionView}]{\sphinxcrossref{InteractionView}}} \sphinxbfcode{\sphinxupquote{getInteractionView}}}{}{}
\end{fulllineitems}



\paragraph{getPlayerName}
\label{\detokenize{source/it/unicam/cs/pa/mastermind/ui/ConsoleStartView:getplayername}}\index{getPlayerName(PlayerFactoryRegistry, boolean) (Java method)@\spxentry{getPlayerName(PlayerFactoryRegistry, boolean)}\spxextra{Java method}}

\begin{fulllineitems}
\phantomsection\label{\detokenize{source/it/unicam/cs/pa/mastermind/ui/ConsoleStartView:it.unicam.cs.pa.mastermind.ui.ConsoleStartView.getPlayerName(PlayerFactoryRegistry, boolean)}}\pysiglinewithargsret{protected \sphinxhref{http://docs.oracle.com/javase/8/docs/api/java/lang/String.html}{String} \sphinxbfcode{\sphinxupquote{getPlayerName}}}{{\hyperref[\detokenize{source/it/unicam/cs/pa/mastermind/players/PlayerFactoryRegistry:it.unicam.cs.pa.mastermind.players.PlayerFactoryRegistry}]{\sphinxcrossref{PlayerFactoryRegistry}}}\sphinxstyleemphasis{ registry}, boolean\sphinxstyleemphasis{ isBreaker}}{}
\end{fulllineitems}



\paragraph{main}
\label{\detokenize{source/it/unicam/cs/pa/mastermind/ui/ConsoleStartView:main}}\index{main(String{[}{]}) (Java method)@\spxentry{main(String{[}{]})}\spxextra{Java method}}

\begin{fulllineitems}
\phantomsection\label{\detokenize{source/it/unicam/cs/pa/mastermind/ui/ConsoleStartView:it.unicam.cs.pa.mastermind.ui.ConsoleStartView.main(String__)}}\pysiglinewithargsret{public static void \sphinxbfcode{\sphinxupquote{main}}}{\sphinxhref{http://docs.oracle.com/javase/8/docs/api/java/lang/String.html}{String}{[}{]}\sphinxstyleemphasis{ args}}{}
\end{fulllineitems}



\paragraph{showLogo}
\label{\detokenize{source/it/unicam/cs/pa/mastermind/ui/ConsoleStartView:showlogo}}\index{showLogo() (Java method)@\spxentry{showLogo()}\spxextra{Java method}}

\begin{fulllineitems}
\phantomsection\label{\detokenize{source/it/unicam/cs/pa/mastermind/ui/ConsoleStartView:it.unicam.cs.pa.mastermind.ui.ConsoleStartView.showLogo()}}\pysiglinewithargsret{protected void \sphinxbfcode{\sphinxupquote{showLogo}}}{}{}
\end{fulllineitems}



\paragraph{showNewGameStarting}
\label{\detokenize{source/it/unicam/cs/pa/mastermind/ui/ConsoleStartView:shownewgamestarting}}\index{showNewGameStarting() (Java method)@\spxentry{showNewGameStarting()}\spxextra{Java method}}

\begin{fulllineitems}
\phantomsection\label{\detokenize{source/it/unicam/cs/pa/mastermind/ui/ConsoleStartView:it.unicam.cs.pa.mastermind.ui.ConsoleStartView.showNewGameStarting()}}\pysiglinewithargsret{protected void \sphinxbfcode{\sphinxupquote{showNewGameStarting}}}{}{}
\end{fulllineitems}



\subsection{InteractionView}
\label{\detokenize{source/it/unicam/cs/pa/mastermind/ui/InteractionView:interactionview}}\label{\detokenize{source/it/unicam/cs/pa/mastermind/ui/InteractionView::doc}}\index{InteractionView (Java class)@\spxentry{InteractionView}\spxextra{Java class}}

\begin{fulllineitems}
\phantomsection\label{\detokenize{source/it/unicam/cs/pa/mastermind/ui/InteractionView:it.unicam.cs.pa.mastermind.ui.InteractionView}}\pysigline{public abstract class \sphinxbfcode{\sphinxupquote{InteractionView}} extends {\hyperref[\detokenize{source/it/unicam/cs/pa/mastermind/ui/BoardObserver:it.unicam.cs.pa.mastermind.ui.BoardObserver}]{\sphinxcrossref{BoardObserver}}}}
\sphinxstylestrong{Responsabilità}: fornire ai giocatori coinvolti in una singola partita interazioni con quest’ultima.
\begin{quote}\begin{description}
\item[{Author}] \leavevmode
Francesco Pio Stelluti, Francesco Coppola

\end{description}\end{quote}

\end{fulllineitems}



\subsubsection{Fields}
\label{\detokenize{source/it/unicam/cs/pa/mastermind/ui/InteractionView:fields}}

\paragraph{currentSequenceLength}
\label{\detokenize{source/it/unicam/cs/pa/mastermind/ui/InteractionView:currentsequencelength}}\index{currentSequenceLength (Java field)@\spxentry{currentSequenceLength}\spxextra{Java field}}

\begin{fulllineitems}
\phantomsection\label{\detokenize{source/it/unicam/cs/pa/mastermind/ui/InteractionView:it.unicam.cs.pa.mastermind.ui.InteractionView.currentSequenceLength}}\pysigline{protected int \sphinxbfcode{\sphinxupquote{currentSequenceLength}}}
La lunghezza della sequenza da indovinare.

\end{fulllineitems}



\paragraph{currentSequenceToGuess}
\label{\detokenize{source/it/unicam/cs/pa/mastermind/ui/InteractionView:currentsequencetoguess}}\index{currentSequenceToGuess (Java field)@\spxentry{currentSequenceToGuess}\spxextra{Java field}}

\begin{fulllineitems}
\phantomsection\label{\detokenize{source/it/unicam/cs/pa/mastermind/ui/InteractionView:it.unicam.cs.pa.mastermind.ui.InteractionView.currentSequenceToGuess}}\pysigline{protected \sphinxhref{http://docs.oracle.com/javase/8/docs/api/java/util/List.html}{List}\textless{}{\hyperref[\detokenize{source/it/unicam/cs/pa/mastermind/gamecore/ColorPegs:it.unicam.cs.pa.mastermind.gamecore.ColorPegs}]{\sphinxcrossref{ColorPegs}}}\textgreater{} \sphinxbfcode{\sphinxupquote{currentSequenceToGuess}}}
La sequenza da indovinare.

\end{fulllineitems}



\paragraph{lastAttemptAndClue}
\label{\detokenize{source/it/unicam/cs/pa/mastermind/ui/InteractionView:lastattemptandclue}}\index{lastAttemptAndClue (Java field)@\spxentry{lastAttemptAndClue}\spxextra{Java field}}

\begin{fulllineitems}
\phantomsection\label{\detokenize{source/it/unicam/cs/pa/mastermind/ui/InteractionView:it.unicam.cs.pa.mastermind.ui.InteractionView.lastAttemptAndClue}}\pysigline{protected \sphinxhref{http://docs.oracle.com/javase/8/docs/api/java/util/Map.html}{Map}.Entry\textless{}\sphinxhref{http://docs.oracle.com/javase/8/docs/api/java/util/List.html}{List}\textless{}{\hyperref[\detokenize{source/it/unicam/cs/pa/mastermind/gamecore/ColorPegs:it.unicam.cs.pa.mastermind.gamecore.ColorPegs}]{\sphinxcrossref{ColorPegs}}}\textgreater{}, \sphinxhref{http://docs.oracle.com/javase/8/docs/api/java/util/List.html}{List}\textless{}{\hyperref[\detokenize{source/it/unicam/cs/pa/mastermind/gamecore/ColorPegs:it.unicam.cs.pa.mastermind.gamecore.ColorPegs}]{\sphinxcrossref{ColorPegs}}}\textgreater{}\textgreater{} \sphinxbfcode{\sphinxupquote{lastAttemptAndClue}}}
Singola entry di una mappa, contenente l’ultima lista di ColorPegs inseriti e la relativa sequenza indizio.

\end{fulllineitems}



\subsubsection{Methods}
\label{\detokenize{source/it/unicam/cs/pa/mastermind/ui/InteractionView:methods}}

\paragraph{endingScreen}
\label{\detokenize{source/it/unicam/cs/pa/mastermind/ui/InteractionView:endingscreen}}\index{endingScreen(String) (Java method)@\spxentry{endingScreen(String)}\spxextra{Java method}}

\begin{fulllineitems}
\phantomsection\label{\detokenize{source/it/unicam/cs/pa/mastermind/ui/InteractionView:it.unicam.cs.pa.mastermind.ui.InteractionView.endingScreen(String)}}\pysiglinewithargsret{public abstract void \sphinxbfcode{\sphinxupquote{endingScreen}}}{\sphinxhref{http://docs.oracle.com/javase/8/docs/api/java/lang/String.html}{String}\sphinxstyleemphasis{ gameEndingMessage}}{}
Interazione finale con il giocatore relativa al termine di una partita
\begin{quote}\begin{description}
\item[{Parametri}] \leavevmode\begin{itemize}
\item {} 
\sphinxstyleliteralstrong{\sphinxupquote{gameEndingMessage}} \textendash{} stringa con il messaggio finale da mostrare al giocatore

\end{itemize}

\end{description}\end{quote}

\end{fulllineitems}



\paragraph{getCurrentSequenceLength}
\label{\detokenize{source/it/unicam/cs/pa/mastermind/ui/InteractionView:getcurrentsequencelength}}\index{getCurrentSequenceLength() (Java method)@\spxentry{getCurrentSequenceLength()}\spxextra{Java method}}

\begin{fulllineitems}
\phantomsection\label{\detokenize{source/it/unicam/cs/pa/mastermind/ui/InteractionView:it.unicam.cs.pa.mastermind.ui.InteractionView.getCurrentSequenceLength()}}\pysiglinewithargsret{public int \sphinxbfcode{\sphinxupquote{getCurrentSequenceLength}}}{}{}
Metodo getter che restituisce la lunghezza della sequenza da indovinare.
\begin{quote}\begin{description}
\item[{Ritorna}] \leavevmode
int il valore intero di tale lunghezza

\end{description}\end{quote}

\end{fulllineitems}



\paragraph{getCurrentSequenceToGuess}
\label{\detokenize{source/it/unicam/cs/pa/mastermind/ui/InteractionView:getcurrentsequencetoguess}}\index{getCurrentSequenceToGuess() (Java method)@\spxentry{getCurrentSequenceToGuess()}\spxextra{Java method}}

\begin{fulllineitems}
\phantomsection\label{\detokenize{source/it/unicam/cs/pa/mastermind/ui/InteractionView:it.unicam.cs.pa.mastermind.ui.InteractionView.getCurrentSequenceToGuess()}}\pysiglinewithargsret{public \sphinxhref{http://docs.oracle.com/javase/8/docs/api/java/util/List.html}{List}\textless{}{\hyperref[\detokenize{source/it/unicam/cs/pa/mastermind/gamecore/ColorPegs:it.unicam.cs.pa.mastermind.gamecore.ColorPegs}]{\sphinxcrossref{ColorPegs}}}\textgreater{} \sphinxbfcode{\sphinxupquote{getCurrentSequenceToGuess}}}{}{}
Metodo getter che restituisce la sequenza da indovinare.
\begin{quote}\begin{description}
\item[{Ritorna}] \leavevmode
List la lista di ColorPegs da indovinare

\end{description}\end{quote}

\end{fulllineitems}



\paragraph{getIndexSequence}
\label{\detokenize{source/it/unicam/cs/pa/mastermind/ui/InteractionView:getindexsequence}}\index{getIndexSequence(boolean) (Java method)@\spxentry{getIndexSequence(boolean)}\spxextra{Java method}}

\begin{fulllineitems}
\phantomsection\label{\detokenize{source/it/unicam/cs/pa/mastermind/ui/InteractionView:it.unicam.cs.pa.mastermind.ui.InteractionView.getIndexSequence(boolean)}}\pysiglinewithargsret{public abstract \sphinxhref{http://docs.oracle.com/javase/8/docs/api/java/util/List.html}{List}\textless{}\sphinxhref{http://docs.oracle.com/javase/8/docs/api/java/lang/Integer.html}{Integer}\textgreater{} \sphinxbfcode{\sphinxupquote{getIndexSequence}}}{boolean\sphinxstyleemphasis{ toGuess}}{}
Interazione con l’utente fisico o altra entità per poter ottenere gli indici associati ai diversi valori di \sphinxcode{\sphinxupquote{ColorPegs}}. Se il valore restituito contiene l”\sphinxcode{\sphinxupquote{Integer}} 0 è stata rappresentata la volontà di un giocatore \sphinxcode{\sphinxupquote{CodeBreaker}} di arrendersi.
\begin{quote}\begin{description}
\item[{Parametri}] \leavevmode\begin{itemize}
\item {} 
\sphinxstyleliteralstrong{\sphinxupquote{toGuess}} \textendash{} flag che indica se la sequenza di interi da ottenere si riferisce alla sequenza da indovinare o meno

\end{itemize}

\item[{Ritorna}] \leavevmode
List contenente gli indici da 1 a currentSequenceLength, associati all’enum ColorPegs

\end{description}\end{quote}

\end{fulllineitems}



\paragraph{getLastAttemptAndClue}
\label{\detokenize{source/it/unicam/cs/pa/mastermind/ui/InteractionView:getlastattemptandclue}}\index{getLastAttemptAndClue() (Java method)@\spxentry{getLastAttemptAndClue()}\spxextra{Java method}}

\begin{fulllineitems}
\phantomsection\label{\detokenize{source/it/unicam/cs/pa/mastermind/ui/InteractionView:it.unicam.cs.pa.mastermind.ui.InteractionView.getLastAttemptAndClue()}}\pysiglinewithargsret{public \sphinxhref{http://docs.oracle.com/javase/8/docs/api/java/util/Map.html}{Map}.Entry\textless{}\sphinxhref{http://docs.oracle.com/javase/8/docs/api/java/util/List.html}{List}\textless{}{\hyperref[\detokenize{source/it/unicam/cs/pa/mastermind/gamecore/ColorPegs:it.unicam.cs.pa.mastermind.gamecore.ColorPegs}]{\sphinxcrossref{ColorPegs}}}\textgreater{}, \sphinxhref{http://docs.oracle.com/javase/8/docs/api/java/util/List.html}{List}\textless{}{\hyperref[\detokenize{source/it/unicam/cs/pa/mastermind/gamecore/ColorPegs:it.unicam.cs.pa.mastermind.gamecore.ColorPegs}]{\sphinxcrossref{ColorPegs}}}\textgreater{}\textgreater{} \sphinxbfcode{\sphinxupquote{getLastAttemptAndClue}}}{}{}
Metodo getter che restituisce la entry di mappa contenente l’ultima lista di ColorPegs inseriti e la relativa sequenza indizio.
\begin{quote}\begin{description}
\item[{Ritorna}] \leavevmode
Map.Entry contenente l’ultima lista di ColorPegs inseriti e la relativa sequenza indizio.

\end{description}\end{quote}

\end{fulllineitems}



\subsection{StartStats}
\label{\detokenize{source/it/unicam/cs/pa/mastermind/ui/StartStats:startstats}}\label{\detokenize{source/it/unicam/cs/pa/mastermind/ui/StartStats::doc}}\index{StartStats (Java class)@\spxentry{StartStats}\spxextra{Java class}}

\begin{fulllineitems}
\phantomsection\label{\detokenize{source/it/unicam/cs/pa/mastermind/ui/StartStats:it.unicam.cs.pa.mastermind.ui.StartStats}}\pysigline{public class \sphinxbfcode{\sphinxupquote{StartStats}}}
\sphinxstylestrong{Responsabilità}: tenere traccia delle informazioni necessarie per poter iniziare una nuova partita.
\begin{quote}\begin{description}
\item[{Author}] \leavevmode
Francesco Pio Stelluti, Francesco Coppola

\end{description}\end{quote}

\end{fulllineitems}



\subsubsection{Fields}
\label{\detokenize{source/it/unicam/cs/pa/mastermind/ui/StartStats:fields}}

\paragraph{highTresholdLength}
\label{\detokenize{source/it/unicam/cs/pa/mastermind/ui/StartStats:hightresholdlength}}\index{highTresholdLength (Java field)@\spxentry{highTresholdLength}\spxextra{Java field}}

\begin{fulllineitems}
\phantomsection\label{\detokenize{source/it/unicam/cs/pa/mastermind/ui/StartStats:it.unicam.cs.pa.mastermind.ui.StartStats.highTresholdLength}}\pysigline{ int \sphinxbfcode{\sphinxupquote{highTresholdLength}}}
\end{fulllineitems}



\paragraph{lowTresholdAttempts}
\label{\detokenize{source/it/unicam/cs/pa/mastermind/ui/StartStats:lowtresholdattempts}}\index{lowTresholdAttempts (Java field)@\spxentry{lowTresholdAttempts}\spxextra{Java field}}

\begin{fulllineitems}
\phantomsection\label{\detokenize{source/it/unicam/cs/pa/mastermind/ui/StartStats:it.unicam.cs.pa.mastermind.ui.StartStats.lowTresholdAttempts}}\pysigline{ int \sphinxbfcode{\sphinxupquote{lowTresholdAttempts}}}
\end{fulllineitems}



\paragraph{lowTresholdLength}
\label{\detokenize{source/it/unicam/cs/pa/mastermind/ui/StartStats:lowtresholdlength}}\index{lowTresholdLength (Java field)@\spxentry{lowTresholdLength}\spxextra{Java field}}

\begin{fulllineitems}
\phantomsection\label{\detokenize{source/it/unicam/cs/pa/mastermind/ui/StartStats:it.unicam.cs.pa.mastermind.ui.StartStats.lowTresholdLength}}\pysigline{ int \sphinxbfcode{\sphinxupquote{lowTresholdLength}}}
\end{fulllineitems}



\subsubsection{Constructors}
\label{\detokenize{source/it/unicam/cs/pa/mastermind/ui/StartStats:constructors}}

\paragraph{StartStats}
\label{\detokenize{source/it/unicam/cs/pa/mastermind/ui/StartStats:id1}}\index{StartStats() (Java constructor)@\spxentry{StartStats()}\spxextra{Java constructor}}

\begin{fulllineitems}
\phantomsection\label{\detokenize{source/it/unicam/cs/pa/mastermind/ui/StartStats:it.unicam.cs.pa.mastermind.ui.StartStats.StartStats()}}\pysiglinewithargsret{public \sphinxbfcode{\sphinxupquote{StartStats}}}{}{}
\end{fulllineitems}



\subsubsection{Methods}
\label{\detokenize{source/it/unicam/cs/pa/mastermind/ui/StartStats:methods}}

\paragraph{getAttempts}
\label{\detokenize{source/it/unicam/cs/pa/mastermind/ui/StartStats:getattempts}}\index{getAttempts() (Java method)@\spxentry{getAttempts()}\spxextra{Java method}}

\begin{fulllineitems}
\phantomsection\label{\detokenize{source/it/unicam/cs/pa/mastermind/ui/StartStats:it.unicam.cs.pa.mastermind.ui.StartStats.getAttempts()}}\pysiglinewithargsret{public int \sphinxbfcode{\sphinxupquote{getAttempts}}}{}{}
\end{fulllineitems}



\paragraph{getBreakers}
\label{\detokenize{source/it/unicam/cs/pa/mastermind/ui/StartStats:getbreakers}}\index{getBreakers() (Java method)@\spxentry{getBreakers()}\spxextra{Java method}}

\begin{fulllineitems}
\phantomsection\label{\detokenize{source/it/unicam/cs/pa/mastermind/ui/StartStats:it.unicam.cs.pa.mastermind.ui.StartStats.getBreakers()}}\pysiglinewithargsret{public {\hyperref[\detokenize{source/it/unicam/cs/pa/mastermind/players/BreakerFactoryRegistry:it.unicam.cs.pa.mastermind.players.BreakerFactoryRegistry}]{\sphinxcrossref{BreakerFactoryRegistry}}} \sphinxbfcode{\sphinxupquote{getBreakers}}}{}{}
\end{fulllineitems}



\paragraph{getCurrentBreaker}
\label{\detokenize{source/it/unicam/cs/pa/mastermind/ui/StartStats:getcurrentbreaker}}\index{getCurrentBreaker() (Java method)@\spxentry{getCurrentBreaker()}\spxextra{Java method}}

\begin{fulllineitems}
\phantomsection\label{\detokenize{source/it/unicam/cs/pa/mastermind/ui/StartStats:it.unicam.cs.pa.mastermind.ui.StartStats.getCurrentBreaker()}}\pysiglinewithargsret{public {\hyperref[\detokenize{source/it/unicam/cs/pa/mastermind/players/CodeBreaker:it.unicam.cs.pa.mastermind.players.CodeBreaker}]{\sphinxcrossref{CodeBreaker}}} \sphinxbfcode{\sphinxupquote{getCurrentBreaker}}}{}{}
\end{fulllineitems}



\paragraph{getCurrentGame}
\label{\detokenize{source/it/unicam/cs/pa/mastermind/ui/StartStats:getcurrentgame}}\index{getCurrentGame() (Java method)@\spxentry{getCurrentGame()}\spxextra{Java method}}

\begin{fulllineitems}
\phantomsection\label{\detokenize{source/it/unicam/cs/pa/mastermind/ui/StartStats:it.unicam.cs.pa.mastermind.ui.StartStats.getCurrentGame()}}\pysiglinewithargsret{public {\hyperref[\detokenize{source/it/unicam/cs/pa/mastermind/gamecore/SingleMatch:it.unicam.cs.pa.mastermind.gamecore.SingleMatch}]{\sphinxcrossref{SingleMatch}}} \sphinxbfcode{\sphinxupquote{getCurrentGame}}}{}{}
\end{fulllineitems}



\paragraph{getCurrentMaker}
\label{\detokenize{source/it/unicam/cs/pa/mastermind/ui/StartStats:getcurrentmaker}}\index{getCurrentMaker() (Java method)@\spxentry{getCurrentMaker()}\spxextra{Java method}}

\begin{fulllineitems}
\phantomsection\label{\detokenize{source/it/unicam/cs/pa/mastermind/ui/StartStats:it.unicam.cs.pa.mastermind.ui.StartStats.getCurrentMaker()}}\pysiglinewithargsret{public {\hyperref[\detokenize{source/it/unicam/cs/pa/mastermind/players/CodeMaker:it.unicam.cs.pa.mastermind.players.CodeMaker}]{\sphinxcrossref{CodeMaker}}} \sphinxbfcode{\sphinxupquote{getCurrentMaker}}}{}{}
\end{fulllineitems}



\paragraph{getHighTresholdLength}
\label{\detokenize{source/it/unicam/cs/pa/mastermind/ui/StartStats:gethightresholdlength}}\index{getHighTresholdLength() (Java method)@\spxentry{getHighTresholdLength()}\spxextra{Java method}}

\begin{fulllineitems}
\phantomsection\label{\detokenize{source/it/unicam/cs/pa/mastermind/ui/StartStats:it.unicam.cs.pa.mastermind.ui.StartStats.getHighTresholdLength()}}\pysiglinewithargsret{public int \sphinxbfcode{\sphinxupquote{getHighTresholdLength}}}{}{}
\end{fulllineitems}



\paragraph{getIntView}
\label{\detokenize{source/it/unicam/cs/pa/mastermind/ui/StartStats:getintview}}\index{getIntView() (Java method)@\spxentry{getIntView()}\spxextra{Java method}}

\begin{fulllineitems}
\phantomsection\label{\detokenize{source/it/unicam/cs/pa/mastermind/ui/StartStats:it.unicam.cs.pa.mastermind.ui.StartStats.getIntView()}}\pysiglinewithargsret{public {\hyperref[\detokenize{source/it/unicam/cs/pa/mastermind/ui/InteractionView:it.unicam.cs.pa.mastermind.ui.InteractionView}]{\sphinxcrossref{InteractionView}}} \sphinxbfcode{\sphinxupquote{getIntView}}}{}{}
\end{fulllineitems}



\paragraph{getLowTresholdAttempts}
\label{\detokenize{source/it/unicam/cs/pa/mastermind/ui/StartStats:getlowtresholdattempts}}\index{getLowTresholdAttempts() (Java method)@\spxentry{getLowTresholdAttempts()}\spxextra{Java method}}

\begin{fulllineitems}
\phantomsection\label{\detokenize{source/it/unicam/cs/pa/mastermind/ui/StartStats:it.unicam.cs.pa.mastermind.ui.StartStats.getLowTresholdAttempts()}}\pysiglinewithargsret{public int \sphinxbfcode{\sphinxupquote{getLowTresholdAttempts}}}{}{}
\end{fulllineitems}



\paragraph{getLowTresholdLength}
\label{\detokenize{source/it/unicam/cs/pa/mastermind/ui/StartStats:getlowtresholdlength}}\index{getLowTresholdLength() (Java method)@\spxentry{getLowTresholdLength()}\spxextra{Java method}}

\begin{fulllineitems}
\phantomsection\label{\detokenize{source/it/unicam/cs/pa/mastermind/ui/StartStats:it.unicam.cs.pa.mastermind.ui.StartStats.getLowTresholdLength()}}\pysiglinewithargsret{public int \sphinxbfcode{\sphinxupquote{getLowTresholdLength}}}{}{}
\end{fulllineitems}



\paragraph{getMakers}
\label{\detokenize{source/it/unicam/cs/pa/mastermind/ui/StartStats:getmakers}}\index{getMakers() (Java method)@\spxentry{getMakers()}\spxextra{Java method}}

\begin{fulllineitems}
\phantomsection\label{\detokenize{source/it/unicam/cs/pa/mastermind/ui/StartStats:it.unicam.cs.pa.mastermind.ui.StartStats.getMakers()}}\pysiglinewithargsret{public {\hyperref[\detokenize{source/it/unicam/cs/pa/mastermind/players/MakerFactoryRegistry:it.unicam.cs.pa.mastermind.players.MakerFactoryRegistry}]{\sphinxcrossref{MakerFactoryRegistry}}} \sphinxbfcode{\sphinxupquote{getMakers}}}{}{}
\end{fulllineitems}



\paragraph{getNewGame}
\label{\detokenize{source/it/unicam/cs/pa/mastermind/ui/StartStats:getnewgame}}\index{getNewGame() (Java method)@\spxentry{getNewGame()}\spxextra{Java method}}

\begin{fulllineitems}
\phantomsection\label{\detokenize{source/it/unicam/cs/pa/mastermind/ui/StartStats:it.unicam.cs.pa.mastermind.ui.StartStats.getNewGame()}}\pysiglinewithargsret{public {\hyperref[\detokenize{source/it/unicam/cs/pa/mastermind/gamecore/NewGameStats:it.unicam.cs.pa.mastermind.gamecore.NewGameStats}]{\sphinxcrossref{NewGameStats}}} \sphinxbfcode{\sphinxupquote{getNewGame}}}{}{}
\end{fulllineitems}



\paragraph{getSequenceLength}
\label{\detokenize{source/it/unicam/cs/pa/mastermind/ui/StartStats:getsequencelength}}\index{getSequenceLength() (Java method)@\spxentry{getSequenceLength()}\spxextra{Java method}}

\begin{fulllineitems}
\phantomsection\label{\detokenize{source/it/unicam/cs/pa/mastermind/ui/StartStats:it.unicam.cs.pa.mastermind.ui.StartStats.getSequenceLength()}}\pysiglinewithargsret{public int \sphinxbfcode{\sphinxupquote{getSequenceLength}}}{}{}
\end{fulllineitems}



\paragraph{isKeepSettings}
\label{\detokenize{source/it/unicam/cs/pa/mastermind/ui/StartStats:iskeepsettings}}\index{isKeepSettings() (Java method)@\spxentry{isKeepSettings()}\spxextra{Java method}}

\begin{fulllineitems}
\phantomsection\label{\detokenize{source/it/unicam/cs/pa/mastermind/ui/StartStats:it.unicam.cs.pa.mastermind.ui.StartStats.isKeepSettings()}}\pysiglinewithargsret{public boolean \sphinxbfcode{\sphinxupquote{isKeepSettings}}}{}{}
\end{fulllineitems}



\paragraph{isToContinue}
\label{\detokenize{source/it/unicam/cs/pa/mastermind/ui/StartStats:istocontinue}}\index{isToContinue() (Java method)@\spxentry{isToContinue()}\spxextra{Java method}}

\begin{fulllineitems}
\phantomsection\label{\detokenize{source/it/unicam/cs/pa/mastermind/ui/StartStats:it.unicam.cs.pa.mastermind.ui.StartStats.isToContinue()}}\pysiglinewithargsret{public boolean \sphinxbfcode{\sphinxupquote{isToContinue}}}{}{}
\end{fulllineitems}



\paragraph{resetLengthAttempts}
\label{\detokenize{source/it/unicam/cs/pa/mastermind/ui/StartStats:resetlengthattempts}}\index{resetLengthAttempts() (Java method)@\spxentry{resetLengthAttempts()}\spxextra{Java method}}

\begin{fulllineitems}
\phantomsection\label{\detokenize{source/it/unicam/cs/pa/mastermind/ui/StartStats:it.unicam.cs.pa.mastermind.ui.StartStats.resetLengthAttempts()}}\pysiglinewithargsret{public void \sphinxbfcode{\sphinxupquote{resetLengthAttempts}}}{}{}
Vengono impostati i valori standard del numero di tentativi e della lunghezza delle sequenze

\end{fulllineitems}



\paragraph{setAttempts}
\label{\detokenize{source/it/unicam/cs/pa/mastermind/ui/StartStats:setattempts}}\index{setAttempts(int) (Java method)@\spxentry{setAttempts(int)}\spxextra{Java method}}

\begin{fulllineitems}
\phantomsection\label{\detokenize{source/it/unicam/cs/pa/mastermind/ui/StartStats:it.unicam.cs.pa.mastermind.ui.StartStats.setAttempts(int)}}\pysiglinewithargsret{public void \sphinxbfcode{\sphinxupquote{setAttempts}}}{int\sphinxstyleemphasis{ attempts}}{}
\end{fulllineitems}



\paragraph{setBreakers}
\label{\detokenize{source/it/unicam/cs/pa/mastermind/ui/StartStats:setbreakers}}\index{setBreakers(BreakerFactoryRegistry) (Java method)@\spxentry{setBreakers(BreakerFactoryRegistry)}\spxextra{Java method}}

\begin{fulllineitems}
\phantomsection\label{\detokenize{source/it/unicam/cs/pa/mastermind/ui/StartStats:it.unicam.cs.pa.mastermind.ui.StartStats.setBreakers(BreakerFactoryRegistry)}}\pysiglinewithargsret{public void \sphinxbfcode{\sphinxupquote{setBreakers}}}{{\hyperref[\detokenize{source/it/unicam/cs/pa/mastermind/players/BreakerFactoryRegistry:it.unicam.cs.pa.mastermind.players.BreakerFactoryRegistry}]{\sphinxcrossref{BreakerFactoryRegistry}}}\sphinxstyleemphasis{ breakers}}{}
\end{fulllineitems}



\paragraph{setCurrentBreaker}
\label{\detokenize{source/it/unicam/cs/pa/mastermind/ui/StartStats:setcurrentbreaker}}\index{setCurrentBreaker(CodeBreaker) (Java method)@\spxentry{setCurrentBreaker(CodeBreaker)}\spxextra{Java method}}

\begin{fulllineitems}
\phantomsection\label{\detokenize{source/it/unicam/cs/pa/mastermind/ui/StartStats:it.unicam.cs.pa.mastermind.ui.StartStats.setCurrentBreaker(CodeBreaker)}}\pysiglinewithargsret{public void \sphinxbfcode{\sphinxupquote{setCurrentBreaker}}}{{\hyperref[\detokenize{source/it/unicam/cs/pa/mastermind/players/CodeBreaker:it.unicam.cs.pa.mastermind.players.CodeBreaker}]{\sphinxcrossref{CodeBreaker}}}\sphinxstyleemphasis{ currentBreaker}}{}
\end{fulllineitems}



\paragraph{setCurrentGame}
\label{\detokenize{source/it/unicam/cs/pa/mastermind/ui/StartStats:setcurrentgame}}\index{setCurrentGame(SingleMatch) (Java method)@\spxentry{setCurrentGame(SingleMatch)}\spxextra{Java method}}

\begin{fulllineitems}
\phantomsection\label{\detokenize{source/it/unicam/cs/pa/mastermind/ui/StartStats:it.unicam.cs.pa.mastermind.ui.StartStats.setCurrentGame(SingleMatch)}}\pysiglinewithargsret{public void \sphinxbfcode{\sphinxupquote{setCurrentGame}}}{{\hyperref[\detokenize{source/it/unicam/cs/pa/mastermind/gamecore/SingleMatch:it.unicam.cs.pa.mastermind.gamecore.SingleMatch}]{\sphinxcrossref{SingleMatch}}}\sphinxstyleemphasis{ currentGame}}{}
\end{fulllineitems}



\paragraph{setCurrentMaker}
\label{\detokenize{source/it/unicam/cs/pa/mastermind/ui/StartStats:setcurrentmaker}}\index{setCurrentMaker(CodeMaker) (Java method)@\spxentry{setCurrentMaker(CodeMaker)}\spxextra{Java method}}

\begin{fulllineitems}
\phantomsection\label{\detokenize{source/it/unicam/cs/pa/mastermind/ui/StartStats:it.unicam.cs.pa.mastermind.ui.StartStats.setCurrentMaker(CodeMaker)}}\pysiglinewithargsret{public void \sphinxbfcode{\sphinxupquote{setCurrentMaker}}}{{\hyperref[\detokenize{source/it/unicam/cs/pa/mastermind/players/CodeMaker:it.unicam.cs.pa.mastermind.players.CodeMaker}]{\sphinxcrossref{CodeMaker}}}\sphinxstyleemphasis{ currentMaker}}{}
\end{fulllineitems}



\paragraph{setHighTresholdLength}
\label{\detokenize{source/it/unicam/cs/pa/mastermind/ui/StartStats:sethightresholdlength}}\index{setHighTresholdLength(int) (Java method)@\spxentry{setHighTresholdLength(int)}\spxextra{Java method}}

\begin{fulllineitems}
\phantomsection\label{\detokenize{source/it/unicam/cs/pa/mastermind/ui/StartStats:it.unicam.cs.pa.mastermind.ui.StartStats.setHighTresholdLength(int)}}\pysiglinewithargsret{public void \sphinxbfcode{\sphinxupquote{setHighTresholdLength}}}{int\sphinxstyleemphasis{ highTresholdLength}}{}
\end{fulllineitems}



\paragraph{setIntView}
\label{\detokenize{source/it/unicam/cs/pa/mastermind/ui/StartStats:setintview}}\index{setIntView(InteractionView) (Java method)@\spxentry{setIntView(InteractionView)}\spxextra{Java method}}

\begin{fulllineitems}
\phantomsection\label{\detokenize{source/it/unicam/cs/pa/mastermind/ui/StartStats:it.unicam.cs.pa.mastermind.ui.StartStats.setIntView(InteractionView)}}\pysiglinewithargsret{public void \sphinxbfcode{\sphinxupquote{setIntView}}}{{\hyperref[\detokenize{source/it/unicam/cs/pa/mastermind/ui/InteractionView:it.unicam.cs.pa.mastermind.ui.InteractionView}]{\sphinxcrossref{InteractionView}}}\sphinxstyleemphasis{ intView}}{}
\end{fulllineitems}



\paragraph{setKeepSettings}
\label{\detokenize{source/it/unicam/cs/pa/mastermind/ui/StartStats:setkeepsettings}}\index{setKeepSettings(boolean) (Java method)@\spxentry{setKeepSettings(boolean)}\spxextra{Java method}}

\begin{fulllineitems}
\phantomsection\label{\detokenize{source/it/unicam/cs/pa/mastermind/ui/StartStats:it.unicam.cs.pa.mastermind.ui.StartStats.setKeepSettings(boolean)}}\pysiglinewithargsret{public void \sphinxbfcode{\sphinxupquote{setKeepSettings}}}{boolean\sphinxstyleemphasis{ keepSettings}}{}
\end{fulllineitems}



\paragraph{setLowTresholdAttempts}
\label{\detokenize{source/it/unicam/cs/pa/mastermind/ui/StartStats:setlowtresholdattempts}}\index{setLowTresholdAttempts(int) (Java method)@\spxentry{setLowTresholdAttempts(int)}\spxextra{Java method}}

\begin{fulllineitems}
\phantomsection\label{\detokenize{source/it/unicam/cs/pa/mastermind/ui/StartStats:it.unicam.cs.pa.mastermind.ui.StartStats.setLowTresholdAttempts(int)}}\pysiglinewithargsret{public void \sphinxbfcode{\sphinxupquote{setLowTresholdAttempts}}}{int\sphinxstyleemphasis{ lowTresholdAttempts}}{}
\end{fulllineitems}



\paragraph{setLowTresholdLength}
\label{\detokenize{source/it/unicam/cs/pa/mastermind/ui/StartStats:setlowtresholdlength}}\index{setLowTresholdLength(int) (Java method)@\spxentry{setLowTresholdLength(int)}\spxextra{Java method}}

\begin{fulllineitems}
\phantomsection\label{\detokenize{source/it/unicam/cs/pa/mastermind/ui/StartStats:it.unicam.cs.pa.mastermind.ui.StartStats.setLowTresholdLength(int)}}\pysiglinewithargsret{public void \sphinxbfcode{\sphinxupquote{setLowTresholdLength}}}{int\sphinxstyleemphasis{ lowTresholdLength}}{}
\end{fulllineitems}



\paragraph{setMakers}
\label{\detokenize{source/it/unicam/cs/pa/mastermind/ui/StartStats:setmakers}}\index{setMakers(MakerFactoryRegistry) (Java method)@\spxentry{setMakers(MakerFactoryRegistry)}\spxextra{Java method}}

\begin{fulllineitems}
\phantomsection\label{\detokenize{source/it/unicam/cs/pa/mastermind/ui/StartStats:it.unicam.cs.pa.mastermind.ui.StartStats.setMakers(MakerFactoryRegistry)}}\pysiglinewithargsret{public void \sphinxbfcode{\sphinxupquote{setMakers}}}{{\hyperref[\detokenize{source/it/unicam/cs/pa/mastermind/players/MakerFactoryRegistry:it.unicam.cs.pa.mastermind.players.MakerFactoryRegistry}]{\sphinxcrossref{MakerFactoryRegistry}}}\sphinxstyleemphasis{ makers}}{}
\end{fulllineitems}



\paragraph{setNewGame}
\label{\detokenize{source/it/unicam/cs/pa/mastermind/ui/StartStats:setnewgame}}\index{setNewGame(NewGameStats) (Java method)@\spxentry{setNewGame(NewGameStats)}\spxextra{Java method}}

\begin{fulllineitems}
\phantomsection\label{\detokenize{source/it/unicam/cs/pa/mastermind/ui/StartStats:it.unicam.cs.pa.mastermind.ui.StartStats.setNewGame(NewGameStats)}}\pysiglinewithargsret{public void \sphinxbfcode{\sphinxupquote{setNewGame}}}{{\hyperref[\detokenize{source/it/unicam/cs/pa/mastermind/gamecore/NewGameStats:it.unicam.cs.pa.mastermind.gamecore.NewGameStats}]{\sphinxcrossref{NewGameStats}}}\sphinxstyleemphasis{ newGame}}{}
\end{fulllineitems}



\paragraph{setSequenceLength}
\label{\detokenize{source/it/unicam/cs/pa/mastermind/ui/StartStats:setsequencelength}}\index{setSequenceLength(int) (Java method)@\spxentry{setSequenceLength(int)}\spxextra{Java method}}

\begin{fulllineitems}
\phantomsection\label{\detokenize{source/it/unicam/cs/pa/mastermind/ui/StartStats:it.unicam.cs.pa.mastermind.ui.StartStats.setSequenceLength(int)}}\pysiglinewithargsret{public void \sphinxbfcode{\sphinxupquote{setSequenceLength}}}{int\sphinxstyleemphasis{ sequenceLength}}{}
\end{fulllineitems}



\paragraph{setToContinue}
\label{\detokenize{source/it/unicam/cs/pa/mastermind/ui/StartStats:settocontinue}}\index{setToContinue(boolean) (Java method)@\spxentry{setToContinue(boolean)}\spxextra{Java method}}

\begin{fulllineitems}
\phantomsection\label{\detokenize{source/it/unicam/cs/pa/mastermind/ui/StartStats:it.unicam.cs.pa.mastermind.ui.StartStats.setToContinue(boolean)}}\pysiglinewithargsret{public void \sphinxbfcode{\sphinxupquote{setToContinue}}}{boolean\sphinxstyleemphasis{ toContinue}}{}
\end{fulllineitems}



\subsection{StartView}
\label{\detokenize{source/it/unicam/cs/pa/mastermind/ui/StartView:startview}}\label{\detokenize{source/it/unicam/cs/pa/mastermind/ui/StartView::doc}}\index{StartView (Java class)@\spxentry{StartView}\spxextra{Java class}}

\begin{fulllineitems}
\phantomsection\label{\detokenize{source/it/unicam/cs/pa/mastermind/ui/StartView:it.unicam.cs.pa.mastermind.ui.StartView}}\pysigline{public abstract class \sphinxbfcode{\sphinxupquote{StartView}}}
\sphinxstylestrong{Responsabilità}: fornire agli utenti fisici coinvolti nel gioco l’interazione per poter iniziare nuove partite.
\begin{quote}\begin{description}
\item[{Author}] \leavevmode
Francesco Pio Stelluti, Francesco Coppola

\end{description}\end{quote}

\end{fulllineitems}



\subsubsection{Fields}
\label{\detokenize{source/it/unicam/cs/pa/mastermind/ui/StartView:fields}}

\paragraph{startStats}
\label{\detokenize{source/it/unicam/cs/pa/mastermind/ui/StartView:startstats}}\index{startStats (Java field)@\spxentry{startStats}\spxextra{Java field}}

\begin{fulllineitems}
\phantomsection\label{\detokenize{source/it/unicam/cs/pa/mastermind/ui/StartView:it.unicam.cs.pa.mastermind.ui.StartView.startStats}}\pysigline{protected {\hyperref[\detokenize{source/it/unicam/cs/pa/mastermind/ui/StartStats:it.unicam.cs.pa.mastermind.ui.StartStats}]{\sphinxcrossref{StartStats}}} \sphinxbfcode{\sphinxupquote{startStats}}}
Istanza della classe \sphinxcode{\sphinxupquote{StartStats}}.

\end{fulllineitems}



\subsubsection{Constructors}
\label{\detokenize{source/it/unicam/cs/pa/mastermind/ui/StartView:constructors}}

\paragraph{StartView}
\label{\detokenize{source/it/unicam/cs/pa/mastermind/ui/StartView:id1}}\index{StartView() (Java constructor)@\spxentry{StartView()}\spxextra{Java constructor}}

\begin{fulllineitems}
\phantomsection\label{\detokenize{source/it/unicam/cs/pa/mastermind/ui/StartView:it.unicam.cs.pa.mastermind.ui.StartView.StartView()}}\pysiglinewithargsret{public \sphinxbfcode{\sphinxupquote{StartView}}}{}{}
\end{fulllineitems}



\subsubsection{Methods}
\label{\detokenize{source/it/unicam/cs/pa/mastermind/ui/StartView:methods}}

\paragraph{askNewAttempts}
\label{\detokenize{source/it/unicam/cs/pa/mastermind/ui/StartView:asknewattempts}}\index{askNewAttempts() (Java method)@\spxentry{askNewAttempts()}\spxextra{Java method}}

\begin{fulllineitems}
\phantomsection\label{\detokenize{source/it/unicam/cs/pa/mastermind/ui/StartView:it.unicam.cs.pa.mastermind.ui.StartView.askNewAttempts()}}\pysiglinewithargsret{protected abstract int \sphinxbfcode{\sphinxupquote{askNewAttempts}}}{}{}
Gestione dell’interazione con l’utente fisico per l’impostazione di un nuovo valore di numero di tentativi massimi richiesti al \sphinxcode{\sphinxupquote{CodeBreaker}} all’interno della nuova partita.
\begin{quote}\begin{description}
\item[{Ritorna}] \leavevmode
int numero di tentativi massimi richiesti al \sphinxcode{\sphinxupquote{CodeBreaker}} all’interno della nuova partita.

\end{description}\end{quote}

\end{fulllineitems}



\paragraph{askNewGameSettings}
\label{\detokenize{source/it/unicam/cs/pa/mastermind/ui/StartView:asknewgamesettings}}\index{askNewGameSettings() (Java method)@\spxentry{askNewGameSettings()}\spxextra{Java method}}

\begin{fulllineitems}
\phantomsection\label{\detokenize{source/it/unicam/cs/pa/mastermind/ui/StartView:it.unicam.cs.pa.mastermind.ui.StartView.askNewGameSettings()}}\pysiglinewithargsret{protected abstract {\hyperref[\detokenize{source/it/unicam/cs/pa/mastermind/gamecore/NewGameStats:it.unicam.cs.pa.mastermind.gamecore.NewGameStats}]{\sphinxcrossref{NewGameStats}}} \sphinxbfcode{\sphinxupquote{askNewGameSettings}}}{}{}
Interazione con l’utente fisico a fronte della conclusione di una singola partita.
\begin{quote}\begin{description}
\item[{Ritorna}] \leavevmode
NewGameStats contenente informazioni relative all’inizio di una nuova partita e alle impostazioni correlate.

\end{description}\end{quote}

\end{fulllineitems}



\paragraph{askNewLength}
\label{\detokenize{source/it/unicam/cs/pa/mastermind/ui/StartView:asknewlength}}\index{askNewLength() (Java method)@\spxentry{askNewLength()}\spxextra{Java method}}

\begin{fulllineitems}
\phantomsection\label{\detokenize{source/it/unicam/cs/pa/mastermind/ui/StartView:it.unicam.cs.pa.mastermind.ui.StartView.askNewLength()}}\pysiglinewithargsret{protected abstract int \sphinxbfcode{\sphinxupquote{askNewLength}}}{}{}
Gestione dell’interazione con l’utente fisico per l’impostazione di un nuovo valore della lunghezza delle sequenze di elementi presenti nella nuova partita.
\begin{quote}\begin{description}
\item[{Ritorna}] \leavevmode
int valore della lunghezza delle sequenze di elementi presenti nella nuova partita.

\end{description}\end{quote}

\end{fulllineitems}



\paragraph{askNewSettings}
\label{\detokenize{source/it/unicam/cs/pa/mastermind/ui/StartView:asknewsettings}}\index{askNewSettings() (Java method)@\spxentry{askNewSettings()}\spxextra{Java method}}

\begin{fulllineitems}
\phantomsection\label{\detokenize{source/it/unicam/cs/pa/mastermind/ui/StartView:it.unicam.cs.pa.mastermind.ui.StartView.askNewSettings()}}\pysiglinewithargsret{protected abstract boolean \sphinxbfcode{\sphinxupquote{askNewSettings}}}{}{}
Gestione dell’interazione con l’utente fisico per l’impostazione o meno di nuove impostazioni relative alla nuova partita.
\begin{quote}\begin{description}
\item[{Ritorna}] \leavevmode
boolean volontà dell’utente fisico di decidere nuove impostazioni per la nuova partita.

\end{description}\end{quote}

\end{fulllineitems}



\paragraph{badEnding}
\label{\detokenize{source/it/unicam/cs/pa/mastermind/ui/StartView:badending}}\index{badEnding(String) (Java method)@\spxentry{badEnding(String)}\spxextra{Java method}}

\begin{fulllineitems}
\phantomsection\label{\detokenize{source/it/unicam/cs/pa/mastermind/ui/StartView:it.unicam.cs.pa.mastermind.ui.StartView.badEnding(String)}}\pysiglinewithargsret{protected abstract void \sphinxbfcode{\sphinxupquote{badEnding}}}{\sphinxhref{http://docs.oracle.com/javase/8/docs/api/java/lang/String.html}{String}\sphinxstyleemphasis{ reason}}{}
Gestione anticipata della conclusione dell’intero gioco, richiamata ad esempio per il sollevamento di errori importanti.
\begin{quote}\begin{description}
\item[{Parametri}] \leavevmode\begin{itemize}
\item {} 
\sphinxstyleliteralstrong{\sphinxupquote{reason}} \textendash{} 

\end{itemize}

\end{description}\end{quote}

\end{fulllineitems}



\paragraph{ending}
\label{\detokenize{source/it/unicam/cs/pa/mastermind/ui/StartView:ending}}\index{ending() (Java method)@\spxentry{ending()}\spxextra{Java method}}

\begin{fulllineitems}
\phantomsection\label{\detokenize{source/it/unicam/cs/pa/mastermind/ui/StartView:it.unicam.cs.pa.mastermind.ui.StartView.ending()}}\pysiglinewithargsret{protected abstract void \sphinxbfcode{\sphinxupquote{ending}}}{}{}
Gestione della conclusione dell’intero gioco dopo la fine di ogni singola partita.

\end{fulllineitems}



\paragraph{getInteractionView}
\label{\detokenize{source/it/unicam/cs/pa/mastermind/ui/StartView:getinteractionview}}\index{getInteractionView() (Java method)@\spxentry{getInteractionView()}\spxextra{Java method}}

\begin{fulllineitems}
\phantomsection\label{\detokenize{source/it/unicam/cs/pa/mastermind/ui/StartView:it.unicam.cs.pa.mastermind.ui.StartView.getInteractionView()}}\pysiglinewithargsret{protected abstract {\hyperref[\detokenize{source/it/unicam/cs/pa/mastermind/ui/InteractionView:it.unicam.cs.pa.mastermind.ui.InteractionView}]{\sphinxcrossref{InteractionView}}} \sphinxbfcode{\sphinxupquote{getInteractionView}}}{}{}
Ottenimento dell’oggetto \sphinxcode{\sphinxupquote{InteractionView}} associato alla particolare implementazione di \sphinxcode{\sphinxupquote{StartView}}.
\begin{quote}\begin{description}
\item[{Ritorna}] \leavevmode
InteractionView associata all’oggetto \sphinxcode{\sphinxupquote{StartView}}.

\end{description}\end{quote}

\end{fulllineitems}



\paragraph{getPlayerName}
\label{\detokenize{source/it/unicam/cs/pa/mastermind/ui/StartView:getplayername}}\index{getPlayerName(PlayerFactoryRegistry, boolean) (Java method)@\spxentry{getPlayerName(PlayerFactoryRegistry, boolean)}\spxextra{Java method}}

\begin{fulllineitems}
\phantomsection\label{\detokenize{source/it/unicam/cs/pa/mastermind/ui/StartView:it.unicam.cs.pa.mastermind.ui.StartView.getPlayerName(PlayerFactoryRegistry, boolean)}}\pysiglinewithargsret{protected abstract \sphinxhref{http://docs.oracle.com/javase/8/docs/api/java/lang/String.html}{String} \sphinxbfcode{\sphinxupquote{getPlayerName}}}{{\hyperref[\detokenize{source/it/unicam/cs/pa/mastermind/players/PlayerFactoryRegistry:it.unicam.cs.pa.mastermind.players.PlayerFactoryRegistry}]{\sphinxcrossref{PlayerFactoryRegistry}}}\sphinxstyleemphasis{ registry}, boolean\sphinxstyleemphasis{ isBreaker}}{}
Gestione dell’interazione dell’utente fisico per la scelta della particolare implementazione dei giocatori che verranno coinvolti nella nuova partita.
\begin{quote}\begin{description}
\item[{Parametri}] \leavevmode\begin{itemize}
\item {} 
\sphinxstyleliteralstrong{\sphinxupquote{registry}} \textendash{} registro contenente le informazioni sulle classi \sphinxcode{\sphinxupquote{PlayerFactory}} relative alle implementazioni dei giocatori.

\item {} 
\sphinxstyleliteralstrong{\sphinxupquote{isBreaker}} \textendash{} flag che indica se la scelta è relativa ad un giocatore \sphinxcode{\sphinxupquote{CodeBreaker}} o meno.

\end{itemize}

\item[{Ritorna}] \leavevmode
String rappresentante l’implementazione del giocatore scelta per la nuova partita.

\end{description}\end{quote}

\end{fulllineitems}



\paragraph{showLogo}
\label{\detokenize{source/it/unicam/cs/pa/mastermind/ui/StartView:showlogo}}\index{showLogo() (Java method)@\spxentry{showLogo()}\spxextra{Java method}}

\begin{fulllineitems}
\phantomsection\label{\detokenize{source/it/unicam/cs/pa/mastermind/ui/StartView:it.unicam.cs.pa.mastermind.ui.StartView.showLogo()}}\pysiglinewithargsret{protected abstract void \sphinxbfcode{\sphinxupquote{showLogo}}}{}{}
Gestione del logo di avvio del gioco.

\end{fulllineitems}



\paragraph{showNewGameStarting}
\label{\detokenize{source/it/unicam/cs/pa/mastermind/ui/StartView:shownewgamestarting}}\index{showNewGameStarting() (Java method)@\spxentry{showNewGameStarting()}\spxextra{Java method}}

\begin{fulllineitems}
\phantomsection\label{\detokenize{source/it/unicam/cs/pa/mastermind/ui/StartView:it.unicam.cs.pa.mastermind.ui.StartView.showNewGameStarting()}}\pysiglinewithargsret{protected abstract void \sphinxbfcode{\sphinxupquote{showNewGameStarting}}}{}{}
Gestione del messaggio di avvio di una singola partita.

\end{fulllineitems}



\paragraph{startUp}
\label{\detokenize{source/it/unicam/cs/pa/mastermind/ui/StartView:startup}}\index{startUp() (Java method)@\spxentry{startUp()}\spxextra{Java method}}

\begin{fulllineitems}
\phantomsection\label{\detokenize{source/it/unicam/cs/pa/mastermind/ui/StartView:it.unicam.cs.pa.mastermind.ui.StartView.startUp()}}\pysiglinewithargsret{public void \sphinxbfcode{\sphinxupquote{startUp}}}{}{}
Gestione completa dell’interazione con l’utente fisico per poter iniziare una nuova partita.

\end{fulllineitems}



\chapter{Test realizzati in JUnit}
\label{\detokenize{test/packages:test-realizzati-in-junit}}\label{\detokenize{test/packages::doc}}
Di seguito è possibile analizzare in maniera dettagliata e scrupolosa quelli che sono
i \sphinxstylestrong{test} che sono stati prodotti per mostrare il corretto funzionamento del progetto.

Essi infatti \sphinxstylestrong{garantiscono oggettivamente} che il codice si comporti come previsto.


\section{it.unicam.cs.pa.mastermind.test}
\label{\detokenize{test/it/unicam/cs/pa/mastermind/test/package-index:it-unicam-cs-pa-mastermind-test}}\label{\detokenize{test/it/unicam/cs/pa/mastermind/test/package-index::doc}}
Il seguente package contiene i vari test che andaranno effettuati all’interno del progetto, per testarne la qualità, la bontà e soprattutto l’efficenza.

\phantomsection\label{\detokenize{test/it/unicam/cs/pa/mastermind/test/package-index:package-it.unicam.cs.pa.mastermind.test}}\index{it.unicam.cs.pa.mastermind.test (package)@\spxentry{it.unicam.cs.pa.mastermind.test}\spxextra{package}}

\subsection{GameCoreBoardControllerTest}
\label{\detokenize{test/it/unicam/cs/pa/mastermind/test/GameCoreBoardControllerTest:gamecoreboardcontrollertest}}\label{\detokenize{test/it/unicam/cs/pa/mastermind/test/GameCoreBoardControllerTest::doc}}\index{GameCoreBoardControllerTest (Java class)@\spxentry{GameCoreBoardControllerTest}\spxextra{Java class}}

\begin{fulllineitems}
\phantomsection\label{\detokenize{test/it/unicam/cs/pa/mastermind/test/GameCoreBoardControllerTest:it.unicam.cs.pa.mastermind.test.GameCoreBoardControllerTest}}\pysigline{ class \sphinxbfcode{\sphinxupquote{GameCoreBoardControllerTest}}}
Test di controllo utili alle meccaniche del coordinatore di gioco.
\begin{quote}\begin{description}
\item[{Author}] \leavevmode
Francesco Pio Stelluti, Francesco Coppola

\end{description}\end{quote}

\end{fulllineitems}



\subsubsection{Fields}
\label{\detokenize{test/it/unicam/cs/pa/mastermind/test/GameCoreBoardControllerTest:fields}}

\paragraph{attempt}
\label{\detokenize{test/it/unicam/cs/pa/mastermind/test/GameCoreBoardControllerTest:attempt}}\index{attempt (Java field)@\spxentry{attempt}\spxextra{Java field}}

\begin{fulllineitems}
\phantomsection\label{\detokenize{test/it/unicam/cs/pa/mastermind/test/GameCoreBoardControllerTest:it.unicam.cs.pa.mastermind.test.GameCoreBoardControllerTest.attempt}}\pysigline{ \sphinxhref{http://docs.oracle.com/javase/8/docs/api/java/util/List.html}{List}\textless{}{\hyperref[\detokenize{source/it/unicam/cs/pa/mastermind/gamecore/ColorPegs:it.unicam.cs.pa.mastermind.gamecore.ColorPegs}]{\sphinxcrossref{ColorPegs}}}\textgreater{} \sphinxbfcode{\sphinxupquote{attempt}}}
\end{fulllineitems}



\paragraph{toGuess}
\label{\detokenize{test/it/unicam/cs/pa/mastermind/test/GameCoreBoardControllerTest:toguess}}\index{toGuess (Java field)@\spxentry{toGuess}\spxextra{Java field}}

\begin{fulllineitems}
\phantomsection\label{\detokenize{test/it/unicam/cs/pa/mastermind/test/GameCoreBoardControllerTest:it.unicam.cs.pa.mastermind.test.GameCoreBoardControllerTest.toGuess}}\pysigline{ \sphinxhref{http://docs.oracle.com/javase/8/docs/api/java/util/List.html}{List}\textless{}{\hyperref[\detokenize{source/it/unicam/cs/pa/mastermind/gamecore/ColorPegs:it.unicam.cs.pa.mastermind.gamecore.ColorPegs}]{\sphinxcrossref{ColorPegs}}}\textgreater{} \sphinxbfcode{\sphinxupquote{toGuess}}}
\end{fulllineitems}



\subsubsection{Methods}
\label{\detokenize{test/it/unicam/cs/pa/mastermind/test/GameCoreBoardControllerTest:methods}}

\paragraph{setUp}
\label{\detokenize{test/it/unicam/cs/pa/mastermind/test/GameCoreBoardControllerTest:setup}}\index{setUp() (Java method)@\spxentry{setUp()}\spxextra{Java method}}

\begin{fulllineitems}
\phantomsection\label{\detokenize{test/it/unicam/cs/pa/mastermind/test/GameCoreBoardControllerTest:it.unicam.cs.pa.mastermind.test.GameCoreBoardControllerTest.setUp()}}\pysiglinewithargsret{ void \sphinxbfcode{\sphinxupquote{setUp}}}{}{}
Setup of the board runned before each other test.

\end{fulllineitems}



\paragraph{testBoardController}
\label{\detokenize{test/it/unicam/cs/pa/mastermind/test/GameCoreBoardControllerTest:testboardcontroller}}\index{testBoardController() (Java method)@\spxentry{testBoardController()}\spxextra{Java method}}

\begin{fulllineitems}
\phantomsection\label{\detokenize{test/it/unicam/cs/pa/mastermind/test/GameCoreBoardControllerTest:it.unicam.cs.pa.mastermind.test.GameCoreBoardControllerTest.testBoardController()}}\pysiglinewithargsret{ void \sphinxbfcode{\sphinxupquote{testBoardController}}}{}{}
Test method for \sphinxhref{https://docs.oracle.com/en/java/javase/12/docs/api/index.html/it/unicam/cs/pa/mastermind/gamecore/BoardController.html\#BoardController(it.unicam.cs.pa.mastermind.gamecore.BoardModel)}{\sphinxcode{\sphinxupquote{it.unicam.cs.pa.mastermind.gamecore.BoardController.BoardController(it.unicam.cs.pa.mastermind.gamecore.BoardModel)}}}.

\end{fulllineitems}



\paragraph{testGetBoardReference}
\label{\detokenize{test/it/unicam/cs/pa/mastermind/test/GameCoreBoardControllerTest:testgetboardreference}}\index{testGetBoardReference() (Java method)@\spxentry{testGetBoardReference()}\spxextra{Java method}}

\begin{fulllineitems}
\phantomsection\label{\detokenize{test/it/unicam/cs/pa/mastermind/test/GameCoreBoardControllerTest:it.unicam.cs.pa.mastermind.test.GameCoreBoardControllerTest.testGetBoardReference()}}\pysiglinewithargsret{ void \sphinxbfcode{\sphinxupquote{testGetBoardReference}}}{}{}
Test method for {\hyperref[\detokenize{source/it/unicam/cs/pa/mastermind/gamecore/BoardController:it.unicam.cs.pa.mastermind.gamecore.BoardController.getBoardReference()}]{\sphinxcrossref{\sphinxcode{\sphinxupquote{it.unicam.cs.pa.mastermind.gamecore.BoardController.getBoardReference()}}}}}.

\end{fulllineitems}



\paragraph{testGetSequenceLength}
\label{\detokenize{test/it/unicam/cs/pa/mastermind/test/GameCoreBoardControllerTest:testgetsequencelength}}\index{testGetSequenceLength() (Java method)@\spxentry{testGetSequenceLength()}\spxextra{Java method}}

\begin{fulllineitems}
\phantomsection\label{\detokenize{test/it/unicam/cs/pa/mastermind/test/GameCoreBoardControllerTest:it.unicam.cs.pa.mastermind.test.GameCoreBoardControllerTest.testGetSequenceLength()}}\pysiglinewithargsret{ void \sphinxbfcode{\sphinxupquote{testGetSequenceLength}}}{}{}
Test method for \sphinxhref{https://docs.oracle.com/en/java/javase/12/docs/api/index.html/it/unicam/cs/pa/mastermind/gamecore/BoardController.html\#getSequenceLength()}{\sphinxcode{\sphinxupquote{it.unicam.cs.pa.mastermind.gamecore.BoardController.getSequenceLength()}}}.

\end{fulllineitems}



\paragraph{testGetSequenceToGuess}
\label{\detokenize{test/it/unicam/cs/pa/mastermind/test/GameCoreBoardControllerTest:testgetsequencetoguess}}\index{testGetSequenceToGuess() (Java method)@\spxentry{testGetSequenceToGuess()}\spxextra{Java method}}

\begin{fulllineitems}
\phantomsection\label{\detokenize{test/it/unicam/cs/pa/mastermind/test/GameCoreBoardControllerTest:it.unicam.cs.pa.mastermind.test.GameCoreBoardControllerTest.testGetSequenceToGuess()}}\pysiglinewithargsret{ void \sphinxbfcode{\sphinxupquote{testGetSequenceToGuess}}}{}{}
Test method for \sphinxhref{https://docs.oracle.com/en/java/javase/12/docs/api/index.html/it/unicam/cs/pa/mastermind/gamecore/BoardController.html\#getSequenceToGuess()}{\sphinxcode{\sphinxupquote{it.unicam.cs.pa.mastermind.gamecore.BoardController.getSequenceToGuess()}}}.

\end{fulllineitems}



\paragraph{testInsertCodeToGuess}
\label{\detokenize{test/it/unicam/cs/pa/mastermind/test/GameCoreBoardControllerTest:testinsertcodetoguess}}\index{testInsertCodeToGuess() (Java method)@\spxentry{testInsertCodeToGuess()}\spxextra{Java method}}

\begin{fulllineitems}
\phantomsection\label{\detokenize{test/it/unicam/cs/pa/mastermind/test/GameCoreBoardControllerTest:it.unicam.cs.pa.mastermind.test.GameCoreBoardControllerTest.testInsertCodeToGuess()}}\pysiglinewithargsret{ void \sphinxbfcode{\sphinxupquote{testInsertCodeToGuess}}}{}{}
Test method for \sphinxhref{https://docs.oracle.com/en/java/javase/12/docs/api/index.html/it/unicam/cs/pa/mastermind/gamecore/BoardController.html\#insertCodeToGuess(java.util.List)}{\sphinxcode{\sphinxupquote{it.unicam.cs.pa.mastermind.gamecore.BoardController.insertCodeToGuess(java.util.List)}}}.

\end{fulllineitems}



\paragraph{testInsertNewAttempt}
\label{\detokenize{test/it/unicam/cs/pa/mastermind/test/GameCoreBoardControllerTest:testinsertnewattempt}}\index{testInsertNewAttempt() (Java method)@\spxentry{testInsertNewAttempt()}\spxextra{Java method}}

\begin{fulllineitems}
\phantomsection\label{\detokenize{test/it/unicam/cs/pa/mastermind/test/GameCoreBoardControllerTest:it.unicam.cs.pa.mastermind.test.GameCoreBoardControllerTest.testInsertNewAttempt()}}\pysiglinewithargsret{ void \sphinxbfcode{\sphinxupquote{testInsertNewAttempt}}}{}{}
Test method for \sphinxhref{https://docs.oracle.com/en/java/javase/12/docs/api/index.html/it/unicam/cs/pa/mastermind/gamecore/BoardController.html\#insertNewAttempt(java.util.List)}{\sphinxcode{\sphinxupquote{it.unicam.cs.pa.mastermind.gamecore.BoardController.insertNewAttempt(java.util.List)}}}.

\end{fulllineitems}



\subsection{GameCoreBoardModelTest}
\label{\detokenize{test/it/unicam/cs/pa/mastermind/test/GameCoreBoardModelTest:gamecoreboardmodeltest}}\label{\detokenize{test/it/unicam/cs/pa/mastermind/test/GameCoreBoardModelTest::doc}}\index{GameCoreBoardModelTest (Java class)@\spxentry{GameCoreBoardModelTest}\spxextra{Java class}}

\begin{fulllineitems}
\phantomsection\label{\detokenize{test/it/unicam/cs/pa/mastermind/test/GameCoreBoardModelTest:it.unicam.cs.pa.mastermind.test.GameCoreBoardModelTest}}\pysigline{ class \sphinxbfcode{\sphinxupquote{GameCoreBoardModelTest}}}
Test di controllo all’interno della board.
\begin{quote}\begin{description}
\item[{Author}] \leavevmode
Francesco Pio Stelluti, Francesco Coppola

\end{description}\end{quote}

\end{fulllineitems}



\subsubsection{Fields}
\label{\detokenize{test/it/unicam/cs/pa/mastermind/test/GameCoreBoardModelTest:fields}}

\paragraph{attempt}
\label{\detokenize{test/it/unicam/cs/pa/mastermind/test/GameCoreBoardModelTest:attempt}}\index{attempt (Java field)@\spxentry{attempt}\spxextra{Java field}}

\begin{fulllineitems}
\phantomsection\label{\detokenize{test/it/unicam/cs/pa/mastermind/test/GameCoreBoardModelTest:it.unicam.cs.pa.mastermind.test.GameCoreBoardModelTest.attempt}}\pysigline{ \sphinxhref{http://docs.oracle.com/javase/8/docs/api/java/util/List.html}{List}\textless{}{\hyperref[\detokenize{source/it/unicam/cs/pa/mastermind/gamecore/ColorPegs:it.unicam.cs.pa.mastermind.gamecore.ColorPegs}]{\sphinxcrossref{ColorPegs}}}\textgreater{} \sphinxbfcode{\sphinxupquote{attempt}}}
\end{fulllineitems}



\paragraph{toGuess}
\label{\detokenize{test/it/unicam/cs/pa/mastermind/test/GameCoreBoardModelTest:toguess}}\index{toGuess (Java field)@\spxentry{toGuess}\spxextra{Java field}}

\begin{fulllineitems}
\phantomsection\label{\detokenize{test/it/unicam/cs/pa/mastermind/test/GameCoreBoardModelTest:it.unicam.cs.pa.mastermind.test.GameCoreBoardModelTest.toGuess}}\pysigline{ \sphinxhref{http://docs.oracle.com/javase/8/docs/api/java/util/List.html}{List}\textless{}{\hyperref[\detokenize{source/it/unicam/cs/pa/mastermind/gamecore/ColorPegs:it.unicam.cs.pa.mastermind.gamecore.ColorPegs}]{\sphinxcrossref{ColorPegs}}}\textgreater{} \sphinxbfcode{\sphinxupquote{toGuess}}}
\end{fulllineitems}



\subsubsection{Methods}
\label{\detokenize{test/it/unicam/cs/pa/mastermind/test/GameCoreBoardModelTest:methods}}

\paragraph{setUp}
\label{\detokenize{test/it/unicam/cs/pa/mastermind/test/GameCoreBoardModelTest:setup}}\index{setUp() (Java method)@\spxentry{setUp()}\spxextra{Java method}}

\begin{fulllineitems}
\phantomsection\label{\detokenize{test/it/unicam/cs/pa/mastermind/test/GameCoreBoardModelTest:it.unicam.cs.pa.mastermind.test.GameCoreBoardModelTest.setUp()}}\pysiglinewithargsret{ void \sphinxbfcode{\sphinxupquote{setUp}}}{}{}
Setup of the board runned before each other test.

\end{fulllineitems}



\paragraph{testAddAttempt}
\label{\detokenize{test/it/unicam/cs/pa/mastermind/test/GameCoreBoardModelTest:testaddattempt}}\index{testAddAttempt() (Java method)@\spxentry{testAddAttempt()}\spxextra{Java method}}

\begin{fulllineitems}
\phantomsection\label{\detokenize{test/it/unicam/cs/pa/mastermind/test/GameCoreBoardModelTest:it.unicam.cs.pa.mastermind.test.GameCoreBoardModelTest.testAddAttempt()}}\pysiglinewithargsret{ void \sphinxbfcode{\sphinxupquote{testAddAttempt}}}{}{}
Test method for \sphinxhref{https://docs.oracle.com/en/java/javase/12/docs/api/index.html/it/unicam/cs/pa/mastermind/gamecore/BoardModel.html\#addAttempt(java.util.List,java.util.List)}{\sphinxcode{\sphinxupquote{it.unicam.cs.pa.mastermind.gamecore.BoardModel.addAttempt(java.util.List,java.util.List)}}}.

\end{fulllineitems}



\paragraph{testAttemptsInserted}
\label{\detokenize{test/it/unicam/cs/pa/mastermind/test/GameCoreBoardModelTest:testattemptsinserted}}\index{testAttemptsInserted() (Java method)@\spxentry{testAttemptsInserted()}\spxextra{Java method}}

\begin{fulllineitems}
\phantomsection\label{\detokenize{test/it/unicam/cs/pa/mastermind/test/GameCoreBoardModelTest:it.unicam.cs.pa.mastermind.test.GameCoreBoardModelTest.testAttemptsInserted()}}\pysiglinewithargsret{ void \sphinxbfcode{\sphinxupquote{testAttemptsInserted}}}{}{}
Test method for {\hyperref[\detokenize{source/it/unicam/cs/pa/mastermind/gamecore/BoardModel:it.unicam.cs.pa.mastermind.gamecore.BoardModel.attemptsInserted()}]{\sphinxcrossref{\sphinxcode{\sphinxupquote{it.unicam.cs.pa.mastermind.gamecore.BoardModel.attemptsInserted()}}}}}.

\end{fulllineitems}



\paragraph{testBoard}
\label{\detokenize{test/it/unicam/cs/pa/mastermind/test/GameCoreBoardModelTest:testboard}}\index{testBoard() (Java method)@\spxentry{testBoard()}\spxextra{Java method}}

\begin{fulllineitems}
\phantomsection\label{\detokenize{test/it/unicam/cs/pa/mastermind/test/GameCoreBoardModelTest:it.unicam.cs.pa.mastermind.test.GameCoreBoardModelTest.testBoard()}}\pysiglinewithargsret{ void \sphinxbfcode{\sphinxupquote{testBoard}}}{}{}
Test method for \sphinxhref{https://docs.oracle.com/en/java/javase/12/docs/api/index.html/it/unicam/cs/pa/mastermind/gamecore/BoardModel.html\#Board(int,int)}{\sphinxcode{\sphinxupquote{it.unicam.cs.pa.mastermind.gamecore.BoardModel.Board(int,int)}}}.

\end{fulllineitems}



\paragraph{testIsEmpty}
\label{\detokenize{test/it/unicam/cs/pa/mastermind/test/GameCoreBoardModelTest:testisempty}}\index{testIsEmpty() (Java method)@\spxentry{testIsEmpty()}\spxextra{Java method}}

\begin{fulllineitems}
\phantomsection\label{\detokenize{test/it/unicam/cs/pa/mastermind/test/GameCoreBoardModelTest:it.unicam.cs.pa.mastermind.test.GameCoreBoardModelTest.testIsEmpty()}}\pysiglinewithargsret{ void \sphinxbfcode{\sphinxupquote{testIsEmpty}}}{}{}
Test method for \sphinxhref{https://docs.oracle.com/en/java/javase/12/docs/api/index.html/it/unicam/cs/pa/mastermind/gamecore/BoardModel.html\#isEmpty()}{\sphinxcode{\sphinxupquote{it.unicam.cs.pa.mastermind.gamecore.BoardModel.isEmpty()}}}.

\end{fulllineitems}



\paragraph{testLastAttemptAndClue}
\label{\detokenize{test/it/unicam/cs/pa/mastermind/test/GameCoreBoardModelTest:testlastattemptandclue}}\index{testLastAttemptAndClue() (Java method)@\spxentry{testLastAttemptAndClue()}\spxextra{Java method}}

\begin{fulllineitems}
\phantomsection\label{\detokenize{test/it/unicam/cs/pa/mastermind/test/GameCoreBoardModelTest:it.unicam.cs.pa.mastermind.test.GameCoreBoardModelTest.testLastAttemptAndClue()}}\pysiglinewithargsret{ void \sphinxbfcode{\sphinxupquote{testLastAttemptAndClue}}}{}{}
Test method for {\hyperref[\detokenize{source/it/unicam/cs/pa/mastermind/gamecore/BoardModel:it.unicam.cs.pa.mastermind.gamecore.BoardModel.lastAttemptAndClue()}]{\sphinxcrossref{\sphinxcode{\sphinxupquote{it.unicam.cs.pa.mastermind.gamecore.BoardModel.lastAttemptAndClue()}}}}}.

\end{fulllineitems}



\paragraph{testLeftAttempts}
\label{\detokenize{test/it/unicam/cs/pa/mastermind/test/GameCoreBoardModelTest:testleftattempts}}\index{testLeftAttempts() (Java method)@\spxentry{testLeftAttempts()}\spxextra{Java method}}

\begin{fulllineitems}
\phantomsection\label{\detokenize{test/it/unicam/cs/pa/mastermind/test/GameCoreBoardModelTest:it.unicam.cs.pa.mastermind.test.GameCoreBoardModelTest.testLeftAttempts()}}\pysiglinewithargsret{ void \sphinxbfcode{\sphinxupquote{testLeftAttempts}}}{}{}
Test method for {\hyperref[\detokenize{source/it/unicam/cs/pa/mastermind/gamecore/BoardModel:it.unicam.cs.pa.mastermind.gamecore.BoardModel.leftAttempts()}]{\sphinxcrossref{\sphinxcode{\sphinxupquote{it.unicam.cs.pa.mastermind.gamecore.BoardModel.leftAttempts()}}}}}.

\end{fulllineitems}



\paragraph{testSetSequenceToGuess}
\label{\detokenize{test/it/unicam/cs/pa/mastermind/test/GameCoreBoardModelTest:testsetsequencetoguess}}\index{testSetSequenceToGuess() (Java method)@\spxentry{testSetSequenceToGuess()}\spxextra{Java method}}

\begin{fulllineitems}
\phantomsection\label{\detokenize{test/it/unicam/cs/pa/mastermind/test/GameCoreBoardModelTest:it.unicam.cs.pa.mastermind.test.GameCoreBoardModelTest.testSetSequenceToGuess()}}\pysiglinewithargsret{ void \sphinxbfcode{\sphinxupquote{testSetSequenceToGuess}}}{}{}
Test method for \sphinxhref{https://docs.oracle.com/en/java/javase/12/docs/api/index.html/it/unicam/cs/pa/mastermind/gamecore/BoardModel.html\#setSequenceToGuess(java.util.List)}{\sphinxcode{\sphinxupquote{it.unicam.cs.pa.mastermind.gamecore.BoardModel.setSequenceToGuess(java.util.List)}}}.

\end{fulllineitems}



\subsection{PlayersFactoryRegistry}
\label{\detokenize{test/it/unicam/cs/pa/mastermind/test/PlayersFactoryRegistry:playersfactoryregistry}}\label{\detokenize{test/it/unicam/cs/pa/mastermind/test/PlayersFactoryRegistry::doc}}\index{PlayersFactoryRegistry (Java class)@\spxentry{PlayersFactoryRegistry}\spxextra{Java class}}

\begin{fulllineitems}
\phantomsection\label{\detokenize{test/it/unicam/cs/pa/mastermind/test/PlayersFactoryRegistry:it.unicam.cs.pa.mastermind.test.PlayersFactoryRegistry}}\pysigline{ class \sphinxbfcode{\sphinxupquote{PlayersFactoryRegistry}}}
Test di controllo utili alla generazione delle factory relativi ai player.
\begin{quote}\begin{description}
\item[{Author}] \leavevmode
Francesco Pio Stelluti, Francesco Coppola

\end{description}\end{quote}

\end{fulllineitems}



\subsubsection{Fields}
\label{\detokenize{test/it/unicam/cs/pa/mastermind/test/PlayersFactoryRegistry:fields}}

\paragraph{playersFactory}
\label{\detokenize{test/it/unicam/cs/pa/mastermind/test/PlayersFactoryRegistry:playersfactory}}\index{playersFactory (Java field)@\spxentry{playersFactory}\spxextra{Java field}}

\begin{fulllineitems}
\phantomsection\label{\detokenize{test/it/unicam/cs/pa/mastermind/test/PlayersFactoryRegistry:it.unicam.cs.pa.mastermind.test.PlayersFactoryRegistry.playersFactory}}\pysigline{ \sphinxhref{http://docs.oracle.com/javase/8/docs/api/java/util/List.html}{List}\textless{}\sphinxhref{http://docs.oracle.com/javase/8/docs/api/java/lang/String.html}{String}\textgreater{} \sphinxbfcode{\sphinxupquote{playersFactory}}}
\end{fulllineitems}



\subsubsection{Methods}
\label{\detokenize{test/it/unicam/cs/pa/mastermind/test/PlayersFactoryRegistry:methods}}

\paragraph{testBreakerFactoryRegistry}
\label{\detokenize{test/it/unicam/cs/pa/mastermind/test/PlayersFactoryRegistry:testbreakerfactoryregistry}}\index{testBreakerFactoryRegistry() (Java method)@\spxentry{testBreakerFactoryRegistry()}\spxextra{Java method}}

\begin{fulllineitems}
\phantomsection\label{\detokenize{test/it/unicam/cs/pa/mastermind/test/PlayersFactoryRegistry:it.unicam.cs.pa.mastermind.test.PlayersFactoryRegistry.testBreakerFactoryRegistry()}}\pysiglinewithargsret{ void \sphinxbfcode{\sphinxupquote{testBreakerFactoryRegistry}}}{}{}
Test method for \sphinxhref{https://docs.oracle.com/en/java/javase/12/docs/api/index.html/it/unicam/cs/pa/mastermind/players/BreakerFactoryRegistry.html\#BreakerFactoryRegistry()}{\sphinxcode{\sphinxupquote{it.unicam.cs.pa.mastermind.players.BreakerFactoryRegistry.BreakerFactoryRegistry()}}}.
\begin{quote}\begin{description}
\item[{Solleva}] \leavevmode\begin{itemize}
\item {} 
{\hyperref[\detokenize{source/it/unicam/cs/pa/mastermind/players/BadRegistryException:it.unicam.cs.pa.mastermind.players.BadRegistryException}]{\sphinxcrossref{\sphinxstyleliteralstrong{\sphinxupquote{BadRegistryException}}}}} \textendash{} 

\end{itemize}

\end{description}\end{quote}

\end{fulllineitems}



\paragraph{testCheckRightPathName}
\label{\detokenize{test/it/unicam/cs/pa/mastermind/test/PlayersFactoryRegistry:testcheckrightpathname}}\index{testCheckRightPathName() (Java method)@\spxentry{testCheckRightPathName()}\spxextra{Java method}}

\begin{fulllineitems}
\phantomsection\label{\detokenize{test/it/unicam/cs/pa/mastermind/test/PlayersFactoryRegistry:it.unicam.cs.pa.mastermind.test.PlayersFactoryRegistry.testCheckRightPathName()}}\pysiglinewithargsret{ void \sphinxbfcode{\sphinxupquote{testCheckRightPathName}}}{}{}
Test method for the check of the existence of the path name passed in the constructor.
\begin{quote}\begin{description}
\item[{Solleva}] \leavevmode\begin{itemize}
\item {} 
{\hyperref[\detokenize{source/it/unicam/cs/pa/mastermind/players/BadRegistryException:it.unicam.cs.pa.mastermind.players.BadRegistryException}]{\sphinxcrossref{\sphinxstyleliteralstrong{\sphinxupquote{BadRegistryException}}}}} \textendash{} 

\item {} 
\sphinxstyleliteralstrong{\sphinxupquote{IOException}} \textendash{} 

\end{itemize}

\end{description}\end{quote}

\end{fulllineitems}



\paragraph{testGetFactoryByName}
\label{\detokenize{test/it/unicam/cs/pa/mastermind/test/PlayersFactoryRegistry:testgetfactorybyname}}\index{testGetFactoryByName() (Java method)@\spxentry{testGetFactoryByName()}\spxextra{Java method}}

\begin{fulllineitems}
\phantomsection\label{\detokenize{test/it/unicam/cs/pa/mastermind/test/PlayersFactoryRegistry:it.unicam.cs.pa.mastermind.test.PlayersFactoryRegistry.testGetFactoryByName()}}\pysiglinewithargsret{ void \sphinxbfcode{\sphinxupquote{testGetFactoryByName}}}{}{}
Test method for \sphinxhref{https://docs.oracle.com/en/java/javase/12/docs/api/index.html/it/unicam/cs/pa/mastermind/players/PlayerFactoryRegistry.html\#getFactoryByName(java.lang.String)}{\sphinxcode{\sphinxupquote{it.unicam.cs.pa.mastermind.players.PlayerFactoryRegistry.getFactoryByName(java.lang.String)}}}.
\begin{quote}\begin{description}
\item[{Solleva}] \leavevmode\begin{itemize}
\item {} 
{\hyperref[\detokenize{source/it/unicam/cs/pa/mastermind/players/BadRegistryException:it.unicam.cs.pa.mastermind.players.BadRegistryException}]{\sphinxcrossref{\sphinxstyleliteralstrong{\sphinxupquote{BadRegistryException}}}}} \textendash{} 

\end{itemize}

\end{description}\end{quote}

\end{fulllineitems}



\paragraph{testGetPlayersNames}
\label{\detokenize{test/it/unicam/cs/pa/mastermind/test/PlayersFactoryRegistry:testgetplayersnames}}\index{testGetPlayersNames() (Java method)@\spxentry{testGetPlayersNames()}\spxextra{Java method}}

\begin{fulllineitems}
\phantomsection\label{\detokenize{test/it/unicam/cs/pa/mastermind/test/PlayersFactoryRegistry:it.unicam.cs.pa.mastermind.test.PlayersFactoryRegistry.testGetPlayersNames()}}\pysiglinewithargsret{ void \sphinxbfcode{\sphinxupquote{testGetPlayersNames}}}{}{}
Test method for {\hyperref[\detokenize{source/it/unicam/cs/pa/mastermind/players/PlayerFactoryRegistry:it.unicam.cs.pa.mastermind.players.PlayerFactoryRegistry.getPlayersNames()}]{\sphinxcrossref{\sphinxcode{\sphinxupquote{it.unicam.cs.pa.mastermind.players.PlayerFactoryRegistry.getPlayersNames()}}}}}.
\begin{quote}\begin{description}
\item[{Solleva}] \leavevmode\begin{itemize}
\item {} 
{\hyperref[\detokenize{source/it/unicam/cs/pa/mastermind/players/BadRegistryException:it.unicam.cs.pa.mastermind.players.BadRegistryException}]{\sphinxcrossref{\sphinxstyleliteralstrong{\sphinxupquote{BadRegistryException}}}}} \textendash{} 

\end{itemize}

\end{description}\end{quote}

\end{fulllineitems}



\paragraph{testMakerFactoryRegistry}
\label{\detokenize{test/it/unicam/cs/pa/mastermind/test/PlayersFactoryRegistry:testmakerfactoryregistry}}\index{testMakerFactoryRegistry() (Java method)@\spxentry{testMakerFactoryRegistry()}\spxextra{Java method}}

\begin{fulllineitems}
\phantomsection\label{\detokenize{test/it/unicam/cs/pa/mastermind/test/PlayersFactoryRegistry:it.unicam.cs.pa.mastermind.test.PlayersFactoryRegistry.testMakerFactoryRegistry()}}\pysiglinewithargsret{ void \sphinxbfcode{\sphinxupquote{testMakerFactoryRegistry}}}{}{}
Test method for \sphinxhref{https://docs.oracle.com/en/java/javase/12/docs/api/index.html/it/unicam/cs/pa/mastermind/players/MakerFactoryRegistry.html\#MakerFactoryRegistry()}{\sphinxcode{\sphinxupquote{it.unicam.cs.pa.mastermind.players.MakerFactoryRegistry.MakerFactoryRegistry()}}}.
\begin{quote}\begin{description}
\item[{Solleva}] \leavevmode\begin{itemize}
\item {} 
{\hyperref[\detokenize{source/it/unicam/cs/pa/mastermind/players/BadRegistryException:it.unicam.cs.pa.mastermind.players.BadRegistryException}]{\sphinxcrossref{\sphinxstyleliteralstrong{\sphinxupquote{BadRegistryException}}}}} \textendash{} 

\end{itemize}

\end{description}\end{quote}

\end{fulllineitems}



\subsection{PlayersInteractiveBreakerTest}
\label{\detokenize{test/it/unicam/cs/pa/mastermind/test/PlayersInteractiveBreakerTest:playersinteractivebreakertest}}\label{\detokenize{test/it/unicam/cs/pa/mastermind/test/PlayersInteractiveBreakerTest::doc}}\index{PlayersInteractiveBreakerTest (Java class)@\spxentry{PlayersInteractiveBreakerTest}\spxextra{Java class}}

\begin{fulllineitems}
\phantomsection\label{\detokenize{test/it/unicam/cs/pa/mastermind/test/PlayersInteractiveBreakerTest:it.unicam.cs.pa.mastermind.test.PlayersInteractiveBreakerTest}}\pysigline{ class \sphinxbfcode{\sphinxupquote{PlayersInteractiveBreakerTest}}}
Test di controllo utili alla generazione di un player decodficatore di natura umana.
\begin{quote}\begin{description}
\item[{Author}] \leavevmode
Francesco Pio Stelluti, Francesco Coppola

\end{description}\end{quote}

\end{fulllineitems}



\subsubsection{Methods}
\label{\detokenize{test/it/unicam/cs/pa/mastermind/test/PlayersInteractiveBreakerTest:methods}}

\paragraph{testGetAttempt}
\label{\detokenize{test/it/unicam/cs/pa/mastermind/test/PlayersInteractiveBreakerTest:testgetattempt}}\index{testGetAttempt() (Java method)@\spxentry{testGetAttempt()}\spxextra{Java method}}

\begin{fulllineitems}
\phantomsection\label{\detokenize{test/it/unicam/cs/pa/mastermind/test/PlayersInteractiveBreakerTest:it.unicam.cs.pa.mastermind.test.PlayersInteractiveBreakerTest.testGetAttempt()}}\pysiglinewithargsret{ void \sphinxbfcode{\sphinxupquote{testGetAttempt}}}{}{}
Test method for \sphinxhref{https://docs.oracle.com/en/java/javase/12/docs/api/index.html/it/unicam/cs/pa/mastermind/players/InteractiveBreaker.html\#getAttempt(int,it.unicam.cs.pa.mastermind.ui.InteractionView)}{\sphinxcode{\sphinxupquote{it.unicam.cs.pa.mastermind.players.InteractiveBreaker.getAttempt(int,it.unicam.cs.pa.mastermind.ui.InteractionView)}}}.

\end{fulllineitems}



\paragraph{testInteractiveBreaker}
\label{\detokenize{test/it/unicam/cs/pa/mastermind/test/PlayersInteractiveBreakerTest:testinteractivebreaker}}\index{testInteractiveBreaker() (Java method)@\spxentry{testInteractiveBreaker()}\spxextra{Java method}}

\begin{fulllineitems}
\phantomsection\label{\detokenize{test/it/unicam/cs/pa/mastermind/test/PlayersInteractiveBreakerTest:it.unicam.cs.pa.mastermind.test.PlayersInteractiveBreakerTest.testInteractiveBreaker()}}\pysiglinewithargsret{ void \sphinxbfcode{\sphinxupquote{testInteractiveBreaker}}}{}{}
Test method for {\hyperref[\detokenize{source/it/unicam/cs/pa/mastermind/players/InteractiveBreaker:it.unicam.cs.pa.mastermind.players.InteractiveBreaker.InteractiveBreaker()}]{\sphinxcrossref{\sphinxcode{\sphinxupquote{it.unicam.cs.pa.mastermind.players.InteractiveBreaker.InteractiveBreaker()}}}}}.

\end{fulllineitems}



\subsection{PlayersInteractiveMakerTest}
\label{\detokenize{test/it/unicam/cs/pa/mastermind/test/PlayersInteractiveMakerTest:playersinteractivemakertest}}\label{\detokenize{test/it/unicam/cs/pa/mastermind/test/PlayersInteractiveMakerTest::doc}}\index{PlayersInteractiveMakerTest (Java class)@\spxentry{PlayersInteractiveMakerTest}\spxextra{Java class}}

\begin{fulllineitems}
\phantomsection\label{\detokenize{test/it/unicam/cs/pa/mastermind/test/PlayersInteractiveMakerTest:it.unicam.cs.pa.mastermind.test.PlayersInteractiveMakerTest}}\pysigline{ class \sphinxbfcode{\sphinxupquote{PlayersInteractiveMakerTest}}}
Test di controllo utili alla generazione di un player codficatore di natura umana.
\begin{quote}\begin{description}
\item[{Author}] \leavevmode
Francesco Pio Stelluti, Francesco Coppola

\end{description}\end{quote}

\end{fulllineitems}



\subsubsection{Methods}
\label{\detokenize{test/it/unicam/cs/pa/mastermind/test/PlayersInteractiveMakerTest:methods}}

\paragraph{testGetCodeToGuess}
\label{\detokenize{test/it/unicam/cs/pa/mastermind/test/PlayersInteractiveMakerTest:testgetcodetoguess}}\index{testGetCodeToGuess() (Java method)@\spxentry{testGetCodeToGuess()}\spxextra{Java method}}

\begin{fulllineitems}
\phantomsection\label{\detokenize{test/it/unicam/cs/pa/mastermind/test/PlayersInteractiveMakerTest:it.unicam.cs.pa.mastermind.test.PlayersInteractiveMakerTest.testGetCodeToGuess()}}\pysiglinewithargsret{ void \sphinxbfcode{\sphinxupquote{testGetCodeToGuess}}}{}{}
Test method for \sphinxhref{https://docs.oracle.com/en/java/javase/12/docs/api/index.html/it/unicam/cs/pa/mastermind/players/InteractiveMaker.html\#getCodeToGuess(int,it.unicam.cs.pa.mastermind.ui.InteractionView)}{\sphinxcode{\sphinxupquote{it.unicam.cs.pa.mastermind.players.InteractiveMaker.getCodeToGuess(int,it.unicam.cs.pa.mastermind.ui.InteractionView)}}}.

\end{fulllineitems}



\subsection{PlayersRandomBotBreakerTest}
\label{\detokenize{test/it/unicam/cs/pa/mastermind/test/PlayersRandomBotBreakerTest:playersrandombotbreakertest}}\label{\detokenize{test/it/unicam/cs/pa/mastermind/test/PlayersRandomBotBreakerTest::doc}}\index{PlayersRandomBotBreakerTest (Java class)@\spxentry{PlayersRandomBotBreakerTest}\spxextra{Java class}}

\begin{fulllineitems}
\phantomsection\label{\detokenize{test/it/unicam/cs/pa/mastermind/test/PlayersRandomBotBreakerTest:it.unicam.cs.pa.mastermind.test.PlayersRandomBotBreakerTest}}\pysigline{ class \sphinxbfcode{\sphinxupquote{PlayersRandomBotBreakerTest}}}
Test di controllo utili alla generazione di un player decodficatore di natura bot.
\begin{quote}\begin{description}
\item[{Author}] \leavevmode
Francesco Pio Stelluti, Francesco Coppola

\end{description}\end{quote}

\end{fulllineitems}



\subsubsection{Methods}
\label{\detokenize{test/it/unicam/cs/pa/mastermind/test/PlayersRandomBotBreakerTest:methods}}

\paragraph{testGetAttempt}
\label{\detokenize{test/it/unicam/cs/pa/mastermind/test/PlayersRandomBotBreakerTest:testgetattempt}}\index{testGetAttempt() (Java method)@\spxentry{testGetAttempt()}\spxextra{Java method}}

\begin{fulllineitems}
\phantomsection\label{\detokenize{test/it/unicam/cs/pa/mastermind/test/PlayersRandomBotBreakerTest:it.unicam.cs.pa.mastermind.test.PlayersRandomBotBreakerTest.testGetAttempt()}}\pysiglinewithargsret{ void \sphinxbfcode{\sphinxupquote{testGetAttempt}}}{}{}
Test method for \sphinxhref{https://docs.oracle.com/en/java/javase/12/docs/api/index.html/it/unicam/cs/pa/mastermind/players/RandomBotBreaker.html\#getAttempt(int,it.unicam.cs.pa.mastermind.ui.InteractionManager)}{\sphinxcode{\sphinxupquote{it.unicam.cs.pa.mastermind.players.RandomBotBreaker.getAttempt(int,it.unicam.cs.pa.mastermind.ui.InteractionManager)}}}.

\end{fulllineitems}



\subsection{PlayersRandomBotMakerTest}
\label{\detokenize{test/it/unicam/cs/pa/mastermind/test/PlayersRandomBotMakerTest:playersrandombotmakertest}}\label{\detokenize{test/it/unicam/cs/pa/mastermind/test/PlayersRandomBotMakerTest::doc}}\index{PlayersRandomBotMakerTest (Java class)@\spxentry{PlayersRandomBotMakerTest}\spxextra{Java class}}

\begin{fulllineitems}
\phantomsection\label{\detokenize{test/it/unicam/cs/pa/mastermind/test/PlayersRandomBotMakerTest:it.unicam.cs.pa.mastermind.test.PlayersRandomBotMakerTest}}\pysigline{ class \sphinxbfcode{\sphinxupquote{PlayersRandomBotMakerTest}}}
Test di controllo utili alla generazione di un player codficatore di natura bot.
\begin{quote}\begin{description}
\item[{Author}] \leavevmode
Francesco Pio Stelluti, Francesco Coppola

\end{description}\end{quote}

\end{fulllineitems}



\subsubsection{Methods}
\label{\detokenize{test/it/unicam/cs/pa/mastermind/test/PlayersRandomBotMakerTest:methods}}

\paragraph{testGetCodeToGuess}
\label{\detokenize{test/it/unicam/cs/pa/mastermind/test/PlayersRandomBotMakerTest:testgetcodetoguess}}\index{testGetCodeToGuess() (Java method)@\spxentry{testGetCodeToGuess()}\spxextra{Java method}}

\begin{fulllineitems}
\phantomsection\label{\detokenize{test/it/unicam/cs/pa/mastermind/test/PlayersRandomBotMakerTest:it.unicam.cs.pa.mastermind.test.PlayersRandomBotMakerTest.testGetCodeToGuess()}}\pysiglinewithargsret{ void \sphinxbfcode{\sphinxupquote{testGetCodeToGuess}}}{}{}
Test method for \sphinxhref{https://docs.oracle.com/en/java/javase/12/docs/api/index.html/it/unicam/cs/pa/mastermind/players/RandomBotMaker.html\#getCodeToGuess(int,it.unicam.cs.pa.mastermind.ui.InteractionManager)}{\sphinxcode{\sphinxupquote{it.unicam.cs.pa.mastermind.players.RandomBotMaker.getCodeToGuess(int,it.unicam.cs.pa.mastermind.ui.InteractionManager)}}}.

\end{fulllineitems}



\subsection{SimulationGame}
\label{\detokenize{test/it/unicam/cs/pa/mastermind/test/SimulationGame:simulationgame}}\label{\detokenize{test/it/unicam/cs/pa/mastermind/test/SimulationGame::doc}}\index{SimulationGame (Java class)@\spxentry{SimulationGame}\spxextra{Java class}}

\begin{fulllineitems}
\phantomsection\label{\detokenize{test/it/unicam/cs/pa/mastermind/test/SimulationGame:it.unicam.cs.pa.mastermind.test.SimulationGame}}\pysigline{ class \sphinxbfcode{\sphinxupquote{SimulationGame}}}
Il seguente test simula il corretto funzionamento di una singola partita.
\begin{quote}\begin{description}
\item[{Author}] \leavevmode
Francesco Pio Stelluti, Francesco Coppola

\end{description}\end{quote}

\end{fulllineitems}



\subsubsection{Methods}
\label{\detokenize{test/it/unicam/cs/pa/mastermind/test/SimulationGame:methods}}

\paragraph{testSimulationGame}
\label{\detokenize{test/it/unicam/cs/pa/mastermind/test/SimulationGame:testsimulationgame}}\index{testSimulationGame() (Java method)@\spxentry{testSimulationGame()}\spxextra{Java method}}

\begin{fulllineitems}
\phantomsection\label{\detokenize{test/it/unicam/cs/pa/mastermind/test/SimulationGame:it.unicam.cs.pa.mastermind.test.SimulationGame.testSimulationGame()}}\pysiglinewithargsret{ void \sphinxbfcode{\sphinxupquote{testSimulationGame}}}{}{}
\end{fulllineitems}



\subsection{UIConsoleInteractionViewTest}
\label{\detokenize{test/it/unicam/cs/pa/mastermind/test/UIConsoleInteractionViewTest:uiconsoleinteractionviewtest}}\label{\detokenize{test/it/unicam/cs/pa/mastermind/test/UIConsoleInteractionViewTest::doc}}\index{UIConsoleInteractionViewTest (Java class)@\spxentry{UIConsoleInteractionViewTest}\spxextra{Java class}}

\begin{fulllineitems}
\phantomsection\label{\detokenize{test/it/unicam/cs/pa/mastermind/test/UIConsoleInteractionViewTest:it.unicam.cs.pa.mastermind.test.UIConsoleInteractionViewTest}}\pysigline{ class \sphinxbfcode{\sphinxupquote{UIConsoleInteractionViewTest}}}
Test di controllo utili al check dell’unica instanza della classe sotto esamina.
\begin{quote}\begin{description}
\item[{Author}] \leavevmode
Francesco Pio Stelluti, Francesco Coppola

\end{description}\end{quote}

\end{fulllineitems}



\subsubsection{Methods}
\label{\detokenize{test/it/unicam/cs/pa/mastermind/test/UIConsoleInteractionViewTest:methods}}

\paragraph{testGetIstance}
\label{\detokenize{test/it/unicam/cs/pa/mastermind/test/UIConsoleInteractionViewTest:testgetistance}}\index{testGetIstance() (Java method)@\spxentry{testGetIstance()}\spxextra{Java method}}

\begin{fulllineitems}
\phantomsection\label{\detokenize{test/it/unicam/cs/pa/mastermind/test/UIConsoleInteractionViewTest:it.unicam.cs.pa.mastermind.test.UIConsoleInteractionViewTest.testGetIstance()}}\pysiglinewithargsret{ void \sphinxbfcode{\sphinxupquote{testGetIstance}}}{}{}
Test method for {\hyperref[\detokenize{source/it/unicam/cs/pa/mastermind/ui/ConsoleInteractionView:it.unicam.cs.pa.mastermind.ui.ConsoleInteractionView.getInstance()}]{\sphinxcrossref{\sphinxcode{\sphinxupquote{it.unicam.cs.pa.mastermind.ui.ConsoleInteractionView.getInstance()}}}}}.

\end{fulllineitems}



\subsection{UIConsoleStartViewTest}
\label{\detokenize{test/it/unicam/cs/pa/mastermind/test/UIConsoleStartViewTest:uiconsolestartviewtest}}\label{\detokenize{test/it/unicam/cs/pa/mastermind/test/UIConsoleStartViewTest::doc}}\index{UIConsoleStartViewTest (Java class)@\spxentry{UIConsoleStartViewTest}\spxextra{Java class}}

\begin{fulllineitems}
\phantomsection\label{\detokenize{test/it/unicam/cs/pa/mastermind/test/UIConsoleStartViewTest:it.unicam.cs.pa.mastermind.test.UIConsoleStartViewTest}}\pysigline{ class \sphinxbfcode{\sphinxupquote{UIConsoleStartViewTest}}}
Test di controllo utili al check dell’unica instanza della classe sotto esamina.
\begin{quote}\begin{description}
\item[{Author}] \leavevmode
Francesco Pio Stelluti, Francesco Coppola

\end{description}\end{quote}

\end{fulllineitems}



\subsubsection{Methods}
\label{\detokenize{test/it/unicam/cs/pa/mastermind/test/UIConsoleStartViewTest:methods}}

\paragraph{testGetIstance}
\label{\detokenize{test/it/unicam/cs/pa/mastermind/test/UIConsoleStartViewTest:testgetistance}}\index{testGetIstance() (Java method)@\spxentry{testGetIstance()}\spxextra{Java method}}

\begin{fulllineitems}
\phantomsection\label{\detokenize{test/it/unicam/cs/pa/mastermind/test/UIConsoleStartViewTest:it.unicam.cs.pa.mastermind.test.UIConsoleStartViewTest.testGetIstance()}}\pysiglinewithargsret{ void \sphinxbfcode{\sphinxupquote{testGetIstance}}}{}{}
Test method for {\hyperref[\detokenize{source/it/unicam/cs/pa/mastermind/ui/ConsoleStartView:it.unicam.cs.pa.mastermind.ui.ConsoleStartView.getInstance()}]{\sphinxcrossref{\sphinxcode{\sphinxupquote{it.unicam.cs.pa.mastermind.ui.ConsoleStartView.getInstance()}}}}}.

\end{fulllineitems}




\renewcommand{\indexname}{Indice}
\printindex
\end{document}